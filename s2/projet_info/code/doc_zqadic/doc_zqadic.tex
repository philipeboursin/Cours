\documentclass[11pt]{article}

\usepackage{import}
\import{D:/Bureau/Documents/Maths/Latex/Packages/}{article.tex}

% Couleur de correction
% \newcommand{\cor}[1]{{#1}}
\newcommand{\cor}[1]{{\color{red} #1}}

\begin{document}

%Page de garde
\title{Documentation pour \mintinline{c}{n2adic.h}}
\date{\today}
\author{Guillemot Alexandre, Soumier Julien}
\maketitle

%Table des matières
\tableofcontents

%Document
\section{Introduction}
/* Soit $p$ un nombre premier et $q = p^d \in \mathbb{Z}$. Ce module permet de faire des calculs sur $\mathbb{Z}_{q}$, en représentant l'extension comme un quotient de $\mathbb{Z}_p[X]$ par un polynôme $M \in \mathbb{F}_2[X]$ irréductible.*/


\section{Structure de données}
Un élément de $\mathbb{Z}_q$ est la classe d'équivalence d'un élément de $\mathbb{Z}_p[X]$ modulo $M$. On le représente donc par un élément de $\mathbb{Z}_p[X]$, que l'on réduira modulo $M$ tout au long des calculs. On dira qu'il est sous forme réduite s'il est réduit modulo $M$

Type représentant un élément de $\mathbb{Z}_q$.
\begin{minted}{c}
    typedef padic_poly_t n2adic_t;
\end{minted}

Fonction renvoyant la précision à laquelle \mintinline{c}{x} est représenté.
\begin{minted}{c}
    slong n2adic_prec(n2adic_t x);
\end{minted}

Fonction renvoyant la valuation de \mintinline{c}{x}.
\begin{minted}{c}
    slong n2adic_val(n2adic_t x);
\end{minted}


\section{Contexte}
/* Un contexte d'entiers $q$-adiques contient les informations nécessaires aux calculs dans $\mathbb{Z}_q$, ainsi que différents éléments précalculés permettant d'accélérer certains calculs. */

/* Différents types de polynômes pouvant représenter l'extension $\mathbb{Q}_q$ de $\mathbb{Q}_p$. */
enum rep_type {TEICHMULLER, SPARSE};

/* Type représentant un contexte d'entiers $q$-adiques. */
typedef struct _n2adic_ctx_t
{
    slong prec; // Précision maximale des calculs dans l'extension, dans le cas où le représentant est calculé à une précision donnée (e.g. module de Teichmuller)
    slong deg; // Degré de l'extension
    enum rep_type type; // Type de représentant
    fmpz_t p; // Nombre premier tel que $p^{deg} = q$
    n2adic_t* C;// Pointeur vers un tableau contenant les éléments $C_j \in Z_q$. Reste null si le type n'est pas TEICHMULLER
    padic_ctx_t pctx; // Contexte $p$-adique associé au sous-corps de l'extension
    padic_poly_t M; // Polynôme représentant de l'extension
} n2adic_ctx_t[1];

/* Procédure permettant d'initialiser un contexte \mintinline{c}{n2adic_ctx} à partir d'un polynôme \mintinline{c}{M} $\in \mathbb{Z}_p$ (supposé irréductible), dans un contexte \mintinline{c}{padic_ctx}. La précision maximale de l'extension sera donnée par la précision de \mintinline{c}{M} si \mintinline{c}{type == TEICHMULLER}. */
void n2adic_ctx_init_padic_poly(n2adic_ctx_t n2adic_ctx, padic_poly_t M, padic_ctx_t padic_ctx, enum rep_type type);

/* Procédure calculant le module de Teichmuller de \mintinline{c}{m} $\in \mathbb{F}_p[X]$, vu comme un polynôme de $\mathbb{Z}_p[X]$, à précision \mintinline{c}{N}. Le résultat est mis dans \mintinline{c}{M}". /!\ Ne marche qu'avec $p = 2$ (dans le contexte) /!\ */
void _teichmuller_modulus(padic_poly_t M, padic_poly_t m, slong N, padic_ctx_t C);

/* Procédure permettant d'initialiser un contexte \mintinline{c}{n2adic_ctx}, avec comme représentant le module de Teichmuller de \mintinline{c}{m} $\in \mathbb{F}_p[X]$ vu comme un polynôme de $\mathbb{Z}[X]$. Les informations \mintinline{c}{min}, \mintinline{c}{max} et \mintinline{c}{mode} permettent d'initialiser le contexte $p$-adique dans lequel seront représentés les coefficients des polynômes représentant les éléments de $\mathbb{Z}_q$ (voir padic.h). /!\ Ne fonctionne qu'avec $p = 2$ /!\  */
void _n2adic_ctx_init_teichmuller(n2adic_ctx_t n2adic_ctx, fmpz_poly_t m, slong prec, slong min, slong max, enum padic_print_mode mode);

/* Procédure permettant d'initlaiser un contexte \mintinline{c}{n2adic_ctx}, avec comme un représentant le module de Teichmuller d'un polynôme alétoire pris dans $\mathbb{F}_p[X]$. Les informations \mintinline{c}{min}, \mintinline{c}{max} et \mintinline{c}{mode} permettent d'initialiser le contexte $p$-adique dans lequel seront représentés les coefficients des polynômes représentant les éléments de $\mathbb{Z}_q$ (voir padic.h). /!\ Ne fonctionne qu'avec $p = 2$ /!\ */
void n2adic_ctx_init_teichmuller(n2adic_ctx_t n2adic_ctx, slong deg, slong prec, slong min, slong max, enum padic_print_mode mode);

/* Procédure permettant d'initlaiser un contexte \mintinline{c}{n2adic_ctx}, avec comme représentant le relèvememnt creux de \mintinline{c}{m} $\in \mathbb{F}_p[X]$ vu comme un polynôme de $\mathbb{Z}[X]$. Les informations \mintinline{c}{min}, \mintinline{c}{max} et \mintinline{c}{mode} permettent d'initialiser le contexte $p$-adique dans lequel seront représentés les coefficients des polynômes représentant les éléments de $\mathbb{Z}_q$ (voir padic.h). */
void _n2adic_ctx_init_sparse(n2adic_ctx_t n2adic_ctx, fmpz_poly_t m, fmpz_t p, slong prec, slong min, slong max, enum padic_print_mode mode);

/* Procédure permettant d'initlaiser un contexte \mintinline{c}{n2adic_ctx}, avec comme un représentant le relèvement creux d'un polynôme alétoire pris dans $\mathbb{F}_p[X]$. Les informations \mintinline{c}{min}, \mintinline{c}{max} et \mintinline{c}{mode} permettent d'initialiser le contexte $p$-adique dans lequel seront représentés les coefficients des polynômes représentant les éléments de $\mathbb{Z}_q$ (voir padic.h). */
void n2adic_ctx_init_sparse(n2adic_ctx_t n2adic_ctx, fmpz_t p, slong deg, slong prec, slong min, slong max, enum padic_print_mode mode);

/* Procédure permettant de récupérer le représentant d'un contexte d'entiers $q$-adiques \mintinline{c}{n2adic_ctx_t}. Met le résultat dans \mintinline{c}{P}. */
void n2adic_ctx_rep(padic_poly_t P, n2adic_ctx_t ctx);


\section{Gestion de la mémoire}

/* Permet d'initialiser la mémoire nécessaire pour un \mintinline{c}{x} $\in \mathbb{Z}_q$. La précision par défaut est donnée par la précision du contexte \mintinline{c}{n2adic_ctx}. */
void n2adic_init(n2adic_t x, n2adic_ctx_t n2adic_ctx);

/* Permet d'initialiser la mémoire nécessaire opur un \mintinline{c}{x} $\in \mathbb{Z}_q$, à précision \mintinline{c}{prec}. */
void n2adic_init2(n2adic_t x, slong prec, n2adic_ctx_t n2adic_ctx);

/* Permet de libérer la mémoire allouée pour \mintinline{c}{x}. */
void n2adic_clear(n2adic_t x);

/* Permet de libérer la mémoire allouée pour \mintinline{c}{ctx} un contexte d'entiers q-adiques. */
void n2adic_ctx_clear(n2adic_ctx_t ctx);


\section{Assignement}

/* Met la valeur de \mintinline{c}{op} dans \mintinline{c}{rop}. */
void n2adic_set(n2adic_t rop, n2adic_t op, n2adic_ctx_t n2adic_ctx);

/* Met dans \mintinline{c}{rop} le représentant réduit modulo le polynôme représentant $\mathbb{Z}_q$ de \mintinline{c}{op}. */
void n2adic_set_padic_poly(n2adic_t rop, padic_poly_t op, n2adic_ctx_t n2adic_ctx);

/* Met dans \mintinline{c}{rop} le représentant réduit modulo le polynôme représentant $\mathbb{Z}_q$ de l'inclusion canonique de \mintinline{c}{op} $\in \mathbb{Z}[X]$ dans $\mathbb{Z}_p[X]$. */
void n2adic_set_fmpz_poly(n2adic_t rop, fmpz_poly_t op, n2adic_ctx_t n2adic_ctx);

/* Met dans rop le relèvement canonique de \mintinline{c}{op} $\in \mathbb{Z}_q$, vu comme un élément de $\mathbb{Z}_p[X]$ à précision donnée, donc un élément de $(\mathbb{Z}/p^{prec} \mathbb{Z})[X]$. */
void n2adic_get_fmpz_poly(fmpz_poly_t rop, n2adic_t op, n2adic_ctx_t n2adic_ctx);

/* Met $1$ dans \mintinline{c}{rop}. */
void n2adic_one(n2adic_t rop);

/* Met $0$ dans \mintinline{c}{rop}. */
void n2adic_zero(n2adic_t rop);


\section{Randomisation}

/* Génère un élément de $\mathbb{Z}_q$ aléatoire. Met le résultat dans \mintinline{c}{x}. */
void n2adic_randtest(n2adic_t x, flint_rand_t state, n2adic_ctx_t ctx);


\section{Comparaison}

/* Renvoie \mintinline{c}{1} si et seulement si \mintinline{c}{x} $=$ \mintinline{c}{y}. */
int n2adic_equal(n2adic_t x, n2adic_t y);

/* Renvoie \mintinline{c}{1} si et seulement si \mintinline{c}{x} $= 1$. */
int n2adic_is_one(n2adic_t x);


\section{Opérations arithmétiques}
void _padic_poly_div_eucl(padic_poly_t A, padic_poly_t B, padic_poly_t R, padic_poly_t Q, padic_ctx_t C);

/* Met sous forme réduite \mintinline{c}{x} $\in \mathbb{Z}_q$. */
void n2adic_reduce(n2adic_t x, n2adic_ctx_t C);

/* Additionne \mintinline{c}{op1} et \mintinline{c}{op2}. Met le résultat dans rop. */
void n2adic_add(n2adic_t rop, n2adic_t op1, n2adic_t op2, n2adic_ctx_t ctx);

/* Réalise la soustration de \mintinline{c}{op1} avec \mintinline{c}{op2}. Met le résultat dans \mintinline{c}{rop}. */
void n2adic_sub(n2adic_t rop, n2adic_t op1, n2adic_t op2, n2adic_ctx_t ctx);

/* Réalise la multiplication de \mintinline{c}{op1} avec \mintinline{c}{op2}. Met le résultat dans \mintinline{c}{rop}. */
void n2adic_mul(n2adic_t rop, n2adic_t op1, n2adic_t op2, n2adic_ctx_t ctx);

/* Inverse \mintinline{c}{op}, en supposant qu'il est inversible. Met le résultat dans \mintinline{c}{rop}. */
void n2adic_inv(n2adic_t rop, n2adic_t op, n2adic_ctx_t n2adic_ctx);

/* Met \mintinline{c}{op} à la puissance \mintinline{c}{e} dans rop. */
void n2adic_pow(n2adic_t rop, n2adic_t op, fmpz_t e, n2adic_ctx_t ctx);


\section{Fonctions spéciales}

/* Réalise la substitution du frobenius en \mintinline{c}{op}, dans l'extension spécifiée par \mintinline{c}{ctx}. Met le résultat dans \mintinline{c}{rop}. */
void n2adic_frobenius_substitution(n2adic_t rop, n2adic_t op, n2adic_ctx_t ctx);

/* Réalise la substitution du frobenius inverse en \mintinline{c}{op}, dans l'extension spécifiée par \mintinline{c}{ctx}. Met le résulta dans \mintinline{c}{rop}. */
void n2adic_inv_frobenius_substitution(n2adic_t rop, n2adic_t op, n2adic_ctx_t ctx);

/* Résout l'équation d'Artin-Schreier avec paramètres \mintinline{c}{alpha}, \mintinline{c}{beta} et \mintinline{c}{gamma}. Met le résultat dans \mintinline{c}{x}. */
void n2adic_artin_schreier_root(n2adic_t x, n2adic_t alpha, n2adic_t beta, n2adic_t gamma, n2adic_ctx_t ctx);


\section{Misc}
/* Affiche un \mintinline{c}{x} de $\mathbb{Z}_q$, représenté comme un élément de $\mathbb{Z}_p[X]$. */
void n2adic_print(n2adic_t x, n2adic_ctx_t n2adic_ctx);


%Bibliographie
% \nocite{*}
% \bibliographystyle{D:/Bureau/Documents/Maths/Latex/custom.bst} 
% \bibliography{D:/Bureau/Documents/Maths/Latex/references.bib}

\end{document}

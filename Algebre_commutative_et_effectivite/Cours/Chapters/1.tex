\chapter{Préliminaires sur les anneaux de polynômes, idéaux, noethérianité}
    Dans ce chapitre, tous les anneaux seront commutatifs. Fixons dès à présent un $k \in \mathbf{Fld}$ (on supposera toujours qu'on dispose d'algorithmes pour les opérations du corps). 
    \section{Anneaux noéthériens}
        \subsection{Définition}
            \begin{defi} (Anneau noéthérien)
                Un anneau est noéthérien si toute suite croissante d'idéaux $I_0 \subseteq I_1 \subseteq I_2 \subseteq \cdots$ est stationnaire i.e. 
                \begin{align*}
                    \exists N \in \mathbb{N} \mid \forall m \geq N,\, I_m = I_N 
                \end{align*}
            \end{defi}
            \begin{prop}
                Un anneau est noéthérien si et seulement si tout idéal de $A$ est finiment engendré. 
            \end{prop}
            \begin{expl}
                Voici des exemples d'anneaux noéthériens/non noéthériens
                \begin{figure}[H]
                    \centering
                    \begin{tabular}{c|c}
                        Anneaux noéthériens & Anneaux non noéthériens \\
                        \hline
                        $\mathbb{Q}$ & $k[\mathbb{N}]$ \\
                        Plus généralement, tout corps $k$ & \\
                        $\mathbb{R}[x]$ & \\
                        Plus généralement, tout PID & \\
                        $\mathbb{Z}$ & \\
                        $k[x_1, \cdots, x_n]$ (conséquence de \ref{base_de_hilbert}) & \\
                        Anneaux finis & \\
                        Anneaux artiniens & \\
                    \end{tabular}
                \end{figure}
            \end{expl}

        \subsection{Théorème de la base de hilbert}
            \begin{theo} (Théorème de la base de Hilbert)
                \label{base_de_hilbert}
                Soit $A$ un anneau noéthérien. Alors $A[x]$ est un anneau noéthérien.
            \end{theo}
            \begin{coro}
                Si $k$ est un corps, alors $k[x_1, \cdots, x_n]$ est noeth pour $n \in \mathbb{N}$.
            \end{coro}
            \begin{proof}
                On veut montrer que tout idéal $I \subrel{id} A[x]$ est finiment engendré. Soit $I \subrel{id} A[x]$, montrons qu'il est finiment engendré. Pour chaque $n \in \mathbb{N}$, soit
                \begin{align*}
                    I_n := \{a_n \in A \mid \exists a_0 + a_1x + \cdots + a_n x^n \in I\}
                \end{align*}
                Il est facile de voir que $I_n \subrel{id} A$. Ensuite $(I_i)$ est croissante, car si $a_i \in I_i$ pour un $i \in \mathbb{N}$, alors $\exists f \in I$ tq le coefficient directeur de $f$ soit $a_i$. Mais alors $xf(x) \in I$ est de degré $i+1$ et son coefficient directeur est encore $a_i$, d'où $a_i \in I_{i+1}$. Ainsi cette suite d'idéaux est stationnaire ($A$ noeth). Notons $N \in \mathbb{N}$ tq $m \geq N \Rightarrow I_m = I_N$. Les idéaux $I_0, \cdots, I_N$ sont finiment engendrés, notons $\{a_{i,j}\}_{1 \leq j \leq r_i}$ des familles génératrices pour $I_i$, pour tout $i \in \lcc 0, N \rcc$. Pour chaque $a_{i,j}$, $\exists f_{ij} \in I$ tq $\mathrm{deg}(f_{ij}) \leq i$ et le terme de degré $i$ de $f_{i,j}$ est $a_{i,j}$ (par définition de $I_i$). Montrons que $I = (\{f_{i,j}\}_{0,1 \leq i,j \leq N,r_i})$ : soit $f \in I$,
                \begin{enumerate}
                    \item si $\mathrm{deg}(f) = 0$, alors posons $a \in A$ tq $f = ax^0$. Ainsi $a \in I_0$, ainsi $\exists b_1, \cdots, b_{r_0}$ tq $a = \sum_{i = 1}^{r_0} b_i a_{0,i}$. Or $f_{0,i} = a_{0,i}x^0$, ainsi $f = \sum_{i = 1}^{r_0} b_i f_{0,i}$.
                    \item Si $d = \mathrm{deg} f > 0$, notons $b$ le coeff directeur de $f$. Ainsi $b \in I_d$ \\
                    \textbf{Cas où $d \leq N$ :} On peut écrire $b = \sum_{i = 1}^{r_d} \lambda_i a_{d,i}$ avec $\lambda_i \in A$. Posons $S = \sum_{i = 1}^{r_d} \lambda_i f_{d,i}$, alors le coefficient directeur de $S$ est précisément $b$ (et $\mathrm{deg} S \leq d$). Ainsi $\mathrm{deg} (f-S) < d$, et $f - S \in I$. Par hypothèse de récurrence, $f - S \in (\{f_{i,j}\})$ et $S \in (\{f_{i,j}\})$, donc finalement $f \in (\{f_{i,j}\})$. \\
                    \textbf{Cas où $d > N$ :} Notons $b$ le coeff directeur de $f$, $b \in I_d = I_N \Rightarrow b = \sum \lambda_i a_{N,i}$. Posons $T := \sum \lambda_i f_{N,i}X^{d-N}$ est de degré $d$ et de coeff directeur $b$, puis on conclut comme précedemment en regardant le polynômes $f - T$.
                \end{enumerate}
                Ainsi les idéaux de $A[x]$ sont finiment engendrés, donc $A[x]$ est noeth.
            \end{proof}

    \section{Division multivariée}
        \subsection{Ordres monomiaux}
            Fixons $k \in \mathbf{Fld}$. Rappelons que si $I \subrel{id} k[x]$ non nul, alors $\exists g \in k[x]$ t.q. $I = (g)$ (car $k[x]$ est principal, euclidien). Soit $f \in k[x]$, alors $f \in (g) \iff g \mid f \iff $ le reste de la division euclidienne de $f$ par $g$ est nul (et on dispose d'un algorithme pour réaliser la division euclidienne). Question : peut-on généraliser à $k[x_1, \cdots, x_n]$ ? 
            \begin{remq}
                Soit $I \subrel{id} k[x]$, $I = (f_1, \cdots, f_r)$. Alors $I = (\mathrm{pgcd}(f_1, \cdots, f_r))$
            \end{remq}
            \begin{defi} (Ordre monomial)
                \label{ordre_mono}
                Un ordre monomial sur $k[x_1, \cdots, x_n]$ est une relation d'ordre $\leq$ sur l'ensemble des $\{x^\alpha = x_1^{\alpha_1} \cdots x_n^{\alpha_n} \mid \alpha \in \mathbb{N}^n\}$ tq
                \begin{enumerate}
                    \item $\leq$ est un ordre total (pour tout $x^\alpha, x^\beta \in k[x_1, \cdots, x_n]$, $(x^\alpha \leq x^\beta) \lor (x^\beta \leq x^\alpha)$).
                    \item $x^\alpha \leq x^\beta \Rightarrow \forall \gamma \in \mathbb{N}^n,\, x^{\alpha + \gamma} \leq x^{\beta + \gamma}$
                    \item $1 \leq x^\alpha$ pour tout $\alpha \in \mathbb{N}^n$.
                \end{enumerate}
            \end{defi}
            \begin{nota}
                On écrira $\alpha \leq \beta$ au lieu de $x^\alpha \leq x^\beta$.
            \end{nota}
            \begin{expl}
                \begin{enumerate}
                    \item Dans $k[x]$, il est facile de vérifier qu'il n'existe qu'un seul ordre monomial $\leq$ : $x^n \leq x^m \iff n \leq m$.
                    \item Ordre lexicographique $\leq_{lex}$ : soient $\alpha, \beta \in \mathbb{N}^n$ tq $\alpha \neq \beta$,
                    \begin{align*}
                        \alpha <_{lex} \beta \iff \exists 1 \leq r \leq n \mid \alpha_i = \beta_i \text{ pour } i < r \text{ et } \alpha_r < \beta_r
                    \end{align*}
                    (i.e. le premier coeff non nul d $\beta - \alpha$ est positif). Par exemple, dans $k[x_1, x_2, x_3]$, $x_1^2 >_{lex} x_1x_2 >_{lex} x_2^2 >_{lex} x_3^{2097434}$
                    \item Ordre lexicographique gradué $\leq_{deglex}$ : Pour $\alpha \in \mathbb{N}^n$, notons $|\alpha| = \sum \alpha_i$. Alors soient $\alpha \neq \beta$ dans $\mathbb{N}^n$,
                    \begin{align*}
                        \alpha <_{deglex} \beta \iff (|\alpha| < |\beta|) \lor (|\alpha| = |\beta| \land \alpha <_{lex} \beta)
                    \end{align*}
                    \item Ordre lexicographique renversé gradué $<_{degrevlex}$ :
                    \begin{align*}
                        \alpha <_{degrevlex} \beta \iff (|\alpha| < |\beta|) \lor (|\alpha| = |\beta| \land (\exists r \in \lcc 1,n \rcc  \mid \forall i \in \lcc r+1, n \rcc,\, \alpha_i = \beta_i  \text{ et } \alpha_r > \beta_r))
                    \end{align*}
                    (la deuxième condition reviens a vérifier que le dernier coeff non nul de $\beta - \alpha$ est négatif dans le cas où $|\alpha| = |\beta|$)
                \end{enumerate}
            \end{expl} \noindent
            \begin{exo}
                Vérifier que ces ordres sont des ordres monomiaux.
            \end{exo} \noindent
            Dans sage, on appelle "term orders" de tels ordres.
            \begin{prop}
                Soit $\leq$ un ordre sur $\mathbb{N}^n$ satisfaisant les propriétés $1$ et $2$ de la def \ref{ordre_mono}. Alors tfae
                \begin{enumerate}\addtocounter{enumi}{2}
                    \item $0_{\mathbb{N}^n} \leq \alpha ,\, \forall \alpha \in \mathbb{N}^n$
                    \item $\leq$ est un bon ordre : $\forall E \subseteq \mathbb{N}^n$ non vide, $E$ contient un élément minimal pour $<$.
                \end{enumerate}
            \end{prop}
            \cor{
            \begin{proof}
                4 $\Rightarrow$ 3 : Supposons qu'il existe $\alpha \in \mathbb{N}^n$ tq $\alpha < 0$, alors $2\alpha < \alpha$, $3\alpha < 2\alpha$ et ainsi de suite, donc $\cdots < 2\alpha < \alpha < 0$, mais alors $\{m\alpha \mid m \in \mathbb{N}\}$ n'a pas d'élément minimal, donc $\leq$ n'est pas un bon ordre. \\
                3 $\Rightarrow$ 4 : Supposons qu'il existe $F \subseteq \mathbb{N}^n$ non vide et sans élément minimal. Posons 
                \begin{align*}
                    m_1 = \min \{\alpha_1 \mid \alpha \in F\}
                \end{align*}
                et notons $\alpha^{(1)} \in F$ tq $\alpha^{(1)}_1 = m_1$. Posons de plus
                \begin{align*}
                    F_1 = \{\beta \in F \mid \beta \leq \alpha^{(1)}\}
                \end{align*}
                Remarquons alors que $F_1$ est non vide (il contient $\alpha^{(1)}$). Construisons maintenant $m_i$, $\alpha^{(i)}$ et $F_i$ par récurrence : supposons que l'on a construit $F_{i-1}$ non vide, alors on constuit $m_i$ comme
                \begin{align*}
                    m_i := \min \{\alpha_i \mid \alpha \in F_{i-1}\}
                \end{align*}
                Il existe alors $\alpha^{(i)} \in F_{i-1}$ tq $\alpha^{(i)}_i = m_i$, puis finalement on construit $F_i$ comme
                \begin{align*}
                    F_i := \{\beta \in F_{i-1} \mid \beta \leq \alpha^{(i)}\}
                \end{align*}
                Remarquons finalement que $F_i$ est encore non vide, puisqu'il contiens $\alpha^{(i)}$. Maintenant $F_n$ n'admet pas d'élément minimal, car sinon en notant $\beta$ un tel élément, et prenons $\gamma \in F$. Alors $\gamma \leq \beta$ implique que $\gamma$ est dans $F_n$, puisque $\gamma \leq \beta \leq \alpha^{(n)}$, et ainsi $\gamma = \beta$ par minimalité de $\beta$ dans $F_n$. Ainsi $\beta$ serait un élément minimal de $F$, qui n'en admet pas. Ainsi il existe $\beta \in F_n$ tel que $\beta < \alpha^{(n)}$. Maintenant comme $\alpha^{(n)} \leq \alpha^{(n-1)} \leq \cdots \leq \alpha^{(1)}$, on a $F_n \subseteq F_{n-1} \subseteq \cdots \subseteq F_0 := F$, et donc pour tout $i \in \lcc 1,n \rcc ,\, \beta \in F_{i-1}$.

                Posons maintenant $m_2 = \min \{\alpha_2 \mid \alpha \in F_1\}$, et prenons $\alpha^{(2)} \in F_1$ tq $\alpha_2^{(2)} = m_2$, $\alpha_1^{(2)} = m_1$. On construit alors $F_2 := \{\beta \in F_1 \mid \beta < \alpha^{(2)}$, puis de manière récursive $m_i$ et $F_i$ pour $i \in \lcc 1,n \rcc$. $F_n$ est infini, et $F_n \subseteq F_{n-1} \subseteq \cdots \subseteq F_1 \subseteq F$. Soit $\beta \in F_n$ tq $\beta < \alpha^{(n)}$, alors $\beta_i \geq \alpha_i^{(n)}$ par construction de $\alpha^{(n)}$. Ainsi $\beta - \alpha^{(n)} \in \mathbb{Z}^n_{> 0}$ Alors $\beta - \alpha^n < 0$, car sinon on aurait $\beta \geq \alpha^{(n)}$. 
            \end{proof}
            }

        \subsection{Algorithme de division multivariée}
            Fixons maintenant un ordre monomial $\leq$ sur $k[x_1, \cdots, x_n]$.
            \begin{defi}
                Soit $f = \sum_{\alpha \in \mathbb{N}^n} \lambda_\alpha x^\alpha \in k[x_1, \cdots, x_n] \bs \{0\}$, 
                \begin{enumerate}
                    \item Le multidegré de $f$ est $\mathrm{mdeg}(f) = \max \{\alpha \in \mathbb{N}^n \mid \lambda_\alpha \neq 0\}$
                    \item Le coefficient dominant de $f$ $\mathrm{LC}(f) = \lambda_{\mathrm{mdeg}(f)}$
                    \item Le mo,ome dominant de $f$ est $\mathrm{LM}(f) = x^{\mathrm{mdeg}(f)}$
                    \item Le terme dominant de $f$ est $\mathrm{LT}(f) = \lambda_{\mathrm{mdeg}(f)}x^{\mathrm{mdeg}(f)}$
                \end{enumerate}
            \end{defi}
            Soit $(f_1, \cdots, f_r)$ un $r$-tuple de polynômes non nuls de $k[x_1, \cdots, x_n]$. Soit $f \in k[x_1, \cdots, x_n]$, on cherche $Q_1, \cdots, Q_r, R \in k[x_1, \cdots, x_n]$ tq
            \begin{enumerate}
                \item $f = Q_1f_1 + \cdots + Q_r f_r + R$
                \item $R = 0$ ou aucun des termes de $R$ n'est divisible par $\mathrm{LT}(f_1), \cdots, \mathrm{LT}(f_r)$.
            \end{enumerate}

            \begin{algorithm}
                \caption{Réalise la division euclidienne multivariée de $f$ par $f_1, \cdots, f_r$}
                \begin{algorithmic}
                    \Function{Division multivariée}{$f , f_1, \cdots, f_r \in k[x_1, \cdots, x_n]$}
                        \State $g \gets f$
                        \State $Q_1, \cdots, Q_r \gets 0$
                        \State $R \gets 0$
                        \While{$g \neq 0$}
                            \State $b = True$
                            \State $i \gets 1$
                            \While{$b$ \textbf{and} $i \leq r$}
                                \If{$\mathrm{LT}(f_i) \mid \mathrm{LT}(g)$}
                                    \State $g \gets g - \frac{\mathrm{LT}(g)}{\mathrm{LT}(f_i)} f_i$
                                    \State $Q_i \gets Q_i + \frac{\mathrm{LT}(g)}{\mathrm{LT}(f_i)}$
                                    \State $b \gets False$
                                \EndIf
                                \State $i \gets i + 1$
                            \EndWhile
                            \If{b}
                                \State $h = LT(g)$
                                \State $g \leftarrow g - h$
                                \State $R \leftarrow R + h$
                            \EndIf
                        \EndWhile
                        \State \Return $R,Q_1, \cdots, Q_r$
                    \EndFunction
                \end{algorithmic}
            \end{algorithm}
            \begin{remq}
                Après chaque tour de boucle while principale, on a toujours 
                \begin{align*}
                    f = g + \sum Q_if_i + R
                \end{align*}
                au vu des calculs réalisés dans la boucle. Et comme l'algorithme se termine lorsque $g = 0$, on obtiens finalement 
                \begin{align*}
                    f = \sum Q_if_i + R
                \end{align*}
                et aucun des termes de $R$ n'est divisible par $\mathrm{LT}(f_i)$ vu que l'on ajoute que des termes divisibles par aucun des $\mathrm{LT}(f_i)$ dans l'algorithme. Finalement, l'algorithme termine puisque à chaque étape de la boucle while principale, le multidegré de $g$ diminue strictement au vu des calculs effectués et du fait que $\leq$ est une relation d'ordre monomiale.
            \end{remq}
            % \textbf{Algorithme}
            % \begin{enumerate}
            %     \item Initialisation : $f^{(0)} := f$, $Q_1^{(0)} , \cdots, Q_r^{(0)} = 0$, $R^{(0)} = 0$.
            %     \item Etapte $m \geq 1$ : Si $f^{(m-1)} = 0$, alors $Q_i := Q_i^{(m_1)}$ et $R = R^{(m-1)}$, terminer l'algo. Sinon, si $\mathrm{LT}(f_1) \mid \mathrm{LT}(f^{(m_1)})$, effectuer :
            %     \begin{align*}
            %         &f^{(m)} \leftarrow f^{(m-1)} - \frac{\mathrm{LT}(f^{(m-1)})}{\mathrm{LT}(f_1)} f_1 \\
            %         &Q_1^{(m)} \leftarrow Q_1^{(m-1)} + \frac{\mathrm{LT}(f^{(m-1)})}{\mathrm{LT}(f_1)} \\
            %         &Q_i^{(m)} \leftarrow Q_i^{(m_1)},\, i \neq 1 \\
            %         &R^{(m)} \leftarrow R^{(m-1)}
            %     \end{align*}
            %     Sinon si $\mathrm{LT}(f_2) \mid \mathrm{LT}(f^{(m-1)})$, effectuer
            %     \begin{align*}
            %         &f^{(m)} \leftarrow f^{(m-1)} - \frac{\mathrm{LT}(f^{(m-1)})}{\mathrm{LT}(f_2)} f_2 \\
            %         &Q_2^{(m)} \leftarrow Q_2^{(m-1)} + \frac{\mathrm{LT}(f^{(m-1)})}{\mathrm{LT}(f_2)} \\
            %         &Q_i^{(m)} \leftarrow Q_i^{(m_1)},\, i \neq 2 \\
            %         &R^{(m)} \leftarrow R^{(m-1)}
            %     \end{align*}
            %     sinon si $\mathrm{LT}(f_3) \mid \mathrm{LT}(f^{(m-1)})$, effectuer ... \\
            %     sinon si $\mathrm{LT}(f_r) \mid \mathrm{LT}(f^{(m-1)})$, effectuer ... \\
            %     sinon effectuer
            %     \begin{align*}
            %         & f^{(m)} \leftarrow f^{(m-1)} - \mathrm{LT}(f^{(m-1)}) \\
            %         & R^{(m)} \leftarrow R^{(m-1)} + \mathrm{LT}(f^{(m-1)}) \\
            %         & Q_i^{(m)} \leftarrow Q_i^{(m-1)}
            %     \end{align*}
            % \end{enumerate}
            % \begin{remq}
            %     A la fin de l'étape $m \geq 0$,
            %     \begin{align*}
            %         f^{(m)} + \sum Q_i^{(m)} f_i + R^{(m)} = f
            %     \end{align*}
            %     Si $f^{(m)} = 0$, alors on a bien $\sum Q_i^{(m)} f_i + R^{(m)} = f$ et alors $R^{(m)} = 0$ ou aucun des termes de $R^{(n)}$ n'est divisible par $\mathrm{LT}(f_1), \cdots, \mathrm{LT}(f_r)$. La procédure s'arrête : sinon, on aurait $f^{(0)},f^{(1)}, \cdots$ avec $\mathrm{mdeg} f^{(0)} > \mathrm{mdeg} f^{(1)} > \cdots$ et ainsi $\{\alpha \mid \exists m \in \mathbb{N} ,\, \alpha = \mathrm{mdeg} f^{(m-1)}\}$ n'a pas d'éléments minimal.
            % \end{remq}
            \begin{nota}
                Le reste obtenu s'écrira $\bar f^{f_1, \cdots, f_t}$. Si $F = \{f_1, \cdots, f_r\}$, on écrira $\bar f^F$.
            \end{nota}
            \begin{remq}
                L'algo donne l'exitence de $Q_i$ et $R$ tq $f = \sum Q_if_i + R$ satisfaisant les conditions imposées précédemment. Ces $Q_i$ et $R$ ne sont pas uniques.
            \end{remq}
            \begin{expl}
                $k[x_1, x_2]$, $<_{lex} =: <$, $f = x_1^2 + x_1x_2 + x_2^2$, $f_1 = x_1$, $f_2 = x_1 + x_2$. Alors 
                \begin{align*}
                    f &= (x_1 + x_2)f_1 + x_2^2 \\
                    \intertext{(Résultat obtenu en appliquant l'algorithme de division multivariée)}
                    &= x_1f_2 + x_2^2 \\
                    &= x_1f_1 + x_2f_2 + 0 \\
                \end{align*}
                donc $f \in (f_1, f_2)$ mais $\bar f^{f_1, f_2} \neq 0$ !
            \end{expl}

    \section{Bases de Gröbner}
        \subsection{Définition}
            \begin{defi} (Base de Gröbner, 1)
                \label{grob_1}
                Soit $I \subrel{id} k[x_1, \cdots, x_n]$ non nul. Une base de Gröbner de $I$ est un ensemble fini $G \subseteq I$ tq
                \begin{enumerate}
                    \item $I = (G)$,
                    \item $f \in I \iff \bar f^G = 0$
                \end{enumerate}
            \end{defi}
            Par convention, $\emptyset$ est une base de Gröbner de l'idéal nul.
            \begin{expl}
                \begin{enumerate}
                    \item Si $0 \neq g \in k[x]$, alors $\{g\}$ est une BDG (base de Gröbner) de $(g)$.
                    \item Si $0 \neq g \in k[x_1, \cdots, x_n]$, alors $\{g\}$ est une BDG de $(g)$.
                \end{enumerate}
            \end{expl}
            Comment peut-on avoir $f \in (f_1, \cdots, f_r)$ mais $\bar f^{f_1, \cdots, f_r} \neq 0$ ? Il faut qu'à une étape de la division, $\mathrm{LT}(f)$ ne soit pas divisible par aucun des $\mathrm{LT}(f_i)$. 
        
        \subsection{Idéaux monomiaux}
            \begin{defi} (Idéal monomial)
                Un idéal $I \subrel{id} k[x_1, \cdots, x_n]$ est monomial s'il existe des monômes $m_1, \cdots, m_r$ tq $I = (m_1, \cdots, m_r)$ (par convention $\{0\}$ est monomial).
            \end{defi}
            \begin{prop}
                \label{lemme_nul}
                Soient $m_1, \cdots, m_r \in k[x_1, \cdots, x_n]$ des monömes, alors $m \in (m_1, \cdots, m_r) \iff m$ est divisible par l'un des $m_i$.
            \end{prop}
            \begin{proof}
                Si $m$ est divisible par l'un des $m_i$, il est clair que $m \in (m_1, \cdots, m_r)$. Pour prouver l'implication réciproque, supposons que $m \in (m_1, \cdots, m_r)$. Alors on peut écrire
                \begin{align*}
                    m = \sum_{i = 1}^r a_i m_i
                \end{align*}
                avec $a_i \in k[x_1, \cdots, x_n]$. Maintenant écrivons chaque $a_i$ comme
                \begin{align*}
                    a_i(x) = \sum_{\alpha \in \mathbb{N}^n} \lambda_\alpha^i x^{\alpha}
                \end{align*}
                Alors
                \begin{align*}
                    m = \sum_{i = 1}^r \sum_{\alpha \in \mathbb{N}^n} \lambda_\alpha^i x^{\alpha} m_i
                \end{align*}
                Maintenant comme $m$ est un monome, il va exister $i, \alpha$ tels que $m = \lambda x^\alpha m_i$, donc $m_i \mid m$.
            \end{proof}
            Soient $f_1, \cdots, f_r \in k[x_1, \cdots, x_n]$. $\mathrm{LT}(f)$ divisible par l'un des $\mathrm{LT}(f_1), \cdots, \mathrm{LT}(f_r)$ si et seulement si $\mathrm{LT}(f) \in (\{\mathrm{LT}(f_i)\})$ d'après la proposition précédente. \\
            \begin{nota}
                Soit $E \subseteq k[x_1, \cdots, x_n]$, on note
                \begin{align*}
                    \mathrm{LT}(E) := \{\mathrm{LT}(f) \mid f \in E\}
                \end{align*}
            \end{nota}
            \begin{defi} (Base de Gröbner, 2)
                Une base de Gröbner d'un idéal $I \subrel{id} k[x_1, \cdots, x_n]$ est un ensemble (fini) $G \subseteq I$ tq $(\mathrm{LT}(I)) = (\mathrm{LT}(G))$
            \end{defi}
            \begin{theo}
                Les deux définitions de bases de Gröbner sont équivalentes.
            \end{theo}
            \begin{proof}
                def 1 $\Rightarrow$ def 2 : Soit $f \in I$ si $\mathrm{LT}(f) \notin (\mathrm{LT}(G))$, alors $\mathrm{LT}(f)$ n'est divisible par aucun des $\mathrm{LT}(g)$, $g \in G$ donc $\bar f^G \neq 0$. \\
                def 2 $\Rightarrow$ def 1 : Notons $G = \{g_1, \cdots, g_r\}$. Soit $f \in I$, on veut que $\bar f^G = 0$. Il suffit de montrer que le reste est nul à chaque étape de l'algo de division. Or à l'étape $0$ il l'est, puis en supposant qu'il l'est à l'étape $m$, on a
                \begin{align*}
                    f = g + \sum Q_i g_i \in I
                \end{align*}
                et donc $g \in I$. Ainsi $LT(g) \in (LT(I)) = (LT(G))$ et donc il existe un $g_i$ tel que $LT(g_i) \mid LT(g)$ daprès \ref{lemme_nul}, et ainsi le reste est inchangé à cette étape.
                % Or
                % \begin{align*}
                %     f - \sum Q_i^{(m)} g_i - R^{(m)} = f^{(m)}
                % \end{align*}
                % et $f - \sum Q_i^{(m)} g_i \in I$. Si $R^{(m)}$, alors $f^{(m)} \in I$, donc $\mathrm{LT}(f^{(m)}) \in (\mathrm{LT}(G))$. D'où $R^{(m+1)} = 0$ puis récurrence.
            \end{proof}
            \begin{theo}
                Tout $I \subrel{id} k[x_1, \cdots, x_n]$ admet une base de Gröbner.
            \end{theo}
            \begin{proof}
                On cherche $G \subrel{fini} I$ tq $(\mathrm{LT}(G)) = (\mathrm{LT}(I))$. D'après \ref{base_de_hilbert}, $\exists H \subrel{fini} \mathrm{LT}(I)$ tq $(H) = (\mathrm{LT}(I))$. Notons $h_1, \cdots, h_r$ des polynômes de $I$ dont les termes dominants sont les éléments de $H$. Alors $\{h_1, \cdots, h_r\}$ est une BDG de $I$.
            \end{proof}

    \section{Algorithme de Buchberger}
        \subsection{Critère de Buchberger}
            \begin{defi}
                $f,g \in k[x_1, \cdots, x_n]$, alors
                \begin{align*}
                    S(f,g) := \frac{\mathrm{ppcm} (\mathrm{LM}(f), \mathrm{LM}(g))}{\mathrm{LT}(f)}f - \frac{\mathrm{ppcm} (\mathrm{LM}(f), \mathrm{LM}(g))}{\mathrm{LT}(g)}g
                \end{align*}
            \end{defi}
            \begin{theo} (Critère de Buchberger)
                Soit $G = \{g_1, \cdot, g_r\} \subseteq k[x_1, \cdots, x_r]$. Alors $G$ est une BDG de $(G)$ si et seulement si $\forall g,h \in G$, $\overline{S(g,h)}^G = 0$
            \end{theo}  
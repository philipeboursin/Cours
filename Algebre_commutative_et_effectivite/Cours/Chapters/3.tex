\chapter{Changements de bases de Grobner}
    \section{Ordres matriciels}
        \begin{defi}
            Soit $M \in M_{m,n}(\mathbb{R})$. On définit une relation $<_M$ sur $\mathbb{N}^n$ de la façon suivante :
            \begin{align*}
                \alpha <_M \beta &\iff M \alpha <_{lex} M \beta \\
            \end{align*}
        \end{defi}
        \begin{expl}
            Sur $k[x_1, x_2, x_3]$, $I_3$ convient pour $<_{lex}$,
            \begin{align*}
                \begin{bmatrix}
                    1 & 1 & 1 \\
                    1 & 0 & 0 \\
                    0 & 1 & 0 \\
                    0 & 0 & 1 \\
                \end{bmatrix}
            \end{align*}
            convient pour $<_{deglex}$,
            \begin{align*}
                \begin{bmatrix}
                    1 & 1 & 1 \\
                    0 & 0 & -1 \\
                    0 & -1 & 0 \\
                    -1 & 0 & 0 &\\
                \end{bmatrix}
            \end{align*}
            convient pour $<_{degrevlex}$.
        \end{expl}
        \begin{remq}
            \begin{align*}
                \begin{bmatrix}
                    1 & 0 & 0 \\
                    -1 & 1 & 0 \\
                    -1 & -1 & 1 \\
                \end{bmatrix}
            \end{align*}
            convient aussi pour $lex$.
        \end{remq}
        \begin{defi} (Noyau à droite)
            Le noyau à droite de $M \in M_{m,n}(\mathbb{R})$ est
            \begin{align*}
                \ker M := \{v \in \mathbb{R}^n \mid Mv = 0\}
            \end{align*}
        \end{defi}
        \begin{prop}
            Soit $M \in M_{m,n}(\mathbb{R})$, alors
            \begin{enumerate}
                \item $\forall \alpha, \beta, \gamma \in \mathbb{N}^n$,
                \begin{align*}
                    \alpha <_M \beta \iff \alpha + \gamma <_M \beta + \gamma
                \end{align*}
                \item Si $\ker M \cap \mathbb{Z} = \{0\}$, alors $\forall \alpha \neq \beta \in \mathbb{N}^n$, $(\alpha <_M \beta) \lor (\beta <_M \alpha)$.
                \item S'il existe une matrice $T \in M_{m,m}(\mathbb{R})$ triangulaire inférieure dont les coefficients diagonaux sont strictements positifs et t.q. $TM \in M_{m,n}(\mathbb{R}_{\geq 0})$, alors $\forall \alpha \in \mathbb{N}^n$, $0 \leq_M \alpha$.
            \end{enumerate}
        \end{prop}
        \begin{proof}
            \begin{enumerate}
                \item
                \begin{align*}
                    \alpha <_M \beta &\iff M \alpha <_{lex} M \beta \\
                    &\iff M \alpha + M \gamma <_{lex} M \beta + M \gamma \\
                    &\iff M(\alpha + \gamma) <_{lex} M(\beta + \gamma) \\
                    &\iff \alpha + \gamma <_M \beta + \gamma \\ 
                \end{align*}
                \item Soient $\alpha \neq \beta \in \mathbb{N}^n$, alors
                \begin{align*}
                    \alpha <_M \beta \lor \beta <_M \alpha &\iff M \alpha <_{lex} M \beta \lor M \beta <_{lex} M \alpha \\
                    &\iff M \alpha \neq M \beta \iff \alpha - \beta \notin \ker M
                \end{align*}
                et comme $\ker M \cap \mathbb{Z}^n = 0$ et $\alpha \neq \beta$, alors $\alpha - \beta \notin \ker M$ est toujours vraie.
                \item Notons $w_i$ les lignes de $M$. $TM$ est obtenue en effectuant les opérations suivantes :
                \begin{itemize}
                    \item Remplacer $w_1$ par un multiple strictement positif de $w_1$.
                    \item Remplacer $w_2$ par un multiple strictement positif de $w_2$ plus une combinaison linéaire de $w_1$.
                    \item Remplacer $w_3$ par un multiple strictement positif de $w_3$ plus une combinaison linéaire de $w_1, w_2$.
                    \item $\vdots$
                \end{itemize}
                Pour comparer $\alpha, \beta \in \mathbb{N}^n$ pour $<_M$ on calcule
                \begin{align*}
                    &M \alpha =
                    \begin{bmatrix}
                        w_1 \cdot \alpha \\
                        \vdots \\
                        w_b \cdot \alpha
                    \end{bmatrix}
                    ,\, M \beta =
                    \begin{bmatrix}
                        w_1 \cdot \beta \\
                        \vdots \\
                        w_b \cdot \beta
                    \end{bmatrix} \\
                \end{align*}
                Montrons que $<_M = <_{TM}$. Notons $T = (t_{ij})_{1 \leq i,j \leq m}$. Alors
                \begin{align*}
                    TM =
                    \begin{bmatrix}
                        t_{11} w_1 \\
                        t_{21} w_1 + t_{22} w_2 \\
                        t_{31} w_1 + t_{32} w_2 + t_{33} w_3 \\
                        \vdots \\
                    \end{bmatrix}
                \end{align*}
                Maintenant
                \begin{align*}
                    \alpha <_M \beta &\iff 
                    \begin{cases}
                        w_1 \alpha < w_1 \beta \\
                        \text{ou alors } w_1\alpha = w_1 \beta \text{ et } w_2 \alpha < w_2 \beta \\
                        \text{ou alors } w_1\alpha = w_1 \beta \text{ et } w_2 \alpha = w_2 \beta \text{ et } w_3 \alpha < w_3 \beta \\
                        \vdots
                    \end{cases} \\
                    &\iff 
                    \begin{cases}
                        t_{11} w_1 \alpha < t_{11} w_1 \beta \\
                        \text{ou alors } t_{11} w_1 \alpha = t_{11} w_1 \beta \text{ et } t_{22} w_2 \alpha + \cor{t_{21} w_1 \alpha} < t_{22} w_2 \beta + \cor{t_{21} w_1 \beta} \\
                        \vdots 
                    \end{cases} \\
                    &\iff TM \alpha <_{lex} TM \beta \iff \alpha <_{TM} \beta
                \end{align*}
                et aini $<_M = <_{TM}$. Maintenant comme $TM \in M_{m,n}(\mathbb{R}_{\geq 0})$, pour tout $\alpha \in \mathbb{N}^n$, $TM\alpha \in \mathbb{R}^n_{\geq 0}$ et donc $0 \leq_{TM} \alpha$, d'où $0 \leq_M \alpha$.
            \end{enumerate}
        \end{proof}
        \begin{coro}
            Pour tout $T$ triangulaire inférieure avec coefficients diagonaux strictement positifs, alors $<_{TM} = <_M$.
        \end{coro}
        \begin{coro}
            Si une ligne de $M$ est combinaison linéaire des lignes au dessus, alors la retirer ne change pas l'ordre matriciel.
        \end{coro}
        \begin{coro}
            Tout ordre matriciel est égal à un ordre matriciel $<_M$, où $M$ a au plus $n$ lignes.
        \end{coro}
        \begin{expl}
            $M = \begin{bmatrix} 1 & \sqrt{2} \end{bmatrix}$ définit un ordre monomial.
        \end{expl}
        \begin{coro}
            Tout ordre monomial matriciel est égal à $<_M$ où $M$ a exactement $n$ lignes.
        \end{coro}
        \begin{proof}
            D'après le corolaire précédent, on peut prendre $M$ avec moins de $n$ lignes. Mais alors rajouter des lignes de zéros ne change pas l'ordre.
        \end{proof}
        \begin{remq}
            Si $n \geq 2$, alors $k[x_1, \cdots, x_n]$ admet une infinité d'ordres monomiaux. Par exemple, pour $n = 2$, pour tout $a \in \mathbb{N}$, on définit
            \begin{align*}
                M_a =
                \begin{bmatrix}
                    1 & a \\
                    0 & 1 \\
                \end{bmatrix}
            \end{align*}
            Alors $y >_{M_a} x^a$ et $y <_{M_a} x^{a+1}$, donc les $<_{M_a}$ définissent une infinité d'ordre monomiaux différents.
        \end{remq}
        \begin{theo} (Robbiano, 1985)
            Tout ordre monomial est un ordre matriciel.
        \end{theo}
        \begin{proof}
            Soit $<$ un ordre monomial sur $\mathbb{N}^n$.
            \item \textbf{Etape 1 :} $<$ s'étend en un unique ordre total additif sur $\mathbb{Z}^n$ : si $\alpha, \beta \in \mathbb{Z}^n$, alors $\exists \gamma \in \mathbb{Z}^n$ tel que $\alpha + \gamma, \beta + \gamma \in \mathbb{N}^n$. On pose ainsi
            \begin{align*}
                \alpha < \beta \iff \alpha + \gamma < \beta + \gamma 
            \end{align*}
            Clairement, cette définition ne dépend pas du choix de $\gamma$. Donc $<$ est étendu en un ordre total à $\mathbb{Z}^n$.
            \item \textbf{Etape 2 :} L'ordre total additif $<$ sur $\mathbb{Z}^n$ s'étend en un unique ordre total additif sur $\mathbb{Q}^n$ : si $\alpha, \beta \in \mathbb{Q}^n$, alors $\exists \lambda \in \mathbb{N}^n$ tq $\lambda \alpha, \lambda \beta \in \mathbb{Z}^n$. Ainsi on pose 
            \begin{align*}
                \alpha < \beta \iff \lambda \alpha < \lambda \beta
            \end{align*}
            Ceci ne dépend pas de $\lambda$, et on a ainsi étendu $<$ à un ordre total additif sur $\mathbb{Q}^n$.
            \item \textbf{Etape 3 :} Soient
            \begin{align*}
                &H_- =  \{v \in \mathbb{Q}^n \mid v < 0\} \\
                &H_+ =  \{v \in \mathbb{Q}^n \mid v > 0\} \\
            \end{align*}
            Ainsi $\mathbb{Q}^n = H_- \sqcup \{0\} \sqcup H_+$. Alors considérons les adhérences $\bar H_-$, $\bar H_+$ dans $\mathbb{R}$, puis $I_0 = \bar H_- \cap \bar H_+$. Montrons que $I_0$ est un sev de $\mathbb{R}^n$ de codimension $1$.
            \begin{itemize}
                \item $H_+, H_-$ sont stables pas somme.
                \item $H_+, H_-$ sont stables par produit par des éléments de $\mathbb{Q}_{>0}$.
                \item L'opération $\sigma : v \mapsto -v$ est une bijection de $H_+$ dans $H_-$.
            \end{itemize}
            Ainsi
            \begin{itemize}
                \item $\bar H_+, \bar H_-$ sont stables par somme.
                \item $\bar H_+, \bar H_-$ sont stables par produits par des éléments de $\mathbb{R}_{\geq 0}$.
                \item $\sigma : v \mapsto -v$ induit une bijection entre $\bar H_+$ et $\bar H_-$.
            \end{itemize}
            Par conséquent, $I_0$ est stable par somme et produit par un réél quelconque. Comme $I_0 \neq \emptyset$, car $0 \in I_0$, ceci donne que $I_0$ est un sev de $\mathbb{R}^n$. Montrons que $\dim I_0 = n - 1$ en montrant que $I_0 \neq \mathbb{R}^n$, et que $\mathbb{R}^n \bs I_0$ n'est pas connexe. Puisque $\mathbb{Q}_{> 0}^n \cap H_- = \emptyset$, on obtiens que $I_0 \neq \mathbb{R}^n$. De plus, $\mathbb{R}^n \bs I_0 = (\bar H_+ \bs I_0) \sqcup (\bar H_- \bs I_0)$, et ces deux composantes sont des fermés, donc $\mathbb{R}^n \bs I_0$ n'est pas connexe.
            \item \textbf{Etape 4 :} Soit $w_1$ un vecteur non nul, orthogonal à $I_0$ tel que pour tout $h \in \bar H_+$, alors $\bra w_1,h \ket\geq 0$ ($w_1$ existe quitte à le multiplier par $-1$, et est unique à produit par $\mathbb{R}_{>0}$ près). Alors pour tout $v \in \mathbb{R}^n$,
            \begin{itemize}
                \item $v \in \bar H_+ \iff \bra w_1, v \ket \geq 0$
                \item $v \in \bar H_- \iff \bra w_1, v \ket \leq 0$
                \item $v \in I_0 \iff \bra w_1, v \ket = 0$
            \end{itemize}
            Si $v,v' \in \mathbb{Q}^n$, alors $v < v' \iff v - v' < 0 \iff v - v' \in H_- \Leftarrow \bra w_1, v - v' \ket < 0$. Le vecteur $w_1$ sera la première ligne d'une matrice $M$ telle que $<_M = <$ sur $\mathbb{N}^n$.
            \item \textbf{Etape 5 :} Si $\bra v - v', w_1 \ket = 0$, alors $v - v' \in I_0$. Soit $G_1 = I_0 \cap \mathbb{Q}^n$, alors $G_1$ est une $\mathbb{Q}$-ev de dimension au plus $n-1$. Posons 
            \begin{align*}
                &H_{1,+} = \{v \in G_1 \mid v > 0 \} \\
                &H_{1,-} = \{v \in G_1 \mid v < 0 \} \\
            \end{align*}
            $I_1 = \bar H_{1,+} \cap \bar H_{1,-}$. Comme pour $I_0$, on montre que $I_1$ est un sev de codim $1$ dans $\bar G_1$. Soit $w_2$ un vecteur orthogonal à $I_1$ dans $\bar G_1$ tq $\forall h \in \bar H_{1,r}$, $\bra w_2, h \ket \geq 0$. On a donc
            \begin{align*}
                \alpha <_{\begin{bmatrix} w_1 \\ w_2 \end{bmatrix}} \beta \Rightarrow
                \begin{cases}
                    w_1\alpha < w_1 \beta \\
                    \text{ou } w_1\alpha = w_1 \beta \text{ et } w_2 \alpha < w_2 \beta \\
                    \text{ou } w_1 \alpha = w_1 \beta \text{ et } w_2 \alpha = w_2 \beta
                \end{cases}
            \end{align*}
            \item \textbf{Etape 6 :} On pose $G_2 = \mathbb{Q}^n \cap I_1$. et ainsi de suite. On construit au plus $n$ vecteur $w_1, \cdots, w_m$ tq
            \begin{align*}
                \alpha <_{\begin{bmatrix} w_1 \\ \vdots \\ w_m \end{bmatrix}} \beta \iff \alpha < \beta
            \end{align*}
        \end{proof}

    \section{Bases de Gröbner marquées, universelles}
        \begin{nota}
            $<$ ordre monomial, $E \subseteq k[x_1, \cdots, x_n]$. Alors
            \begin{align*}
                LT_<(E) := \{LT_<(f) \mid f \in E\}
            \end{align*}
            \begin{align*}
                Mon(E) = \{\bra LT_<(E) \ket \mid < \text{ ordre monomial}\}
            \end{align*}
        \end{nota}
        \begin{theo}
            Soit $I \subrel{id} k[x_1, \cdots, x_n]$. Alors $Mon(I)$ est fini.
        \end{theo}
        \begin{proof}
            Supposons le contraire, pour chaque $J \in Mon(I)$, soit $<^J$ un ordre monomial tel que $J = \bra LT_{<^J}(I) \ket$. Soit
            \begin{align*}
                \Sigma = \{<^J \mid J \in Mon(I)\}
            \end{align*}
            Par le théorème de la base de Hilbert  il existe $f_1, \cdots, f_r \in I$ tq $I = \bra f_1, \cdots, f_r \ket$. Chaque $f_i$ n'a qu'un nombre fini de termes, puisque $\Sigma$ est infini, $\exists \Sigma_1 \subseteq \Sigma$ infini tel que $\forall i \in \lcc 1,r \rcc$, $LT_<(f_i)$ prend la même valeur pour tout $< \in \Sigma_1$. Posons
            \begin{align*}
                J := \bra LT_<(f_1), \cdots, LT_<(f_r) \ket
            \end{align*}
            pour $< \in \Sigma_1$. Montrons que $\{f_1, \cdots, f_r\}$ n'est pas une bdg de $I$, pour $< \in \Sigma_1$. Si c'était le cas, alors ce serait une bdg pour tout $<' \in \Sigma_1$ : 
            \begin{align*}
                \bra LT_<(I) \ket = \bra LT_<(f_i),\, 1 \leq i \leq r \ket  = \bra LT_{<'}(f_i),\, 1 \leq i \leq r \ket \subseteq \bra LT_{<'}(I) \ket
            \end{align*}
            puis si un monôme $m$ est dans $\bra LT_{<'}(I) \ket$ mais pas dans $\bra LT_<(I) \ket$, alors la division de $m$ par $f_1, \cdots, f_r$ donne un reste non nul, pour $<$ comme pour $<'$. Mais si $m = LT_{<'}(f)$, $f \in I$, alors le reste de la division de $f$ par $f_1, \cdots, f_r$ pour $<$ est nul. Ce reste contient pourtant le terme $m$, contradiction. Donc $\{f_1, \cdots, f_r\}$ est une bdg pour tout $<' \in \Sigma_1$, donc pour tout $<, <' \in \Sigma_1$, 
            \begin{align*}
                \bra LT_<(I) \ket = \bra LT_{<'}(I) \ket
            \end{align*}
            mais par définition de $\Sigma$, si $< \neq <'$, alors $\bra LT_<(I) \ket \neq \bra LT_{<'}(I) \ket$, contradiction. Ainsi $\{f_1, \cdots, f_r\}$ n'est pas une bdg pour $I$ et pour $< \in \Sigma_I$. Il existe donc $f_{r+1} \in I$ tq $LT_<(f_{r+1}) \notin \bra LT_<(f_i) \ket = J$. Alors $\exists \Sigma_2 \subseteq \Sigma_1$ infini tel que les valeurs de $LT_<(f_i)$, $i \in \lcc 1,r+1 \rcc$, sont les mêmes pour tout $< \in \Sigma_2$. Comme plus haut, on mq $\{f_1, \cdots, f_{r+1}\}$ n'est pas une bdg de $I$ pour $< \in \Sigma_2$. Donc $\exists f_{r+2} \in I$ tel que $LT_<(f_{r+2}) \notin \bra LT_<(f_1), \cdots, LT_<(f_{r+1}) \ket$ pour $< \in \Sigma_2$. Ainsi on construit par récurrence une famille d'ensembles infinis $\Sigma \supseteq \Sigma_1 \supseteq \Sigma_2 \supseteq \cdots$ et des éléments $f_1, f_2, \cdots$ pour $<_i \in \Sigma_i$ tels que 
            \begin{align*}
                J \varsubsetneq \bra LT_{<_1}(f_1), \cdots, LT_{<_1}(f_{r+1}) \ket \varsubsetneq \bra LT_{<_2}(f_1), \cdots, LT_{<_1}(f_{r+2}) \ket \varsubsetneq \cdots 
            \end{align*}
            ce qui contredit la noethérianité de $k[x_1, \cdots, x_n]$.
        \end{proof}

        \begin{defi} (Base de grobner marquée)
            Soit $I \subrel{id} k[x_1, \cdots, x_n]$. Une base de grobner marquée pour $I$ est un ensemble de polynômes $\{g_1, \cdots, g_r\} \subseteq I$ et un choix de monôme $m_i$ de $g_i$ tel qu'il existe un ordre monomial $<$ pour lequel $\{g_1, \cdots, g_r\}$ est la base de grobner réduite et $m_i = LT_<(g_i)$.
        \end{defi}
        \begin{coro}
            L'ensemble des bdg marquée de $I$ est en bijection avec $Mon(I)$, et est donc fini.
        \end{coro}
        \begin{proof}
            Soit $\{(g_1,m_1), \cdots, (g_r,m_r)\}$ une bdg marquée de $I$. Supposons que $<, <'$ sont deux ordres monomiaux pour lesquels $\{(g_1, m_1), \cdots, (g_r, m_r)\}$ est la base de grobner marquée. Alors
            \begin{align*}
                \bra LT_<(I) \ket = \bra LT_{<'}(I) \ket
            \end{align*}
            En effet, $\bra LT_<(I) \ket = \bra LT_<(g_i) \ket = \bra m_i \ket = \bra LT_{<'}(g_i) \ket = \bra LT_{<'}(I) \ket$. On a donc défini une application
            \begin{align*}
                \begin{array}{cccc}
                    \phi : & \{\text{bdg marquées}\} & \to & Mon(I) \\
                    & \{(g_i,m_i)\} & \mapsto & \bra LT_<(I) \ket \\
                \end{array}
            \end{align*}
            où $<$ est un ordre pour lequel $\{(g_i, m_i)\}$ est une bdg marquée. On définit une inverse $\psi$ à $\phi$ : Soit $J \in Mon(I)$, puis soient $<, <'$ tq $J = \bra LT_<(I) \ket = \bra LT_{<'}(I) \ket$. Alors $<$ et $<'$ définissent la même bdg marquée de $I$ : soit $\{(g_i,m_i)\}$ la base de groebner marquée pour $<$ (la base de Groebner réduite à laquelle on rajoute les monômes dominants), alors
            \begin{align*}
                \bra LT_<(g_i) \ket &= \bra LT_<(I) \ket \\
                &= \bra LT_{<'}(I) \ket \supseteq \bra LT_{<'}(g_i) \ket
            \end{align*}
            Pour chaque $i$, $LT_{<'}(g_i)$ est divisible par l'un des $LT_<(g_j)$, mais comme $(g_i)$ est une bdg réduite, ceci entraine que $LT_{<'}(g_i) = LT_<(g_i)$. En particulier $(g_i, m_i)$ est une bdg, réduite et marquée pour l'ordre $<'$. On a donc défini
            \begin{align*}
                \begin{array}{cccc}
                    \psi : & Mon(I) & \to & \{\text{bdg marquées}\} \\
                    & J & \mapsto & \{(g_i, m_i)\}\\
                \end{array}
            \end{align*}
            et il est clair que $\phi$ et $\psi$ sont mutuellement inverses.
        \end{proof}
        \begin{coro}
            Il existe un ensemble fini $\mathcal{U} \subseteq I$ tel que $\mathcal{U}$ est une bdg de $I$, quelque soit l'ordre monomial.
        \end{coro}
        \begin{defi}
            Ce $\mathcal{U}$ est appelé base de grobner universelle.
        \end{defi}

    \section{Éventail de Gröbner}
        \begin{defi}
            \begin{enumerate}
                \item Un cône dans $\mathbb{R}^n$ est un ensemble ayant la forme
                \begin{align*}
                    C(v_1, \cdots, v_r) := \left\{ \sum_{finie} \lambda_i v_i \mid \lambda_i \geq 0 \right\}
                \end{align*}
                De façon équivalente, un cône est une intersection de demi espaces fermés.
                \item Un hyperplan de définition d'un cône $C$ est hyperplan $H = v^\bot$ tel que $v \cdot C \geq 0$.
                \item Une face d'un cône $C$ est une intersection de $\mathcal{C}$ avec l'un de ses hyperplans de définition. Remarquons que les faces d'un cône sont des cônes.
                \item La dimension d'un cône est la dimension du sous-espace de $\mathbb{R}^n$ qu'il engendre.
                \item Les faces de dimension $1$ de $C$ sont les rayons de $\mathcal{C}$.
                \item Les faces de codimension $1$ de $C$ sont les facettes de $\mathcal{C}$.
                \item Un éventail est un ensemble $\mathcal{F}$ de cônes tels que
                \begin{itemize}
                    \item $C \in \mathcal{F} \Rightarrow$ toute face de $C$ est dans $\mathcal{F}$.
                    \item $C,C' \in \mathcal{F} \Rightarrow C \cap C' \in \mathcal{F}$ et est une face de $C$ et $C'$.
                \end{itemize}
            \end{enumerate}
        \end{defi}

        %Cours du 24/10
        \begin{defi}
            Soit $w \in \mathbb{R}^n_+$. Le $w$-degré d'un monôme $x^\alpha$ est $\deg_w x^\alpha = w \cdot \alpha$. Un polynôme est $w$-homogène si tous ses termes ont le même $w$-degré. Si $0 \neq f \in k[x_1, \cdots, x_n]$, on pose 
            \begin{align*}
                LT_w(f) = \sum \text{termes de $f$ de $w$-degré maximal}
            \end{align*}
            Si $E \subseteq k[x_1, \cdots, x_n]$, $LT_w(E) = \{LT_w(f) \mid f \in E\}$.
        \end{defi}
        \begin{nota}
            Si $<_M$ est un ordre monomial, on écrira $LT_M$ au lieu de $LT_{<M}$.
        \end{nota}
        \begin{prop}
            \label{prop331}
            Soit $<_M$ un ordre monomial, soit $w \in \mathbb{R}_+^n$. Posons
            \begin{align*}
                \bar M :=
                \begin{bmatrix}
                    w \\
                    M \\
                \end{bmatrix}
            \end{align*}
            de sorte que $<_{\bar M}$ soit un ordre monomial.
            \begin{enumerate}
                \item $\forall f \in k[x_1, \cdots, x_n]$, $LT_{\bar M}(f) = LT_M(LT_w(f))$
                \item Si $I \subrel{id} k[x_1, \cdots, x_n]$, alors
                \begin{align*}
                    \bra LT_M \bra LT_w(I) \ket \ket = \bra LT_{\bar M}(I) \ket
                \end{align*}
                \item Si $\bar G$ est une bdg de $I$ pour $<_{\bar M}$, alors $LT_w(\bar G)$ est une bgd de $\bra LT_w(I) \ket$ pour $<_M$.
            \end{enumerate}
        \end{prop}
        \begin{lemm}
            Soit $w \in \mathbb{R}_+^n$. Tout polynôme $f \in k[X_1, \cdots, X_n]$ s'écrit de façon unique comme 
            \begin{align*}
                f = \sum_{d \in \mathbb{N}} f_{(d)}
            \end{align*}
            où $f_{(d)}$ est homogène de $w$-degré $d$.
        \end{lemm}
        \begin{proof}
            On construit une telle décomposition en réunissant les monômes de même $w$-degré. Pour l'unicité, il suffit de remarquer que deux monômes de $w$-degré différent sont forcément différents.
        \end{proof}
        \begin{proof}
            \begin{enumerate}
                \item Notons $x^\alpha = LT_M(LT_w(f))$, puis considérons un autre terme $x^\beta$ de $f$ (avec $\alpha \neq \beta$). Déjà, $x^\alpha$ est de $w$-degré maximal, puisque c'est un monôme de $LT_w(f)$. Alors si $w \cdot \beta < w \cdot \alpha$, on a bien $\alpha >_{\bar M} \beta$. Sinon, $w \cdot \beta = w \cdot \alpha$, donc $x^\beta$ est un terme de $LT_w(f)$, mais donc $\alpha >_M \beta$ par définition de $\alpha$. Et alors on a encore $\alpha >_{\bar M} \beta$, donc finalement $x^\alpha = LT_{\bar M}(f)$.
                \item $\supseteq$ : 
                \begin{align*}
                    \bra LT_{\bar M}(I) \ket = \bra LT_M LT_w(I) \ket \subseteq \bra LT_M \bra LT_w(I) \ket \ket 
                \end{align*}
                $\subseteq$ : Il suffit de montrer que $LT_M \bra LT_w(I) \ket \subseteq \bra LT_{\bar M}(I) \ket$. Soit $f \in \bra LT_w(I) \ket$. Alors 
                \begin{align*}
                    f = \sum_{i = 1}^r q_i LT_w(f_i)
                \end{align*}
                avec $q_i \in k[x_1, \cdots, x_n]$, $f_i \in I$.
                \begin{align*}
                    f &= \sum_{d \in \mathbb{N}} \sum_{i = 1}^r (q_i LT_w(f_i))_{(d)} \\
                    &= \sum_d \sum_{i = 1}^r (q_i)_{(d - \deg_w LT_w f_i)} LT_w(f_i) \\
                \end{align*}
                Si $\sum_{i = 1}^r (q_i)_{(d - \deg_w LT_w f_i)} LT_w(f_i) \neq 0$, alors
                \begin{align*}
                    \sum_{i = 1}^r (q_i)_{(d - \deg_w LT_w f_i)} LT_w(f_i) = LT_w \left( \sum_{i = 1}^r (q_i)_{(d - \deg_w LT_w f_i)} f_i \right) \in LT_w(I) 
                \end{align*}
                Si $\deg_w LT_M(f) = d$, alors
                \begin{align*}
                    LT_M(f) = LT_M \left( \sum (q_i)_{(d - \deg_w LT_w f_i)} LT_w(f_i) \right) \in LT_M LT_w(I) = LT_{\bar M}(I)
                \end{align*}
                \item
                \begin{align*}
                    \bra LT_M \bra LT_w(I) \ket \ket &= \bra LT_{\bar M}(I) \ket \\
                    &= \bra LT_{\bar M}(\bar G) \ket \\
                    &= \bra LT_M LT_w(\bar G) \ket
                \end{align*}
            \end{enumerate}
        \end{proof}
        \begin{prop}
            Soit $<$ un ordre monomial. Soit $I \subrel{id} k[x_1, \cdots, x_n]$. Soit $B$ l'ensemble des monômes qui ne sont pas dans $\bra LT_<(I) \ket$. Soit $\pi : k[x_1, \cdots, x_n] \surjectivearrow k[x_1, \cdots, x_n]/I$ la projection canonique, alors $\pi(B)$ est un base du $k$-ev $k[x_1, \cdots, x_n]/I$.
        \end{prop}
        \begin{proof}
            Soit $G$ une bdg de $I$ pour $<$. Si $0 \neq f \in k[x_1, \cdots, x_n]$, $\bar f^G$ est combinaison linéaire des éléments de $B$. Or $\pi(f) = \pi(\bar f^G)$, d'où $\pi(f)$ est combinaison linéaire des éléments de $\pi(B)$. De plus, $\pi(B)$ est libre car aucune combinaison linéaire d'éléments de $B$ n'est dans $I$.
        \end{proof}
        \begin{coro}
            Si $< \neq <'$ sont deux ordres monomiaux, alors on ne peut pas avoir $\bra LT_<(I) \ket \varsubsetneq \bra LT_{<'}(I) \ket$.
        \end{coro}
        \begin{proof}
            Si on avait une inclusion stricte, alors on aurait que $B' \varsubsetneq B$, et donc $\pi(B') \varsubsetneq \pi(B)$ sont deux bases du même espace vectoriel, impossible.
        \end{proof}
        \begin{defi}
            Soit $I \subrel{id} k[x_1, \cdots, x_n]$. Soit $w \in \mathbb{R}_+^n$.
            \begin{align*}
                C[w] := \left\{ w' \in \mathbb{R}_+^n \mid \bra LT_w(I) \ket = \bra LT_{w'}(I) \ket \right\}
            \end{align*}
        \end{defi}
        \begin{prop}
            \label{prop333} Soit $<_M$ un ordre monomial, et
            \begin{align*}
                \bar M =
                \begin{bmatrix}
                    w \\
                    M \\
                \end{bmatrix}
            \end{align*}
            Soit $\bar G$ la bdg réduite de $I$ pour $<_{\bar M}$. Alors
            \begin{align*}
                C[w] = \{w' \in \mathbb{R}_+^n \mid \forall g \in \bar G,\, LT_w(g) = LT_{w'}(g) \}
            \end{align*}
        \end{prop}
        \begin{proof}
            \item $\subseteq$ : Soit $w' \in C[w]$. Alors $\bra LT_w(I) \ket = \bra LT_{w'} (I) \ket$. D'après \ref{prop331}, $LT_w(\bar G)$ est la bdg réduite de $\bra LT_w(I) \ket$ pour $<_M$. Soit $g \in \bar G$,
            \begin{align*}
                \overline{LT_w'(g)}^{LT_w(\bar G)} = 0
            \end{align*}
            Alors $LT_M LT_{w'}(g) \in \bra LT_M LT_w(\bar G) \ket = \bra LT_{\bar M}(\bar G) \ket$. Comme $\bar G$ est réduite, le seul terme de $g$ qui soit dans $\bra LT_{\bar M}(\bar G) \ket$ est $LT_{\bar M}(g)$. Donc
            \begin{align*}
                LT_M LT_{w'}(g) = LT_{\bar M}(g) = LT_M LT_w (g)
            \end{align*}
            $LT_w(g) = LT_M LT_w(g) = h$, $LT_{w'}(g) = LT_M LT_{w'}(g) + h' = LT_M LT_{w}(g) + h'$. Donc $LT_w(g) - LT_{w'}(g) = h - h'$. Or $LT_w(g) - LT_{w'}(g) \in \bra LT_w(I) \ket$. Donc $\overline{h - h'}^{LT_w(\bar G)} = 0$. Or aucun des termes de $h$ ou $h'$ n'est divisible par un élément de $LT_M LT_w(\bar G) = LT_{\bar M}(\bar G)$. D'où $h - h' = 0$, et $h=h'$. Donc $LT_w(g) = LT_{w'}(g)$. Ceci montre $\subseteq$.
            \item $\supseteq$ : Soit $w' \in \mathbb{R}^n_+$ tq $\forall g \in \bar G$, $LT_w(g) = LT_{w'}(g)$. Alors 
            \begin{align*}
                \bra LT_w(I) \ket = \bra LT_w(\bar G) \ket = \bra LT_{w'}(\bar G) \ket \subseteq \bra LT_{w'}(I) \ket 
            \end{align*}
            Si l'inclusion était stricte, alors on aurait $\bra LT_M \bra LT_w(I) \ket \ket \varsubsetneq \bra LT_M \bra LT_{w'}(I) \ket \ket$ (en effet, si $J \subseteq J'$ et $\bra LT_M(J) \ket = \bra LT_M(J') \ket$, alors une bdg de $J$ pour $<_M$ est forcément une bdg de $J'$ pour $J'$, et donc $J = J'$). Maintenant
            \begin{align*}
                \bra LT_{\bar M}(I) \ket = \bra LT_M \bra LT_w(I) \ket \ket \varsubsetneq \bra LT_M \bra LT_{w'}(I) \ket \ket = \bra LT_{\begin{bmatrix} w \\ M \end{bmatrix}}(I) \ket
            \end{align*}
            contradiction avec le corollaire précédent. Donc $\bra LT_w(I) \ket = \bra LT_{w'}(I) \ket$, d'où $w' \in C[w]$.
        \end{proof}
        \begin{coro}
            $C[w]$ est un cône relativement ouvert, i.e. une intersection de demi-espaces ouverts et d'hyperplans.
        \end{coro}
        \begin{proof}
            Soient $<_M$ un ordre monomial, $\bar M = \begin{bmatrix} w \\ M \end{bmatrix}$, $\bar G$ une bdg réduite de $I$ pour $<_{\bar M}$. Alors
            \begin{align*}
                C[w] = \{w' \in \mathbb{R}_+^n \mid \forall g \in \bar G,\, LT_w(g) = LT_{w'}(g) \}
            \end{align*}
            Donc
            \begin{align*}
                w' \in C[w] &\iff \forall g \in \bar G,\, LT_w(g) = LT_{w'}(g) \\
                &\iff \forall g \in \bar G,
                \begin{cases}
                    \text{Si $x^\alpha$ et $x^\beta$ sont deux monômes de $LT_w(g)$, alors $w' \cdot \alpha = w' \cdot \beta$} \\
                    \text{Si $x^\alpha$ est monôme de $LT_w(g)$ et $x^\beta$ est un monôme de $g$ mais pas} \\
                    \text{de $LT_w(g)$, alors $w' \cdot \alpha > w' \cdot \beta$} \\
                \end{cases} \\
                &\iff
                \begin{cases}
                    w' \in (\alpha - \beta)^\bot \\
                    w'\cdot (\alpha - \beta) > 0 \\
                \end{cases}
            \end{align*}
        \end{proof}
        \begin{expl}
            $I = \bra y^3 - xy, x^2 - y \ket$. $y^3 - xy, x^2 - y$ bdg réduite pour degrevlex (matrice associée
            \begin{align*}
                \begin{bmatrix}
                    1 & 1 \\
                    0 & -1 \\
                    -1 & 0 \\
                \end{bmatrix}
            \end{align*})
            $w = (1,1)$. $v = (v_1, v_2) \in C[w]$, on a 
            \begin{align*}
                \begin{cases}
                    v \cdot (0,3) > v \cdot (1,1) \\
                    v \cdot (2,0) > v \cdot (0,1) \\
                \end{cases}
                \iff \begin{cases}
                    -v_1 + 2v_2 > 0 \\
                    2v_1 - v_2 > 0 \\
                \end{cases}
            \end{align*}
        \end{expl}
        \begin{coro}
            $\overline{C[w]}$ est un cone.
        \end{coro}
        \begin{proof}
            $\overline{C[w]}$ est défini en remplaçant les inégalités strictes du dernier corollaire par des $\leq$.
        \end{proof}
        \begin{defi}
            On définit l'éventail de groebner $GF(I)$ comme l'ensemble des $\overline{C[w]}$.
        \end{defi}
        \begin{theo}
            $GF(I)$ est un éventail fini.
        \end{theo}
        \begin{proof}
            Il faut montrer que
            \begin{enumerate}
                \item Toute face de $\overline{C[w]}$ a la forme $\overline{C[w']}$
                \item $\overline{C[w]} \cap \overline{C[w]}$ est une face de $\overline{C[w]}$ et $\overline{C[w']}$.
                \item $GF(I)$ est fini \cor{(On le montrera plus tard)}.
            \end{enumerate}
            Prouvons les deux premiers points :
            \begin{enumerate}
                \item Soit $F$ une face de $\overline{C[w]}$. $F$ est défini en remplaçant certaines des inégalités "$>0$" dans la définition de $C[w]$ par "$=0$". Soit $w' \in F$, pour tout $g \in \bar G$, les termes de $LT_w(g)$ sont tous des termes de $LT_{w'}(g)$. Posons
                \begin{align*}
                    \overline{\overline{M}} = \begin{bmatrix} w' \\ w \\ M \end{bmatrix}
                \end{align*}
                Alors
                \begin{align}
                    \bra LT_{\overline{\overline{M}}}(I) \ket &= \bra LT_{w'} \bra LT_{\bar M}(I) \ket \ket \\
                    &= \bra LT_{w'} \bra LT_{\bar M}(\bar G) \ket \ket \\
                    &= \bra LT_{\bar M}(\bar G) \ket \\
                    &= \bra LT_{\bar M}(I) \ket
                \end{align}
                (3.3) = (3.4) : $f \in \bra LT_{\bar M}(\bar G) \ket$, alors
                \begin{align*}
                    &\sum_{i = 1}^r g_i LT_{\bar M}(g_i) = \sum_{d \geq 0} \sum_i (g_i)_{d - \deg_{w'}LT_{\bar M}(g_i)} LT_{\bar M}(g_i) \\
                    \Rightarrow &LT_{w'}(f) = \sum_i (g_i)_{d - \deg_{w'}LT_{\bar M}(g_i)} LT_{\bar M}(g_i) \in \bra LT_{\bar M}(\bar G) \ket 
                \end{align*}
                En particulier, $\bar G$ est une bdg pour $w_{\overline{\overline{M}}}$. Donc
                \begin{align*}
                    C[w'] = \{w'' \in \mathbb{R}_+^n \mid \forall g \in \bar G,\, LT_{w'}(g) = LT_{w''}(g) \}
                \end{align*}
                Si $w'$ est "générique" (i.e. que toute inégalité définissant $F$ est stricte pour $w'$). Alors $\overline{C[w']} = F$ car $w'' \in \overline{C[w']}$ ssi $w''$ satisfaisant aux mêmes (in)égalités que $w'$.
                \item Soient $w,w' \in \mathbb{R}_+^n$. Condiréons $\overline{C[w]} \cap \overline{C[w']}$. Si $w'' \in \overline{C[w]} \cap \overline{C[w']}$, alors $\overline{C[w'']}$ est une face de $\overline{C[w]}$ et de $\overline{C[w']}$, par ce qui précède. Prenons $w''$ dans l'intérieur relatif, on obtiens que $\overline{C[w'']} = \overline{C[w]} \cap \overline{C[w']}$.
            \end{enumerate}
        \end{proof}
        \begin{remq}
            En sage, on dispose de la procédure \mintinline{python}{groebner_fan()}
        \end{remq}

        \section{Le cône maximal d'une bdg marquée}
            Soit $G = \{(g_i, x^{\alpha_i})\}_{1 \leq i \leq r}$ une bdgm. Écrivons
            \begin{align*}
                g_i = x^{\alpha_i} + \sum_{\beta \neq \alpha_i} c_{i, \beta} x^\beta
            \end{align*}
            Fixons $<_M$ ordre monomial pour lequel $g$ est la bdgm. Notons $w$ la première ligne de $M$. Alors $\forall i \in \lcc 1,r \rcc$, $\forall \beta$ tq $c_{i, \beta} \neq 0$,
            \begin{align*}
                w \cdot \alpha_i \geq  w \cdot \beta \iff w \cdot (\alpha_i - \beta) \geq 0
            \end{align*}
            \begin{lemm}
                Il existe $w' \in \mathbb{R}^n_+$ tel que $\forall i \in \lcc 1,r \rcc$, $\forall \beta \neq \alpha_i$ t.q. $c_{i, \beta} \neq 0$,
                \begin{align*}
                    w' \cdot \alpha_i > w' \cdot \beta
                \end{align*}
                Alors $G$ est une bdgm pour $<_{\begin{bmatrix} w' \\ M \end{bmatrix}}$
            \end{lemm}
            \begin{proof}
                On peut modifier $w$ ainsi : S'il existe $i, \beta$ tq $c_{i, \beta} \neq 0$ mais $w \cdot \alpha_i = w \cdot \beta$. Alors $w \in (\alpha_i - \beta)^\bot$. Soit $v \in \mathbb{R}^n$ t.q. $v \cdot (\alpha_i - \beta) > 0$. Alors $(w + \varepsilon v) \cdot (\alpha_i - \beta) > 0$ pour tout $\varepsilon > 0$. Pour que les autres inégalités soient respectées, il sufit de prendre $0 < \varepsilon$ petit. On montre que si $w \cdot (\alpha_j - \beta') = 0$, on peut choisir $v$ non-nul dans $(\alpha_j - \beta')^\bot$.
            \end{proof}
            \begin{defi}
                Avec les notations précédentes, le cône de $G$ est $C_G := C[w']$.
            \end{defi}
            \begin{remq}
                $C_G$ est de dimension $n$, car c'est l'intersection d'un nombre fini de demi-espaces ouverts. En particulier, $\bar C_G$ est un cône maximal de l'éventail de Groebner. De plus, tout cône maximal de $GF(I)$ a cette forme. Tout cône a la forme $\overline{C[w]}$. Par ce qui précède, on peut trouver $w'' \in \mathbb{R}^n_+$ tel que $\bar G$ est la bdg pour $\begin{bmatrix} w'' \\ M \end{bmatrix}$ et $LT_{w''}(g)$ sont dest monômes. Donc $\overline{C[w]} \subseteq \overline{C[w'']}$. Comme $\overline{C[w'']}$ est de dimension $n$, tous les cônes maximaux de $GF(I)$ sont de dimension $n$ et ont la forme $\overline{C[w'']}$.
            \end{remq}
            \begin{coro}
                $GF(I)$ est fini.
            \end{coro}
            \begin{expl}
                $k[x,y]$, $I = \bra x^2 - y, xy - y^3, y^5 - y^2 \ket$. Calculons $GF(I)$ : ses cônes maximaux ont la forme $\bar C_G$.
            \end{expl}

        \section{Changement de base de Groebner}
        Soit $I \subrel{id} k[x_1, \cdots, x_n]$
        \begin{itemize}
            \item Soit $G_0$ une bdg marquée de $I$ pour $<_{M_0}$, on note $w_0$ la première ligne de $M_0$, et on calcule $C_{G_0}$.
            \item On cherche $G_1$ une bdg marquée de $I$ pour $<_{M_1}$, on note $w_1$ la première ligne de $M_1$.
            \item On trace dans $GF(I)$ un segment de droite de $w_0$ à $w_1$. Soit $w_{der}$ de dernier point du segment $[w_0, w_1]$ qui soit toujours dans $\overline{C_{G_0}}$ (il existe et est unique par convexité de $\overline{C_{G_0}}$).
            \item Posons
            \begin{align*}
                M_0' =
                \begin{bmatrix}
                    w_0 \\
                    M_1 \\
                \end{bmatrix}
                ,\, M_{der} =
                \begin{bmatrix}
                    w_{der} \\
                    M_1 \\
                \end{bmatrix}
            \end{align*}
            et $G_0'$ et $G_{der}$ les base de Groebner associées.
        \end{itemize}
        Comment passer de $G_0$ à $G_0'$ ?
        
        \subsection{De $G_0$ à $G_0'$}
            On sait que
            \begin{itemize}
                \item $LT_{w_0}(G_0)$ est une bdg réduite de $\bra LT_{w_0}(I) \ket$ pour $M_0$.
                \item $LT_{w_0}(G_0')$ est une bdg réduite de $\bra LT_{w_0}(I) \ket$ pour $M_1$.
            \end{itemize}
            Soit $H = \{h_1, \cdots, h_s\}$ la bdg réduite de $\bra LT_{w_0}(I)\ket$ pour $M_1$.
            \begin{remq}
                La base de Groebner de $\bra LT_{w_0}(I) \ket = \bra LT_{w_0}(G_0) \ket$ est facile à calculer car les $LT_{w_0}(G_0)$ sont "presque des monômes".
            \end{remq}
            Ecrivons 
            \begin{align*}
                h_j = \sum_{g \in G_0} P_{j,g} LT_{w_0}(g)
            \end{align*}
            comme résultat de la division multivariée par $LT_{w_0}(G_0)$ pour $M_0$. Posons 
            \begin{align*}
                \overline{h_j} = \sum_{g \in G_0} P_{j,g} g \in I
            \end{align*}
            Alors $\overline{H} = \{\overline{h_1}, \cdots, \overline{h_s}\}$ est une bdg de $I$ pour $M_0'$. En effet, 
            \begin{align*}
                \bra LT_{M_0'}(I) \ket &= \bra LT_{M_1} \bra LT_{w_0}(I) \ket \ket \\
                &= \bra LT_{M_1}(H) \ket \\
                &= \bra LT_{M_0'}(\overline{H}) \ket \\
            \end{align*}
            Prenons $G_0'$ la bdg réduite qui est la réduction de $\overline{H}$.

        \subsection{De $G_0'$ à $G_{der}$}
            \begin{itemize}
                \item $LT_{w_0}(G_0')$ est une bdg réduite de $\bra LT_{w_0}(I) \ket$ pour $M_1$.
                \item $LT_{w_{der}}(G_0')$ est une bdg réduite de $\bra LT_{w_{der}}(I) \ket$ pour $M_0'$.
            \end{itemize}
            car $G_0'$ est une bdg de $I$ pour $\begin{bmatrix} w_{der} \\ M_0' \end{bmatrix}$. En effet, si $f \in I$, alors réalisons l'algorithme de division multivariée :
            \begin{align}
                f = \sum_{g \in G_0'} q_gg
            \end{align}
            Alors $LT_{M_0'}(f) = \max_{g \in G_0'}(LT_{M_0'} q_gg)$. Donc $LT_{w_{der}}(f)$ contiens $LT_{M_0'}(f)$ comme terme, car $w_{der} \in \overline{C_{G_0'}}$. On en déduit que
            \begin{align*}
                LT_{\begin{bmatrix} w_{der} \\ M_0' \end{bmatrix}}(f) = LT_{M_0'}(f)
            \end{align*}
            et ainsi
            \begin{align*}
                \bra LT_{M_0'}(I) \ket = \bra LT_{\begin{bmatrix} w_{der} \\ M_0' \end{bmatrix}}(I) \ket
            \end{align*}
            Finalement
            \begin{align*}
                \bra LT_{\begin{bmatrix} w_{der} \\ M_0' \end{bmatrix}}(I) \ket &= \bra LT_{M_0'}(I) \ket \\
                &= \bra LT_{M_0'}(G_0') \ket \\
                &= \bra LT_{M_0'} LT_{w_{der}} G_0' \ket \\
                &= \bra LT_{\begin{bmatrix} w_{der} \\ M_0' \end{bmatrix}}(G_0') \ket
            \end{align*}
            Soit $H = \{h_1, \cdots, h_s\}$ une bdg de $\bra LT_{w_{der}}(I) \ket$ pour $M_{der}$.
            \begin{remq}
                Le calcul de $H$ est peu couteux, car les éléments de $LT_{w_{der}}(G_0')$ sont presque tous des monômes. Posons
                \begin{align*}
                    h_j = \sum_{g \in G_0'} P_{j,g}LT_{w_{der}}(g)
                \end{align*}
                (division multivariée), et
                \begin{align*}
                    \overline{h_j} = \sum_{g \in G_0'} P_{j,g}g \in I
                \end{align*}
                Alors $\overline{H} = \{\overline{h_1}, \cdots, \overline{h_s}\}$ est une bdg de $I$ pour $M_{der} = \begin{bmatrix} w_{der} \\ M_1 \end{bmatrix}$ :
                \begin{align*}
                    \bra LT_{M_{der}}(I) \ket &= \bra LT_{M_{der}} \bra LT_{w_{der}}(I) \ket \ket \\
                    &= \bra LT_{M_{der}}(H) \ket \\
                    &= \bra LT_{M_{der}}(\overline{H}) \ket \\
                \end{align*}
                Finalement, posons $G_{der}$ la réduction de $\overline{H}$.
            \end{remq}

        \subsection{$\overline{C_{G_{der}}}$ est plus proche de $w_1$ que $C_{G_0'}$}
            Il suffit de voir que si on avance un peu sur le segment qui relie $w_{der}$ à $w_1$, on reste dans $\overline{C_{G_{der}}}$. Si $g \in G_{der}$, alors $LT_{M_{der}}(g)$ est un terme de $LG_{w_{der}}(g)$. Si on remplace $w_{der}$ par $w_{der} + \varepsilon(w_1 - w_{der})$, $0 < \varepsilon$ petit. Alors dans ce cas $LT_{M_{der}}(g)$ ne change pas. Ainsi on peut répéter les étapes précédentes jusqu'à obtenir $w_1$.
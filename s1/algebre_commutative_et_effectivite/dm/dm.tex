\documentclass[11pt]{article}

\usepackage{import}
\import{D:/Bureau/Documents/Maths/Latex/Packages/}{article.tex}

% Couleur de correction
% \newcommand{\cor}[1]{{#1}}
% \newcommand{\cor}[1]{{\color{red} #1}}

\begin{document}

%Page de garde
\title{Devoir maison}
\date{\today}
\author{Alexandre Guillemot}
\maketitle

%Document
\section{Préliminaires et divers}
    \begin{question}{1.}
        \label{q11}
        Le morphisme canonique $\pi : R \to R/I$ induit une bijection entre les idéaux de $R$ contenant $I$ et les idéaux de $R/I$ (en envoyant un tel idéal $I \subseteq J \subrel{id} R$ sur $\pi(I)$). Alors remarquons que
        \begin{align*}
            \pi(\sqrt{I}) &= \{\pi(x) \in R/I \mid x \in \sqrt{I}\} \\
            &= \{\pi(x) \in R/I \mid x \in R \text{ et } \exists N \in \mathbb{N},\, x^N \in I\} \\
            &= \{\pi(x) \in R/I \mid x \in R \text{ et } \exists N \in \mathbb{N},\, \pi(x)^N = 0\} \\
            &= \{y \in R/I \mid \exists N \in \mathbb{N},\, y^N = 0\} = \sqrt{(0)}
        \end{align*}
        (Plus généralement on a $\pi(\sqrt{J}) = \sqrt{\pi(J)}$ pour tout $I \subseteq J \subrel{id} R$). Ainsi si $I = \sqrt{I}$, alors $\sqrt{(0)} = \pi(I) = \{0\}$, et si $\sqrt{(0)} = \{0\}$, $\pi(\sqrt{I}) = \{0\}$ et donc $\sqrt{I} \subseteq I$, d'où $\sqrt{I} = I$.
    \end{question}
    \begin{question}{2.}
        Pour calculer la dimension d'un idéal, on calcule une famille génératrice de $I \cap k[y_1, \cdots, y_n]$ pour tout $\{y_1, \cdots, y_d\} \subseteq \{x_1, \cdots, x_n\}$, et on renvoie le plus grand $d$ tel que $I \cap k[y_1, \cdots, y_n] = \{0\}$. La famille génératrice est calculée en calculant une base de grobner de $I$ pour l'ordre lexicographique induit par $z_1 < \cdots z_{n - d} < y_1 < \cdots < y_n$, avec $\{z_1, \cdots, z_{n - d}\} = \{x_1, \cdots, x_n\} \bs \{y_1, \cdots, y_d\}$, et en éliminant les $n - d$ premières variables (c'est une base de grobner pour $k[y_1, \cdots, y_d] \cap I$ d'après le théorème d'élimination, donc en particulier une famille génératrice)
    \end{question}
    \begin{question}{3.}
        \label{q3}
        \item Considérons la factorisation en irréductibles unitaires
        \begin{align}
            h = \lambda \prod_{i = 1}^s h_i^{\alpha_i}
        \end{align}
        où $\lambda \in L$, $h_i \in L[x_1, \cdots, x_n]$ sont deux à deux distincts et $\alpha_i \in \mathbb{Z}_{>0}$. Alors considérons $\ell = \prod_{i = 1}^s h_i$, comme les $h_i$ sont deux à deux distincts, $\ell$ est sans carré. Maintenant soit $f \in L[x_1, \cdots, x_n]$ sans facteurs carrés, tel que $f \mid h$, alors $f$ s'écrit comme
        \begin{align*}
            f = \lambda' \prod_{i = 1}^s h_i^{\beta_i}
        \end{align*}
        avec $\lambda' \in L$, $0 \leq \beta_i \leq \alpha_i$ pour tout $1 \leq i \leq s$. Mais comme $f$ est sans facteurs carrés, on doit avoir que pour tout $1 \leq i \leq s$, $\beta_i \leq 1$, et ainsi $f \mid g$. Finalement, si on dispose d'un autre tel diviseur sans carré maximal $\ell'$, alors $\ell  \mid \ell '$ et $\ell ' \mid \ell $, donc $\ell $ et $\ell '$ sont associés dans $L[x_1, \cdots, x_n]$, i.e. $\exists \mu \in L$ tel que $\ell  = \mu \ell '$ et donc $\ell $ est unique à multiplication par un scalaire près.
        \item Montrons que $\sqrt{\bra h \ket} = \bra \ell \ket$ :
        \begin{enumerate}
            \item $\supseteq$ : Il suffit de montrer que $\ell \in \sqrt{\bra h \ket}$ : Pour cela, considérons $\alpha := \max_{1 \leq i \leq s} \{\alpha_i\}$, alors $h \mid \ell^\alpha$ au vu de la définition de $\ell$, et donc $\ell^\alpha \in \bra h \ket$, i.e. $\ell \in \sqrt{\bra h \ket}$.
            \item $\subseteq$ : Soit $f \in \sqrt{\bra h \ket}$, alors $\exists N \geq 0$ tel que $f^N \in \bra h \ket$, i.e. $\exists N \geq 0$ et $g \in L[x_1, \cdots, x_n]$ tels que $f^N = gh$. Maintenant pour tout $1 \leq i \leq s$, $h_i \mid gh$ donc $h_i \mid f^N$, et comme $h_i$ est irréductible, $h_i \mid f$, et donc finalement $\ell = \prod h_i \mid f$ puisque les $h_i$ sont irréductibles et distincts deux à deux.
        \end{enumerate}
    \end{question}

\section{Cas d'un idéal de dimension nulle}
    \begin{question}{1.}
        \item Pour calculer les $h_i$, on calcule une base de grobner $G$ pour $I$ et l'ordre lexicographique induit par $x_1 < x_2 < \cdots < x_{i-1} < x_{i+1} < \cdots < x_n < x_i$, puis on intersecte $G$ avec $k[x_i]$. Ainsi d'après le théorème d'élimination, $k[x_i] \cap G$ est une base de Groebner pour $k[x_i] \cap I$, donc en particulier un ensemble générateur. Il suffit alors de prendre le plus petit élément pour la relation de divisibilité dans $k[x_i] \cap G$. Remarquons que si la base $G$ calculée est réduite, alors on ne peut pas avoir deux éléments dans $G \cap k[x_i]$ (car si on a au moins deux éléments le terme dominant de l'un doit diviser le terme dominant de l'autre), ce qui contredit la définition d'une base de Groebner réduite. Ainsi dans ce cas il suffit de prendre un (le seul !) élément de $G \cap k[x_i]$.
        \item Pour calculer les $\ell_i$, on calcule $h_i/\mathrm{gcd}(h_i, h_i')$. Pour prouver la correction de cette méthode, prouvons que si $k$ est un corps de caractéristique $0$ et $P = \lambda \prod_{i = 1}^s P_i^{\alpha_i} \in k[x]$ la décomposition d'un polynôme en irréductibles, alors
        \begin{align*}
            \mathrm{gcd}(P,P') = \prod_{i = 1}^s P_i^{\alpha_i - 1} =: Q
        \end{align*}
        Dans un premier temps, on travaille toujours à produit par une constante près, donc on peut supposer que $\lambda = 1$. Pour prouver l'affirmation précédente, alculons $P'$ :
        \begin{align*}
            P' &= \sum_{i = 1}^s \left(\alpha_i P_i^{\alpha_i - 1}\prod_{j \neq i} P_j^{\alpha_j}\right) \\
            &= \sum_{i = 1}^s \alpha_i \frac P{P_i}
        \end{align*}
        Alors $Q$ divise $P$, puis $Q$ divise $P'$ comme $k$ est de caractéristique $0$, et ainsi $Q \mid \mathrm{gcd}(P, P')$. Et comme $\mathrm{gcd}(P,P') \mid P$, on a
        \begin{align*}
            \mathrm{gcd}(P, P') = \prod_{i = 1}^s P_i^{\beta_i}
        \end{align*}
        avec $\beta_i \in \{\alpha_i - 1, \alpha_i\}$. Maintenant s'il existe $1 \leq i \leq s$ tel que $\beta_i = \alpha_i$, alors $P_i^{\alpha_i}$ divise $\mathrm{gcd}(P,P')$ donc $P'$, or $P_i^{\alpha_i} \mid (P/P_j)$ pour tout $j \neq i$, et donc on aurait $P_i^{\alpha_i} \mid (P/P_i)$, absurde. On a donc finalement $\mathrm{gcd}(P,P') = Q$, et on a bien
        \begin{align*}
            \frac PQ = \prod_{i = 1}^s P_i
        \end{align*}
        qui est le plus grand diviseur sans facteurs carrés de $P$.
    \end{question}
    \begin{question}{2.}
        \label{q22}
        On a $I \subseteq \sqrt{I}$, puis $\ell_i \in \sqrt{\bra h_i \ket}$ d'après la \hyperref[q3]{question 3}, et $h_i \in I$ donc $\sqrt{\bra h_i \ket} \subseteq \sqrt{I}$. Ainsi $I + \bra l_1, \cdots, l_n \ket \subseteq \sqrt{I}$.
    \end{question}
    \begin{question}{3.a.}
        \label{q3a}
        Rappelons le théorème des restes chinois dans un anneau :
        \begin{theo}
            \label{theo21}
            Soit $R$ un anneau, puis $(I_i)_{1 \leq i \leq s} \subseteq R$ une famille d'idéaux premiers entre eux deux à deux [$I_i + I_j = R$ pour tout $i \neq j$]. Notons $I = \prod_{i = 1}^s I_i$, alors
            \begin{align*}
                R/I \simeq \prod_{i = 1}^s (R/I_i)
            \end{align*}
        \end{theo}
        \begin{proof}
            Par récurrence, il suffit de le prouver pour $s = 2$. Ainsi considérons le morphisme canonique
            \begin{align*}
                \varphi : R \to R/I \times R/J
            \end{align*}
            Alors pour tout $x \in I$, $y \in J$, $\varphi(xy) = 0$, donc $IJ \subseteq \ker \varphi$, puis si $\varphi(r) = 0$, alors $r[I] = 0$ et $r[J] = 0$. On a donc $r \in I \cap J$, mais comme $I + J = R$, il existe $a \in I$, $b \in J$ tels que $a + b = 1$, et alors
            \begin{align*}
                r = r(a + b) = ra + rb \in IJ
            \end{align*}
            Ainsi $\ker \varphi = IJ$ et donc $R/IJ = R/I \times R/J$.
        \end{proof}
        Or ici on peut écrire $\ell_1 = \lambda \prod_{i = 1}^s \ell_1^i$, où les $\ell_1^i$ sont irréductibles et différents deux à deux (donc premier entre eux deux à deux). Ainsi $\langle \ell_1^i \rangle + \langle \ell_1^j \rangle = R$ pour tout $i \neq j$, puis
        \begin{align*}
            \prod_{i = 1}^s \bra \ell_1^i \ket = \bra \ell_1 \ket
        \end{align*}
        ($\prod_{i = 1}^s \bra \ell_1^i \ket$ est par définition engendré par $\prod_{i = 1}^s \ell_1^i = \ell_1/\lambda$). On a donc, d'après le théorème, que
        \begin{align*}
            k[x_1]/\bra \ell_1 \ket \simeq \prod_{i = 1}^s k[x_1]/\bra \ell_1^i \ket
        \end{align*}
        et les $k[x_1]/\bra \ell_1^i \ket$ sont des extensions finies de $k$ vu que les $\ell_1^i$ sont irréductibles.
    \end{question}
    \begin{question}{3.b.}
        Montrons plus généralement que
        \begin{lemm}
            Soient $A,B$ des anneaux commutatifs, alors
            \begin{align*}
                (A \times B)[x] \simeq A[x] \times B[x]
            \end{align*}
        \end{lemm}
        \begin{proof}
            On dispose d'un morphisme canonique $(A \times B)[x] \to A[x] \times B[x]$, induit par les morphismes $(A \times B)[x] \to A[x]$ et $(A \times B)[x] \to B[x]$, eux-mêmes induits par les projections canoniques $A \times B \to A$, $A \times B \to B$ :
            \begin{figure}[H]
                \centering
                \begin{tikzcd}
                    & A \arrow[r]                                                                               & {A[x]}                                 \\
A \times B \arrow[ru, "p"] \arrow[r] \arrow[rd, "q"'] & {(A \times B)[x]} \arrow[r, "\varphi", dashed] \arrow[ru, "{p[x]}"] \arrow[rd, "{q[x]}"'] & {A[x] \times B[x]} \arrow[d] \arrow[u] \\
                    & B \arrow[r]                                                                               & {B[x]}                                
\end{tikzcd}
            \end{figure}
            Comme le foncteur d'oubli des anneaux commutatifs vers les groupes abéliens réfléchis les isomorphismes (en effet, le foncteur d'oubli $\mathbf{CRings} \to \mathbf{Sets}$ les réfléchis, du fait qu'un morphisme d'anneau est un isomorphisme s'il est bijectif), il suffit de vérifier que $\varphi$ est un isomorphisme vu comme un morphisme de groupes abéliens. Maintenant pour tout anneau commutatif $C$, $C[x] \simeq \bigoplus_{\mathbb{N}} C$ en tant que groupes abéliens, et au travers de cet isomorphisme $\varphi$ correspond au morphisme canonique
            \begin{align*}
                \bigoplus_{\mathbb{N}} (A \oplus B) \to \left(\bigoplus_{\mathbb{N}} A\right) \oplus \left(\bigoplus_{\mathbb{N}} B\right)
            \end{align*}
            qui est bien un isomorphisme dans la catégorie des groupes abéliens (puisque les colimites commutent toujours avec les colimites).
        \end{proof}
        Finalement, par récurrence on conclut que le lemme précédent est vrai pour n'importe quel produit fini d'anneaux, et donc en particulier pour n'importe quel produit fini d'extensions finies d'un corps $k$.
    \end{question}
    \begin{question}{3.c.}
        \label{q3c}
        Comme précédemment, considérons un anneau commutatif $C$, et deux $C$-algèbres $A$ et $B$, puis un élément $\ell \in C[x]$. Alors si on regarde $\ell$ comme un élément de $A[x]$ et $B[x]$ via $C[x] \to A[x]$ et $C[x] \to B[x]$, on a
        \begin{align*}
            A[x]/\bra \ell \ket \times B[x]/\bra \ell \ket \simeq (A[x] \times B[x])/\bra (\ell, \ell) \ket
        \end{align*}
        (Plus généralement on a $A/I \times B/J \simeq (A \times B)/(I \times J)$ avec $A,B$ des anneaux commutatifs et $I \subrel{id} A$, $J \subrel{id} B$). Maintenant remarquons que le carré
        \begin{figure}[H]
            \centering
            \begin{tikzcd}
                {(A \times B)[x]} \arrow[r]           & {A[x] \times B[x]}           \\
                {C[x]} \arrow[u, "\psi"] \arrow[r, "\Delta"'] & {C[x] \times C[x]} \arrow[u]
                \end{tikzcd}
        \end{figure}
        est commutatif. En effet, le morphisme du haut est induit par les morphismes $C[x] \to (A \times B)[x] \to A[x]$ et $C[x] \to (A \times B)[x] \to B[x]$, et celui du bas par les morphismes $C[x] \to C[x] \to A[x]$ et $C[x] \to C[x] \to B[x]$, il suffit donc de vérifier qu'ils sont égaux. Mais le diagramme
        \begin{figure}[H]
            \centering
            \begin{tikzcd}
                & {A[x]}                                \\
{C[x]} \arrow[rd] \arrow[ru] \arrow[r] & {(A \times B)[x]} \arrow[d] \arrow[u] \\
                & {B[x]}                               
\end{tikzcd}
        \end{figure}
        commute puisque
        \begin{figure}[H]
            \centering
            \begin{tikzcd}
                & A                              \\
C \arrow[rd] \arrow[r] \arrow[ru] & A \times B \arrow[u] \arrow[d] \\
                & B                             
\end{tikzcd}
        \end{figure}
        commute. Les morphismes $C[x] \to (A \times B)[x] \to A[x] \times B[x]$ et $C[x] \to C[x] \times C[x] \to A[x] \times B[x]$ sont donc égaux par lemme de Yoneda. Finalement, $\psi(\ell)$ correspond avec $(\ell,\ell)$ au travers de $(A \times B)[x] \simeq A[x] \times B[x]$, et donc
        \begin{align*}
            (A[x] \times B[x])/\bra (\ell, \ell) \ket \simeq (A \times B)[x]/\bra \psi(\ell) \ket
        \end{align*}
        Et alors par récurrence on obtiens le résultat pour toute famille finie de $C$-algèbres, et en particulier pour toutes famille d'extensions finies d'un corps $k$ :
        \begin{align*}
            \left( \prod_{i = 1}^s F_i \right)[x]/\bra \psi(\ell) \ket \simeq \prod_{i = 1}^s (F_i[x]/\bra \ell \ket)
        \end{align*}
        où $\psi$ est le morphisme canonique $k[x] \to \left(\prod_{i = 1}^s F_i\right)[x]$.
    \end{question}
    \begin{question}{3.d.}
        \label{q3d}
        Dans cette question, on regarde les éléments de $k[x_1, \cdots, x_{t+1}]$ comme des éléments de $k[x_1, \cdots, x_t][x_{t+1}]$. Alors considérons les morphismes
        \begin{align*}
            &p : k[x_1, \cdots, x_t] \to k[x_1, \cdots, x_t]/\bra \ell_1, \cdots, \ell_t \ket \\
            &p[x_{t + 1}] : k[x_1, \cdots, x_t][x_{t+1}] \to \left(k[x_1, \cdots, x_t]/\bra \ell_1, \cdots, \ell_t \ket \right)[x_{t+1}] \\
            &q : \left(k[x_1, \cdots, x_t]/\bra \ell_1, \cdots, \ell_t \ket \right)[x_{t+1}] \to \left(k[x_1, \cdots, x_t]/\bra \ell_1, \cdots, \ell_t \ket \right)[x_{t+1}]/\bra \ell_{t + 1} \ket
        \end{align*}
        où $\ell_{t + 1}$ est un abus de notation pour $p[x_{t + 1}](\ell_{t + 1})$ dans le membre de droite du dernier morphisme. Alors déjà pour tout $1 \leq i \leq t$,  $p[x_{t + 1}](\ell_i) = 0$ (vu que $p(\ell_i) = 0$) et donc $(q \circ p[x_{t + 1}])(\ell_i) = 0$, puis $(q \circ p[x_{t + 1}])(\ell_{t + 1}) = 0$, donc $\bra \ell_1, \cdots, \ell_{t + 1} \ket \subseteq \ker (q \circ p[x_{t + 1}])$. Pour l'inclusion réciproque, prenons $P \in k[x_1, \cdots, x_t][x_{t+1}]$ qui vérifie $(q \circ p[x_{t + 1}])(P) = 0$. Alors on peut écrire $p[x_{t + 1}](P) = p[x_{t + 1}](Q_{t + 1})p[x_{t + 1}](\ell_{t + 1})$ pour un certain $Q_{t + 1} \in k[x_1, \cdots, x_n]$, et ainsi $p[x_{t + 1}](P - Q_{t + 1}\ell_{t + 1}) = 0$, et donc il existe $Q_1, \cdots, Q_t$ tels que
        \begin{align*}
            P - Q_{t + 1}\ell_{t + 1} = \sum_{i = 1}^t Q_i\ell_i
        \end{align*}
        et donc $P \in \bra \ell_1, \cdots, \ell_{t + 1} \ket$. On a donc prouvé que
        \begin{align*}
            k[x_1, \cdots, x_{t + 1}]/\bra \ell_1, \cdots, \ell_{t + 1} \ket \simeq \left(k[x_1, \cdots, x_t]/\bra \ell_1, \cdots, \ell_t \ket \right)[x_{t+1}]/\bra \ell_{t + 1} \ket
        \end{align*}
    \end{question}
    \begin{question}{3.e.}
        Prouvons par récurrence que pour tout $t \geq 1$, $R/\bra \ell_1, \cdots, \ell_t \ket$ est un produit d'extensions finies de $k$. Si $t = 1$, $R/\bra \ell_1 \ket$ est un produit d'extensions finies de corps, d'après la \hyperref[q3a]{question 3.a.}. Maintenant par récurrence, on a vu à la \hyperref[q3d]{question précédente} que
        \begin{align*}
            k[x_1, \cdots, x_{t + 1}]/\bra \ell_1, \cdots, \ell_{t + 1} \ket \simeq \left(k[x_1, \cdots, x_t]/\bra \ell_1, \cdots, \ell_t \ket \right)[x_{t+1}]/\bra \ell_{t + 1} \ket
        \end{align*}
        Alors par hypothèse de récurrence, $k[x_1, \cdots, x_t]/\bra \ell_1, \cdots, \ell_t \ket$ est un produit d'extensions finies de corps : notons
        \begin{align*}
            k[x_1, \cdots, x_t]/\bra \ell_1, \cdots, \ell_t \ket \simeq \prod_{i = 1}^s F_i
        \end{align*}
        Maintenant d'après la \hyperref[q3c]{question 3.c.} on a
        \begin{align*}
            \left(\prod_{i = 1}^s F_i \right)[x_{t+1}]/\bra \ell_{t + 1} \ket \simeq \prod_{i = 1}^s F_i[x_{t+1}]/\bra \ell_{t + 1} \ket
        \end{align*}
        Il reste donc à montrer que $F_i[x_{t+1}]/\bra \ell_{t + 1} \ket$ est un produit d'extensions finies de $k$. Pour cela, décomposons $\ell_{t + 1}$ dans $k[x_{t + 1}]$ en irréductibles (deux à deux disjoints puisque $\ell_{t + 1}$ est sans facteurs carrés) :
        \begin{align*}
            \ell_{t + 1} = \lambda \prod_{j = 1}^r f_j
        \end{align*}
        avec $f_j \in k[x_{t + 1}]$ pour tout $1 \leq j \leq r$. Maintenant comme les $f_j$ sont irréductibles et que $k$ est parfait, ils sont sans facteurs carrés dans $F_i[x_{t + 1}]$ : en effet, notons $L,L'$ respectivement des corps de décomposition pour $f_j \in k[x_{t + 1}]$ et $f_j \in F_i[x_{t + 1}]$, alors comme $L'$ est une extension de $k$ dans laquelle $f_j$ est scindé, c'est une extension de $L$ telle que le diagramme
        \begin{figure}[H]
            \centering
            \begin{tikzcd}
                & L \arrow[rd, hook]   &    \\
    k \arrow[ru, hook] \arrow[rd, hook] &                      & L' \\
                & F_i \arrow[ru, hook] &   
    \end{tikzcd}
        \end{figure}
        commute. Maintenant si $f_j$ à un facteur carré dans $F_i$, alors il a au moins une racine multiple dans $L'$, ce qui est absurde puisqu'il n'a pas de racine multiple dans $L$ (comme $k$ est parfait) donc dans $L'$. Finalement, les $f_j$ sont sans facteurs carrés, et doivent encore être premiers entre eux deux à deux (réaliser l'algorithme de gcd sur $F_i[x_{t + 1}]$ ou sur $k[x_{t + 1}]$ doit donner le même résultat, mais il donne un élément de $k$ dans $k[x_{t + 1}]$), donc $\ell_{t + 1}$ est encore sans facteurs carrés dans $F_i[x_{t + 1}]$, et on utilise de nouveau le \hyperref[theo21]{théorème des restes chinois} pour conclure que $F_i[x_{t + 1}]/\bra \ell_{t + 1} \ket$ est un produit d'extensions finies de $k$, pour tout $1 \leq i \leq s$.
    \end{question}
    \begin{question}{4.}
        \item Dans un premier temps, si $A,B$ sont des anneaux commutatifs, alors un idéal de $A \times B$ peut toujours s'écrire comme $I \times J$ avec $I,J \subrel{id} A,B$. En effet, si $K \subrel{id} A \times B$, alors considérons $I := \{i \in A \mid (i,0) \in K\}$, $J := \{j \in B \mid (0,j) \in  K\}$. Alors déjà, $K = I \times J$ puisque si $(i,j) \in K$, alors $(1,0) \times (i,j) = (i,0) \in K$ donc $i \in I$, puis de même, $j \in J$. Ensuite si $i \in I$, $j \in J$, alors $(i,j) = (i,0) + (0,j) \in K$. Enfin $I,J$ sont bien des idéaux du fait que $K$ est un idéal. Enfin, par récurrence, on a le même résultat pour un produit fini d'anneaux commutatifs.
        \item Ensuite, considérons un produit de corps $S = \prod_{i = 1}^s K_i$, alors les idéaux des $K_i$ sont soit $K_i$ tout entier, soit $\{0\}$, et comme un idéal $I$ de $S$ est un produit d'idéaux des $K_i$, on peut l'écrire $I = \prod_{i = 1}^s I_i$ avec $I_i = K_i$ ou $\{0\}$. Ainsi un quotient d'un produit de corps est encore un produit de corps (du fait que l'on a toujours $(A \times B)/(I \times J) \simeq A/I \times B/J$ avec $I,J \subrel{id} A,B \in \mathbf{CRings}$ et $\{0\} \times A \simeq A$). Mais si $x = (x_1, \cdots, x_s) \in S = \prod_{i = 1}^s K_i$ est un élément d'un produit de corps, tel que $x^N = 0$ pour un certain $N > 0$, alors $x_i^N = 0 \in K_i$ pour tout $i$ et donc $x_i = 0$ par intégrité de $K_i$, ce qui prouve que $x = 0$ et donc $S$ est réduit. 
        \item Pour conclure, il faut remarquer que pour tout $A$ anneau commutatif, $I,J \subrel{id} A$, si on note $\pi : A \to A/I$, on a $A/(I + J) \simeq (A/I)/\pi(J)$. Ainsi 
        \begin{align*}
            R/I + \bra \ell_1, \cdots, \ell_n \ket \simeq (R/\bra \ell_1, \cdots, \ell_n \ket)/\pi(I)
        \end{align*}
        où $\pi : R \to R/\bra \ell_1, \cdots, \ell_n \ket$ désigne la projection canonique. Mais comme $R/\bra \ell_1, \cdots, \ell_n \ket$ est un produit de corps, on en conclut par le point précédent que $R/\bra \ell_1, \cdots, \ell_n \ket + I$ est un anneau réduit.
    \end{question}
    \begin{question}{5.}
        D'après la \hyperref[q11]{question 1.} des préliminaires, $I + \bra \ell_1, \cdots, \ell_n \ket = \sqrt{I + \bra \ell_1, \cdots, \ell_n \ket}$, et $\sqrt{I} \subseteq \sqrt{I + \bra \ell_1, \cdots, \ell_n \ket}$, donc $\sqrt{I} \subseteq I + \bra \ell_1, \cdots, \ell_n \ket$. On conclut, grâce à la \hyperref[q22]{question 2.}, que
        \begin{align*}
            \sqrt{I} = I + \bra \ell_1, \cdots, \ell_n \ket
        \end{align*}
    \end{question}
    \begin{question}{6.}
        Montrons que $I + \bra \ell_1, \ell_2 \ket = I$ : $\ell_i$ est le plus grand diviseur sans carré de $x_i^p - t$, $i = 1,2$. Mais $x^p - t$ est irréductible dans $k(t)[x]$ : considérons un corps de décomposition $L$ de $x^p - t$, notons $\alpha$ une racine de $x^p - t$ dans $L$. Alors $\alpha^p = t \in L$, et ainsi $x^p - t = (x - \alpha)^p$, et donc $x^p - t$ n'a qu'une racine dans $L$ de multiplicité $p$. Maintenant si $x^p - t = PQ \in k(t)[x]$ avec $P,Q$ non inversibles, alors $P = (x - \alpha)^i$ dans $L$, avec $1 \leq i < p$. Maintenant le terme devant $x^{i - 1}$ vaut $i\alpha$, donc on doit avoir $\alpha \in k(t)$. Mais alors on aurait $t = \alpha^p \in k(t)$, ce qui est absurde. En effet, sinon écrivons $\alpha = f/g$, on aurait $f^p = tg^p$, mais alors en dérivant les deux membres de cette égalit on aurait
        \begin{align*}
            g^p = g^p + ptg^{p-1}g' = pf^{p - 1}f' = 0
        \end{align*}
        et donc $g^p = 0$, d'où $t = 0$, absurde. Ainsi $h_1,h_2$ sont irréductibles et donc $\ell_i = h_i$, d'où $\ell_i \in I$,  $i = 1,2$ et donc $I = I + \bra \ell_1, \ell_2 \ket$. Enfin, 
        \begin{align*}
            (x_1 - x_2)^p = x_1^p - x_2^p = h_1 - h_2 \in I
        \end{align*}
        donc $x_1 - x_2 \in \sqrt{I}$, mais $x_1 - x_2 \notin I$. En effet, si on avait $P,Q \in k(t)[x_1, x_2]$ tels que $x_1 - x_2 = x_1^pP + x_2^pQ - t(P + Q)$, alors tous les monômes du terme de droite font intervenir soit $x_1^p$, soit $x_2^p$, soit $t$, donc aucun d'eux peut être égal à $x_1$ ou $x_2$, absurde. Ainsi $\sqrt{I} \neq I = I + \bra \ell_1, \ell_2 \ket$.
    \end{question}
    
\section{Problème d'appartenance au radical}
    \begin{question}{1.}
        Prouvons l'implication directe : supposons que $f \in \sqrt{I}$, alors $\exists n > 0$ tel que $f^n \in I$. Mais il existe $Q \in k[x_1, \cdots, x_n, t]$ tel que $f^nt^n - 1 = (ft - 1)Q$, et donc 
        \begin{align*}
            1 = f^nt^n - (ft - 1)Q \in \bra I, ft - 1 \ket
        \end{align*}
    \end{question}
    \begin{question}{2.a}
        Pour prouver l'implication réciproque, on procède par contraposée : supposons que $f \notin \sqrt{I}$, comme $K$ est algébriquement clos, $f \notin \mathcal{I}(\mathcal{V}(I))$ (Nullstellensatz de Hilbert). Ainsi il existe $\underline{p} \in \mathcal{V}(I)$ tel que $f(\underline{p}) \neq 0$, et comme $\underline{p} \in \mathcal{V}(I)$, $g(\underline{p}) = 0$ pour tout $g \in I$.
    \end{question}
    \begin{question}{2.b}
        Si $g \in I$, alors $\varepsilon(g) = g(\underline{p}) = 0$, et $\varepsilon(ft - 1) = f(\underline{p})f(\underline{p})^{-1} - 1 = 0$, donc on a bien $\bra I, ft - 1 \ket \subseteq \ker \varepsilon$. Enfin, $\varepsilon(f) = f(\underline{p}) \neq 0$ donc $f \in k[x_1, \cdots, x_n, t] \bs \ker \varepsilon$, et ainsi $\bra I, ft - 1 \ket \subseteq \ker \varepsilon \nsubseteq k[x_1, \cdots, x_n, t]$. Ainsi $k[x_1, \cdots, x_n, t]/\bra I, ft - 1 \ket \neq \{0\}$, et donc $1 \notin \bra I, ft - 1 \ket$. 
    \end{question}

\end{document}

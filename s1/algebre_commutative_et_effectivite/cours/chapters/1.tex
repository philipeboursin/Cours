\chapter{Bases de Gröbner}
    Dans ce chapitre, tous les anneaux sont supposés commutatifs. Fixons dès à présent un corps $k$ (on supposera toujours qu'on dispose d'algorithmes pour les opérations du corps). 
    \section{Préliminaires}
        \begin{defi} (Anneau noéthérien)
            Un anneau est noéthérien si toute suite croissante d'idéaux $I_0 \subseteq I_1 \subseteq I_2 \subseteq \cdots$ est stationnaire i.e. 
            \begin{align*}
                \exists N \in \mathbb{N} \mid \forall m \geq N,\, I_m = I_N 
            \end{align*}
        \end{defi}
        \begin{prop}
            Un anneau est noéthérien si et seulement si tout idéal de $A$ est finiment engendré. 
        \end{prop}
        \begin{expl}
            Voici des exemples d'anneaux noéthériens/non noéthériens
            \begin{figure}[H]
                \centering
                \begin{tabular}{c|c}
                    Anneaux noéthériens & Anneaux non noéthériens \\
                    \hline
                    $\mathbb{Q}$ & $k[\mathbb{N}]$ \\
                    Plus généralement, tout corps $k$ & \\
                    $\mathbb{R}[x]$ & \\
                    Plus généralement, tout PID & \\
                    $\mathbb{Z}$ & \\
                    $k[x_1, \cdots, x_n]$ (conséquence de \ref{base_de_hilbert}) & \\
                    Anneaux finis & \\
                    Anneaux artiniens & \\
                \end{tabular}
            \end{figure}
        \end{expl}
        \begin{theo} (Théorème de la base de Hilbert)
            \label{base_de_hilbert}
            Soit $A$ un anneau noéthérien. Alors $A[x]$ est un anneau noéthérien.
        \end{theo}
        \begin{coro}
            Si $k$ est un corps, alors $k[x_1, \cdots, x_n]$ est noeth pour $n \in \mathbb{N}$.
        \end{coro}
        \begin{proof}
            On veut montrer que tout idéal $I \subrel{id} A[x]$ est finiment engendré. Soit $I \subrel{id} A[x]$, montrons qu'il est finiment engendré. Pour chaque $n \in \mathbb{N}$, soit
            \begin{align*}
                I_n := \{a_n \in A \mid \exists a_0 + a_1x + \cdots + a_n x^n \in I\}
            \end{align*}
            Il est facile de voir que $I_n \subrel{id} A$. Ensuite $(I_i)$ est croissante, car si $a_i \in I_i$ pour un $i \in \mathbb{N}$, alors $\exists f \in I$ tq le coefficient directeur de $f$ soit $a_i$. Mais alors $xf(x) \in I$ est de degré $i+1$ et son coefficient directeur est encore $a_i$, d'où $a_i \in I_{i+1}$. Ainsi cette suite d'idéaux est stationnaire ($A$ noeth). Notons $N \in \mathbb{N}$ tq $m \geq N \Rightarrow I_m = I_N$. Les idéaux $I_0, \cdots, I_N$ sont finiment engendrés, notons $\{a_{i,j}\}_{1 \leq j \leq r_i}$ des familles génératrices pour $I_i$, pour tout $i \in \lcc 0, N \rcc$. Pour chaque $a_{i,j}$, $\exists f_{ij} \in I$ tq $\mathrm{deg}(f_{ij}) \leq i$ et le terme de degré $i$ de $f_{i,j}$ est $a_{i,j}$ (par définition de $I_i$). Montrons que $I = \bra \{f_{i,j}\}_{0,1 \leq i,j \leq N,r_i} \ket$ : soit $f \in I$,
            \begin{enumerate}
                \item si $\mathrm{deg}(f) = 0$, alors posons $a \in A$ tq $f = ax^0$. Ainsi $a \in I_0$, ainsi $\exists b_1, \cdots, b_{r_0}$ tq $a = \sum_{i = 1}^{r_0} b_i a_{0,i}$. Or $f_{0,i} = a_{0,i}x^0$, ainsi $f = \sum_{i = 1}^{r_0} b_i f_{0,i}$.
                \item Si $d = \mathrm{deg} f > 0$, notons $b$ le coeff directeur de $f$. Ainsi $b \in I_d$ \\
                \textbf{Cas où $d \leq N$ :} On peut écrire $b = \sum_{i = 1}^{r_d} \lambda_i a_{d,i}$ avec $\lambda_i \in A$. Posons $S = \sum_{i = 1}^{r_d} \lambda_i f_{d,i}$, alors le coefficient directeur de $S$ est précisément $b$ (et $\mathrm{deg} S \leq d$). Ainsi $\mathrm{deg} (f-S) < d$, et $f - S \in I$. Par hypothèse de récurrence, $f - S \in \bra \{f_{i,j}\} \ket$ et $S \in \bra \{f_{i,j}\} \ket$, donc finalement $f \in \bra \{f_{i,j}\} \ket$. \\
                \textbf{Cas où $d > N$ :} Notons $b$ le coeff directeur de $f$, $b \in I_d = I_N \Rightarrow b = \sum \lambda_i a_{N,i}$. Posons $T := \sum \lambda_i f_{N,i}X^{d-N}$ est de degré $d$ et de coeff directeur $b$, puis on conclut comme précedemment en regardant le polynômes $f - T$.
            \end{enumerate}
            Ainsi les idéaux de $A[x]$ sont finiment engendrés, donc $A[x]$ est noeth.
        \end{proof}

    \section{Division multivariée}
        \subsection{Ordres monomiaux}
            Fixons $k \in \mathbf{Fld}$. Rappelons que si $I \subrel{id} k[x]$ non nul, alors $\exists g \in k[x]$ t.q. $I = \bra g \ket$ (car $k[x]$ est principal, euclidien). Soit $f \in k[x]$, alors $f \in \bra g \ket \iff g \mid f \iff $ le reste de la division euclidienne de $f$ par $g$ est nul (et on dispose d'un algorithme pour réaliser la division euclidienne). Question : peut-on généraliser à $k[x_1, \cdots, x_n]$ ? 
            \begin{remq}
                Soit $I \subrel{id} k[x]$, $I = \bra f_1, \cdots, f_r \ket$. Alors $I = \bra \mathrm{pgcd}(f_1, \cdots, f_r) \ket$
            \end{remq}
            \begin{defi} (Ordre monomial)
                \label{ordre_mono}
                Un ordre monomial sur $k[x_1, \cdots, x_n]$ est une relation d'ordre $\leq$ sur l'ensemble des $\{x^\alpha = x_1^{\alpha_1} \cdots x_n^{\alpha_n} \mid \alpha \in \mathbb{N}^n\}$ tq
                \begin{enumerate}
                    \item $\leq$ est un ordre total (pour tout $x^\alpha, x^\beta \in k[x_1, \cdots, x_n]$, $(x^\alpha \leq x^\beta) \lor (x^\beta \leq x^\alpha)$).
                    \item $x^\alpha \leq x^\beta \Rightarrow \forall \gamma \in \mathbb{N}^n,\, x^{\alpha + \gamma} \leq x^{\beta + \gamma}$
                    \item $1 \leq x^\alpha$ pour tout $\alpha \in \mathbb{N}^n$.
                \end{enumerate}
            \end{defi}
            \begin{nota}
                On écrira $\alpha \leq \beta$ au lieu de $x^\alpha \leq x^\beta$ (la dernière condition devient alors $0 \leq \alpha$ pour tout $\alpha \in \mathbb{N}^n$)
            \end{nota}
            \begin{expl}
                \begin{enumerate}
                    \item Dans $k[x]$, il est facile de vérifier qu'il n'existe qu'un seul ordre monomial $\leq$ : $x^n \leq x^m \iff n \leq m$.
                    \item Ordre lexicographique $\leq_{lex}$ : soient $\alpha, \beta \in \mathbb{N}^n$ tq $\alpha \neq \beta$,
                    \begin{align*}
                        \alpha <_{lex} \beta \iff \exists 1 \leq r \leq n \mid \alpha_i = \beta_i \text{ pour } i < r \text{ et } \alpha_r < \beta_r
                    \end{align*}
                    (i.e. le premier coeff non nul d $\beta - \alpha$ est positif). Par exemple, dans $k[x_1, x_2, x_3]$, $x_1^2 >_{lex} x_1x_2 >_{lex} x_2^2 >_{lex} x_3^{2097434}$. Remarquons qu'avec cette définition, $x_1 >_{lex} x_2 >_{lex} \cdots >_{lex} x_n$.
                    \item Ordre lexicographique gradué $\leq_{deglex}$ : Pour $\alpha \in \mathbb{N}^n$, notons $|\alpha| = \sum \alpha_i$. Alors soient $\alpha \neq \beta$ dans $\mathbb{N}^n$,
                    \begin{align*}
                        \alpha <_{deglex} \beta \iff (|\alpha| < |\beta|) \lor (|\alpha| = |\beta| \land \alpha <_{lex} \beta)
                    \end{align*}
                    \item Ordre lexicographique renversé gradué $<_{degrevlex}$ :
                    \begin{align*}
                        \alpha <_{degrevlex} \beta \iff (|\alpha| < |\beta|) \lor (|\alpha| = |\beta| \land (\exists r \in \lcc 1,n \rcc  \mid \forall i \in \lcc r+1, n \rcc,\, \alpha_i = \beta_i  \text{ et } \alpha_r > \beta_r))
                    \end{align*}
                    (la deuxième condition reviens a vérifier que le dernier coeff non nul de $\beta - \alpha$ est négatif dans le cas où $|\alpha| = |\beta|$)
                \end{enumerate}
            \end{expl} \noindent
            \begin{exo}
                Vérifier que ces ordres sont des ordres monomiaux.
            \end{exo} \noindent
            Dans sage, on appelle "term orders" de tels ordres.
            \begin{prop}
                Soit $\leq$ un ordre sur $\mathbb{N}^n$ satisfaisant les propriétés $1$ et $2$ de la def \ref{ordre_mono}. Alors tfae
                \begin{enumerate}\addtocounter{enumi}{2}
                    \item $0_{\mathbb{N}^n} \leq \alpha ,\, \forall \alpha \in \mathbb{N}^n$
                    \item $\leq$ est un bon ordre : $\forall E \subseteq \mathbb{N}^n$ non vide, $E$ contient un élément minimal pour $<$.
                \end{enumerate}
            \end{prop}
            \begin{proof}
                4 $\Rightarrow$ 3 : Supposons qu'il existe $\alpha \in \mathbb{N}^n$ tq $\alpha < 0$, alors $2\alpha < \alpha$, $3\alpha < 2\alpha$ et ainsi de suite, donc $\cdots < 2\alpha < \alpha < 0$, mais alors $\{m\alpha \mid m \in \mathbb{N}\}$ n'a pas d'élément minimal, donc $\leq$ n'est pas un bon ordre. \\
                3 $\Rightarrow$ 4 : Supposons qu'il existe $F \subseteq \mathbb{N}^n$ non vide et sans élément minimal. Alors considérons l'idéal $I = \bra x^\alpha \mid \alpha \in F \ket$, d'après le théorème de la base de Hilbert, il existe un sous-ensemble fini de $F$, noté $\{\alpha_1, \cdots, \alpha_r\}$ tel que $I = \bra x^{\alpha_1}, \cdots, x^{\alpha_r}\ket$. Alors considérons $m = \min \{\alpha_1, \cdots, \alpha_r\}$, c'est un élément de $F$. Mais par hypothèse, il existe $\beta \in F$ tel que $\beta < s$. Mais comme $x^\beta \in I$, il existe $1 \leq i \leq r$ tel que $x^{\alpha_i} \mid x^\beta$, et ainsi $\beta - \alpha_i \in \mathbb{N}^n$. Mais $\beta - \alpha_i < 0$ car sinon on aurait $\beta \geq \alpha_i \geq m$. 
            \end{proof}

        \subsection{Algorithme de division multivariée}
            Fixons maintenant un ordre monomial $\leq$ sur $k[x_1, \cdots, x_n]$.
            \begin{defi}
                Soit $f = \sum_{\alpha \in \mathbb{N}^n} \lambda_\alpha x^\alpha \in k[x_1, \cdots, x_n] \bs \{0\}$, 
                \begin{enumerate}
                    \item Le multidegré de $f$ est $\mathrm{mdeg}(f) = \max \{\alpha \in \mathbb{N}^n \mid \lambda_\alpha \neq 0\}$
                    \item Le coefficient dominant de $f$ $\mathrm{LC}(f) = \lambda_{\mathrm{mdeg}(f)}$
                    \item Le mo,ome dominant de $f$ est $\mathrm{LM}(f) = x^{\mathrm{mdeg}(f)}$
                    \item Le terme dominant de $f$ est $\mathrm{LT}(f) = \lambda_{\mathrm{mdeg}(f)}x^{\mathrm{mdeg}(f)}$
                \end{enumerate}
            \end{defi}
            Soit $(f_1, \cdots, f_r)$ un $r$-tuple de polynômes non nuls de $k[x_1, \cdots, x_n]$. Soit $f \in k[x_1, \cdots, x_n]$, on cherche $Q_1, \cdots, Q_r, R \in k[x_1, \cdots, x_n]$ tq
            \begin{enumerate}
                \item $f = Q_1f_1 + \cdots + Q_r f_r + R$
                \item $R = 0$ ou aucun des termes de $R$ n'est divisible par $\mathrm{LT}(f_1), \cdots, \mathrm{LT}(f_r)$.
            \end{enumerate}
            
            \begin{algorithm}
                \caption{Réalise la division multivariée de $f$ par $f_1, \cdots, f_r$}
                \begin{algorithmic}
                    \Function{Division multivariée}{$f , f_1, \cdots, f_r \in k[x_1, \cdots, x_n]$}
                        \State $g \gets f$
                        \State $Q_1, \cdots, Q_r \gets 0$
                        \State $R \gets 0$
                        \While{$g \neq 0$}
                            \State $b \gets True$
                            \State $i \gets 1$
                            \While{$b$ \textbf{and} $i \leq r$}
                                \If{$\mathrm{LT}(f_i) \mid \mathrm{LT}(g)$}
                                    \State $g \gets g - \frac{\mathrm{LT}(g)}{\mathrm{LT}(f_i)} f_i$
                                    \State $Q_i \gets Q_i + \frac{\mathrm{LT}(g)}{\mathrm{LT}(f_i)}$
                                    \State $b \gets False$
                                \EndIf
                                \State $i \gets i + 1$
                            \EndWhile
                            \If{b}
                                \State $h \gets LT(g)$
                                \State $g \leftarrow g - h$
                                \State $R \leftarrow R + h$
                            \EndIf
                        \EndWhile
                        \State \Return $R,Q_1, \cdots, Q_r$
                    \EndFunction
                \end{algorithmic}
            \end{algorithm}

            \cleardoublepage

            \begin{remq}
                Après chaque tour de boucle while principale, on a toujours 
                \begin{align*}
                    f = g + \sum Q_if_i + R
                \end{align*}
                au vu des calculs réalisés dans la boucle. Et comme l'algorithme se termine lorsque $g = 0$, on obtiens finalement 
                \begin{align*}
                    f = \sum Q_if_i + R
                \end{align*}
                et aucun des termes de $R$ n'est divisible par $\mathrm{LT}(f_i)$ vu que l'on ajoute que des termes divisibles par aucun des $\mathrm{LT}(f_i)$ dans l'algorithme. Finalement, l'algorithme termine puisque à chaque étape de la boucle while principale, le multidegré de $g$ diminue strictement au vu des calculs effectués et du fait que $\leq$ est une relation d'ordre monomiale.
            \end{remq}
            \begin{nota}
                Le reste obtenu s'écrira $\bar f^{f_1, \cdots, f_t}$. Si $F = \{f_1, \cdots, f_r\}$, on écrira $\bar f^F$.
            \end{nota}
            \begin{remq}
                L'algo donne l'exitence de $Q_i$ et $R$ tq $f = \sum Q_if_i + R$ satisfaisant les conditions imposées précédemment. Ces $Q_i$ et $R$ ne sont pas uniques.
            \end{remq}
            \begin{expl}
                $k[x_1, x_2]$, $<_{lex} =: <$, $f = x_1^2 + x_1x_2 + x_2^2$, $f_1 = x_1$, $f_2 = x_1 + x_2$. Alors 
                \begin{align*}
                    f &= (x_1 + x_2)f_1 + x_2^2 \\
                    \intertext{(Résultat obtenu en appliquant l'algorithme de division multivariée)}
                    &= x_1f_2 + x_2^2 \\
                    &= x_1f_1 + x_2f_2 + 0 \\
                \end{align*}
                donc $f \in (f_1, f_2)$ mais $\bar f^{f_1, f_2} \neq 0$ !
            \end{expl}

    \section{Bases de Gröbner}
        \begin{defi} (Base de Gröbner, 1)
            \label{grob_1}
            Soit $I \subrel{id} k[x_1, \cdots, x_n]$ non nul. Une base de Gröbner de $I$ est un ensemble fini $G \subseteq I$ tq
            \begin{enumerate}
                \item $I = \bra G \ket$,
                \item $f \in I \iff \bar f^G = 0$
            \end{enumerate}
        \end{defi}
        Par convention, $\emptyset$ est une base de Gröbner de l'idéal nul.
        \begin{expl}
            \begin{enumerate}
                \item Si $0 \neq g \in k[x]$, alors $\{g\}$ est une bdG (base de Gröbner) de $\bra g \ket$.
                \item Si $0 \neq g \in k[x_1, \cdots, x_n]$, alors $\{g\}$ est une bdG de $\bra g \ket$.
            \end{enumerate}
        \end{expl}
        Comment peut-on avoir $f \in \bra f_1, \cdots, f_r \ket$ mais $\bar f^{f_1, \cdots, f_r} \neq 0$ ? Il faut qu'à une étape de la division, $\mathrm{LT}(f)$ ne soit pas divisible par aucun des $\mathrm{LT}(f_i)$. 
        \begin{defi} (Idéal monomial)
            Un idéal $I \subrel{id} k[x_1, \cdots, x_n]$ est monomial s'il existe des monômes $m_1, \cdots, m_r$ tq $I = \bra m_1, \cdots, m_r \ket$ (par convention $\{0\}$ est monomial).
        \end{defi}
        \begin{prop}
            \label{prop131}
            Soient $m_1, \cdots, m_r \in k[x_1, \cdots, x_n]$ des monômes, alors 
            \begin{align*}
                m \in \bra m_1, \cdots, m_r \ket \iff m \text{ est divisible par l'un des } m_i
            \end{align*}
        \end{prop}
        \begin{proof}
            Si $m$ est divisible par l'un des $m_i$, il est clair que $m \in \bra m_1, \cdots, m_r \ket$. Pour prouver l'implication réciproque, supposons que $m \in \bra m_1, \cdots, m_r \ket$. Alors on peut écrire
            \begin{align*}
                m = \sum_{i = 1}^r a_i m_i
            \end{align*}
            avec $a_i \in k[x_1, \cdots, x_n]$. Maintenant écrivons chaque $a_i$ comme
            \begin{align*}
                a_i(x) = \sum_{\alpha \in \mathbb{N}^n} \lambda_\alpha^i x^{\alpha}
            \end{align*}
            Alors
            \begin{align*}
                m = \sum_{i = 1}^r \sum_{\alpha \in \mathbb{N}^n} \lambda_\alpha^i x^{\alpha} m_i
            \end{align*}
            Maintenant comme $m$ est un monome, il va exister $i, \alpha, \lambda$ ($\lambda$ sera alors une somme de $\lambda_\alpha^i$) tels que $m = \lambda x^\alpha m_i$, donc $m_i \mid m$.
        \end{proof}
        Soient $f_1, \cdots, f_r \in k[x_1, \cdots, x_n]$. $\mathrm{LT}(f)$ divisible par l'un des $\mathrm{LT}(f_1), \cdots, \mathrm{LT}(f_r)$ si et seulement si $\mathrm{LT}(f) \in \bra \{\mathrm{LT}(f_i)\} \ket$ d'après la proposition précédente. \\
        \begin{nota}
            Soit $E \subseteq k[x_1, \cdots, x_n]$, on note
            \begin{align*}
                \mathrm{LT}(E) := \{\mathrm{LT}(f) \mid f \in E\}
            \end{align*}
        \end{nota}
        \begin{defi} (Base de Gröbner, 2)
            Une base de Gröbner d'un idéal $I \subrel{id} k[x_1, \cdots, x_n]$ est un ensemble (fini) $G \subseteq I$ tq $\bra \mathrm{LT}(I) \ket = \bra \mathrm{LT}(G) \ket$
        \end{defi}
        \begin{theo}
            Les deux définitions de bases de Gröbner sont équivalentes.
        \end{theo}
        \begin{proof}
            def 1 $\Rightarrow$ def 2 : Soit $f \in I$ si $\mathrm{LT}(f) \notin \bra \mathrm{LT}(G) \ket$, alors $\mathrm{LT}(f)$ n'est divisible par aucun des $\mathrm{LT}(g)$, $g \in G$ donc $\bar f^G \neq 0$. \\
            def 2 $\Rightarrow$ def 1 : Notons $G = \{g_1, \cdots, g_r\}$. Soit $f \in I$, on veut que $\bar f^G = 0$. Il suffit de montrer que le reste est nul à chaque étape de l'algo de division. Or à l'étape $0$ il l'est, puis en supposant qu'il l'est à l'étape $m$, on a
            \begin{align*}
                f = g + \sum Q_i g_i \in I
            \end{align*}
            et donc $g \in I$. Ainsi $LT(g) \in \bra LT(I) \ket = \bra LT(G) \ket$ et donc il existe un $g_i$ tel que $LT(g_i) \mid LT(g)$ daprès \ref{prop131}, et ainsi le reste est inchangé à cette étape.
            % Or
            % \begin{align*}
            %     f - \sum Q_i^{(m)} g_i - R^{(m)} = f^{(m)}
            % \end{align*}
            % et $f - \sum Q_i^{(m)} g_i \in I$. Si $R^{(m)}$, alors $f^{(m)} \in I$, donc $\mathrm{LT}(f^{(m)}) \in (\mathrm{LT}(G))$. D'où $R^{(m+1)} = 0$ puis récurrence.
        \end{proof}
        \begin{theo}
            Tout $I \subrel{id} k[x_1, \cdots, x_n]$ admet une base de Gröbner.
        \end{theo}
        \begin{proof}
            On cherche $G \subrel{fini} I$ tq $\bra \mathrm{LT}(G) \ket = \bra \mathrm{LT}(I) \ket$. D'après \ref{base_de_hilbert}, $\exists H \subrel{fini} \mathrm{LT}(I)$ tq $\bra H \ket = \bra \mathrm{LT}(I) \ket$. Notons $h_1, \cdots, h_r$ des polynômes de $I$ dont les termes dominants sont les éléments de $H$. Alors $\{h_1, \cdots, h_r\}$ est une BDG de $I$.
        \end{proof}

    \section{Algorithme de Buchberger}
        \begin{defi}
            $f,g \in k[x_1, \cdots, x_n]$, alors
            \begin{align*}
                S(f,g) := \frac{\mathrm{ppcm} (\mathrm{LM}(f), \mathrm{LM}(g))}{\mathrm{LT}(f)}f - \frac{\mathrm{ppcm} (\mathrm{LM}(f), \mathrm{LM}(g))}{\mathrm{LT}(g)}g
            \end{align*}
        \end{defi}
        \begin{theo} (Critère de Buchberger)
            Soit $G = \{g_1, \cdots, g_r\} \subseteq k[x_1, \cdots, x_r]$. Alors $G$ est une BDG de $I := \bra G \ket$ si et seulement si $\forall g,h \in G$, $\overline{S(g,h)}^G = 0$
        \end{theo}
        \begin{proof}
            $\Rightarrow$ : $G$ bdG, $f,g \in G$. Comme $S(f,g) \in I$, alors $\overline{S(f,g)}^G = 0$. \\
            $\Leftarrow$ : Supposons que pour tout $g,h \in G$, alors $\overline{S(g,h)}^G = 0$. Soit $f \in I$, on veut mq $LT(f) \in \bra LT(G) \ket$. Or $I = \bra g_1, \cdots, g_r \ket$, donc il existe $q_1, \cdots, q_r \in k[x_1, \cdots, x_n]$ tq 
            \begin{align*}
                f = \sum_{i = 1}^r q_ig_i
            \end{align*}
            Alors $LM(f) \leq \max_i \{LM(q_ig_i)\} = \mathbb{M}$.
            \begin{enumerate}
                \item Si $LM(f) = \mathbb{M}$ : Alors $LM(f) = LT(q_ig_i)$ pour un certain $i$. Mais $LM(q_ig_i) = LM(q_i)LM(g_i)$ et donc $LM(f) \in \bra LT(G) \ket$.
                \item Si $LM(f) < \mathbb{M}$ : Soit $1 \leq i_1 < i_2 < \cdots < i_s \leq r$ les indices tels que $LM(q_{i_j}g_{i_j}) = \mathbb{M}$. Alors on peut réécrire $f$ comme
                \begin{align*}
                    f = \sum_{j = 1}^s LT(q_{i_j})g_{i_j} + \sum_{i = 1}^r q_i'g_i
                \end{align*}
                (et donc $LM(q_i'g_i) < \mathbb{M}$). Considérons $\sum_j LT(q_{i_j})g_{i_j}$, on peut l'exprimer en fonction des $S(g_{i_j}, g_{i_{j+1}})$. Pour le voir, notons $h_j = LT(q_{i_j})g_{i_j}$, alors
                \begin{align*}
                    \sum_j h_j = &LC(h_1)\left( \frac{h_1}{LC(h_1)} - \frac{h_2}{LC(h_2)} \right) \\
                    &+ (LC(h_1) + LC(h_2))\left( \frac{h_2}{LC(h_2)} - \frac{h_3}{LC(h_3)} \right) \\
                    &+ (LC(h_1) + LC(h_2) + LC(h_3))\left( \frac{h_3}{LC(h_3)} - \frac{h_4}{LC(h_4)} \right) \\
                    & + \cdots \\
                    &+(LC(h_1) + \cdots + LC(h_{s-1}))\left( \frac{h_{s-1}}{LC(h_{s-1})} - \frac{h_s}{LC(h_s)} \right) \\
                    &+ (LC(h_1) + \cdots + LC(h_s))\frac{h_s}{LC(h_s)}
                \end{align*}
                Or $\sum_j LC(h_j) = 0$ car $LM(f) < \mathbb{M}$, donc le dernier terme s'annule et donc on a bien
                \begin{align*}
                    \sum_j h_j = \sum_{j = 1}^{s-1} \left(\sum_{k = 1}^j LC(h_k) \right) S(h_j,h_{j+1})
                \end{align*}
                \begin{remq}
                    Si $f$ et $g$ sont de même multidegré,
                    \begin{align*}
                        S(f,g) := \frac{1}{\mathrm{LC}(f)}f - \frac{1}{\mathrm{LC}(g)}g
                    \end{align*}
                    Ainsi,
                    \begin{align*}
                        S(h_j, h_{j+1}) = \frac{1}{LC(h_j)}h_j - \frac{1}{LC(h_{j+1})}h_{j+1}
                    \end{align*}
                \end{remq}
                De plus,
                \begin{align*}
                    S(h_j, h_{j+1}) &= \frac{1}{LC(h_j)}h_j - \frac{1}{LC(h_{j+1})}h_{j+1} \\
                    &= \frac{LT(q_{i_j})}{LC(q_{i_j}g_{i_j})}g_{i_j} - \frac{LT(q_{i_{j+1}})}{LC(q_{i_{j+1}}g_{i_{j+1}})}g_{i_{j+1}} \\
                    &= \frac{LM(q_{i_j})}{LC(g_{i_j})}g_{i_j} - \frac{LM(q_{i_{j+1}})}{LC(g_{i_{j+1}})}g_{i_{j+1}} \\
                    &= \frac{LM(g_{i_j}q_{i_j})}{LT(g_{i_j})}g_{i_j} - \frac{LM(g_{i_{j+1}}q_{i_{j+1}})}{LT(g_{i_{j+1}})}g_{i_{j+1}} \\
                    &= m_j S(g_{i_j}, g_{i_{j+1}})
                \end{align*}
                pour un certain monôme $m_j$. Donc
                \begin{align*}
                    f &= \sum_j LT(g_{i_j})g_{i_j} + \sum_i q_i'g_i \\
                    &= \sum_j h_j + \sum_i q_i'g_i \\
                    &= \sum_{j = 1}^{s-1} \left(\sum_{k = 1}^j LC(h_k) \right) S(h_j,h_{j+1}) + \sum_i q_i'g_i \\
                    &= \sum_{j = 1}^{s-1} m_j \left(\sum_{k = 1}^j LC(h_k) \right) S(g_{i_j},g_{i_{j+1}}) + \sum_i q_i'g_i \\
                \end{align*}
                et $\max (LM(q_i'g_i)) < \mathbb{M}$. Par hypothèse, $\overline{S(g_{i_j}, g_{i_{j+1}})}^G = 0$. Donc l'algorithme de division multivariée donne 
                \begin{align*}
                    S(g_{i_j}, g_{i_{j+1}}) = \sum_{i = 1}^r b_i^jg_i
                \end{align*}
                Par définition de l'algorithme, chaque $b_i^jq_i$ est de multidegré au plus $\mathrm{mdeg}(S(g_{i_j}, g_{i_{j+1}}))$. Mais alors
                \begin{align*}
                    \mathrm{mdeg}(m_j S(g_{i_j}, g_{i_{j+1}})) = \mathrm{mdeg}(S(h_j,h_{j+1})) < \mathbb{M}
                \end{align*}
                Donc
                \begin{align*}
                    f &= \sum_{j = 1}^{s-1} \left(\sum_{k = 1}^j LC(h_k) \right) m_j S(g_{i_j},g_{i_{j+1}}) + \sum_i q_i'g_i \\        
                    &= \sum c_ig_i                
                \end{align*}
                avec $LM(c_ig_i) < \mathbb{M}$. Par récurrence sur la différence entre $LM(f) - \mathbb{M}$, on peut conclure.
            \end{enumerate}
        \end{proof}
        \begin{coro} (Algorithme de Buchberger)
            Soit $I = \bra f_1, \cdots, f_r \ket \subrel{id} k[x_1, \cdots, x_n]$. Posons $G^0 = \{f_1, \cdots, f_r\}$ et pour $n \geq 1$, on définit
            \begin{align*}
                G^n = G^{n-1} \cup \left\{\overline{S(f,g)}^{G^{n-1}} \mid f,g \in G^{n-1},\, \overline{S(f,g)}^{G^{n-1}} \neq 0\right\} 
            \end{align*}
            Alors il existe $N \in \mathbb{N}$ tel que $n \geq N \Rightarrow G^n = G^N$. Dans ce cas, $G^N$ est une bdG de $I$.
        \end{coro}
        \begin{proof}
            Si $G^n = G^{n+1}$, alors par le critère de Buchberger $G^n$ est une bdG. Il faut donc montrer que la suite $(G^n)_n$ est stationnaire. Supposons le contraire, alors pour tout $n \geq 0$, $\exists f,g \in G^n$ tq $\overline{S(f,g)}^{G^n} \neq 0$. Par définition de l'algorithme de division multivariée, aucun des termes de $\overline{S(f,g)}^{G^n}$ n'est dans $\bra LT(G^n) \ket$. En particulier, $LT \left(\overline{S(f,g)}^{G^n} \right) \notin \bra LT(G^n) \ket$. On a donc $\bra LT(G^n) \ket \varsubsetneq \bra LT(G^{n+1}) \ket$ et donc on obtiens une suite d'idéaux strictement croissante dans $k[x_1, \cdots, x_n]$, contradiction.
        \end{proof}
        \begin{remq}
            L'algorithme de Buchberger n'est pas optimal. Pour des versions optimisées, voir les algorithmes F4 et F5 (Faugère)
        \end{remq}

    \section{Bases de Gröbner réduites, unicité}
        \begin{expl}
            $\bra x-y,y-z \ket = \bra x-z, y-z \ket \subrel{id} k[x_1, \cdots, x_n]$. Les deux couples de générateurs sont des bdG de l'idéal qu'ils engendrent pour l'ordre lex, on n'a donc pas toujours unicité des bases de Gröbner.
        \end{expl}
        \begin{defi} (bdG réduite)
            Soit $G$ une bdG de $I \subrel{id} k[x_1, \cdots, x_n]$. Cette base est réduite si
            \begin{enumerate}
                \item Pour tout $g \in G$, $LC(g) = 1$
                \item Pour tout $g,h \in G$ distincts, aucun monôme de $g$ n'est divisible par $LT(h)$.
            \end{enumerate}
        \end{defi}
        \begin{theo}
            \label{1.5.1}
            Tout idéal $I \subrel{id} k[x_1, \cdots, x_n]$ admet une unique bdG réduite.
        \end{theo}
        \begin{remq}
            La bdG réduite dépend de l'ordre monomial !
        \end{remq}
        On aura besoin d'outils de réduction.
        \begin{lemm}
            Soit $G = \{g_1, \cdots, g_r\}$ une bdG de $I$ idéal.
            \begin{enumerate}
                \item Si $1 \leq i,j \leq r$ distincts sont tq $LT(g_i) \mid LT(g_j)$, alors $G \bs \{g_j\}$ est une bdG de $I$
                \item Si $h_1, \cdots, h_r \in I$ sont tq $\mathrm{mdeg}(h_i) = \mathrm{mdeg}(g_i)$, alors $H = \{h_1, \cdots, h_r\}$ est une bdG de $I$.
            \end{enumerate}
        \end{lemm}
        \begin{proof}
            \begin{enumerate}
                \item Comme $G$ est une bdG, $\bra LT(G) \ket = \bra LT(I) \ket$. Maintenant si $LT(g_i) \mid LT(g_j)$, alors $\bra LT(G \bs \{g_j\}) \ket = \bra LT(G) \ket$ et donc $G \bs \{g_j\}$ est une bdG.
                \item $\bra LT(G) \ket = \bra LT(H) \ket$ vu que $LM(G) = LM(H)$.
            \end{enumerate}
        \end{proof}
        \begin{proof} (\ref{1.5.1})
            Soit $G = \{g_1, \cdots, g_r\}$ une bdG de $I$.
            \begin{enumerate}
                \item Divisons chaque $g_i$ par $LC(g_i)$. On peut donc supposer que $LC(g_i) = 1$.
                \item Chaque fois que $LT(g_i) \mid LT(g_j)$, on peut toujours retirer $g_j$ et toujours avoir une bdG. On peut donc supposer que $\forall i \neq j$, $LT(g_i) \nmid LT(g_j)$.
                \item Enfin, pour chaque $i$, considérons $\bar g_i^{G \bs \{g_i\}} \in I$, et par définition aucun monôme de $\bar g_i^{G \bs \{g_i\}}$ n'est divisible par un des $LT(g_j)$, et $LT\left(\bar g_i^{G \bs \{g_i\}}\right) = LT(g_i)$. Par le $2$ du lemme, alors $\left\{ \bar g_1^{G \bs \{g_1\}}, \cdots, \bar g_r^{G \bs \{g_r\}} \right\}$ est une bdG, qui de plus est réduite.
            \end{enumerate}
            Ceci prouve l'existence d'une bdG réduite pour $I$. Reste à montrer l"unicité : soient $G,G'$ deus bdG réduites de $I$. Soit $g \in G$, il existe $g' \in G'$ tel que $LT(g') \mid LT(g)$. De même, il existe $g'' \in G$ tel que $LT(g'') \mid LT(g')$, et ainsi $LT(g'') \mid LT(g)$, donc $g'' = g$, et donc $LT(g') = LT(g)$. Ainsi on a montré que $LT(G) = LT(G')$. Considérons maintenant $g - g' \in I$, en particulier $\overline{g - g'}^G = 0$. Notons que si $h \in G \bs \{g\}$, alors aucun des termes de $g$ n'est divisible par $LT(h)$. De même pour $g'$, car $LT(G) = LT(G')$. De même aucun monôme de $g - g'$ n'est divisible par $LT(g)$ car $LT(g) = LT(g')$ donc $LT(g - g') < LT(g)$. D'où $\overline{g - g'}^G = g- g' = 0$ donc $g = g'$.
        \end{proof}

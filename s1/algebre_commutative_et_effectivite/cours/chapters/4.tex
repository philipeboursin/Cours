\chapter{Sous-anneaux, polynômes symétriques, théorie des invariants}
    \section{Problème de l'appartenance à un sous-anneau}
        Soient $f_1, \cdots, f_r \in k[x_1, \cdots, x_n]$. On s'est intéressés à la question d'appartenance $f \in \bra f_1, \cdots, f_r \ket$. On veut maintenant savoir si $f \in k[f_1, \cdots, f_r]$, où $k[f_1, \cdots, f_r]$ désigne l'image du morphisme 
        \begin{align*}
            \begin{array}{cccc}
                \phi : & k[y_1, \cdots, y_r] & \to & k[x_1, \cdots, x_n] \\
                & y_i & \mapsto & f_i \\
            \end{array}
        \end{align*}
        \begin{expl}
            Dans $k[x]$, $\bra x^2 \ket$ est très différent de $k[x^2]$.
        \end{expl}
        \begin{remq}
            Un sous-anneau d'un anneau de polynômes n'est pas nécessairement finiment engendré. Par exemple, $k[x, xy, xy^2, \cdots] \subset k[x,y]$ n'est pas finiment engendré.
        \end{remq}
        \begin{prop}
            Soient $f_1, \cdots, f_r \in k[x_1, \cdots, x_n]$, 
            \begin{align*}
                \phi : k[y_1, \cdots, y_r] \to k[x_1, \cdots, x_n]
            \end{align*}
            ,$<$ un ordre monomial sur $k[x_1, \cdots, x_nn y1, \cdots, y_r]$ tel que tout monôme faisant intervenir un $x_i$ est plus grand que tout monôme en les $y_j$ (par exemple $<_{lex}$), $J = \bra f_1 - y_1, \cdots, f_r - y_r \ket$, et $G = \{g_1, \cdots, g_s\}$ une bdg de $J$ pour $<$. Alors $f \in k[f_1, \cdots, f_r] \iff \overline{f}^G \in k[y_1, \cdots, y_r]$. Dans ce cas,
            \begin{align*}
                f = \phi(\overline{f}^G)
            \end{align*}
        \end{prop}
        \begin{proof}
            \item $\Leftarrow$ : Supposons que $\overline{f}^G \in k[y_1, \cdots, y_r]$, et soit 
            \begin{align*}
                f = \sum_{i = 1}^s q_ig_i + \overline{f}^G
            \end{align*} 
            le résultat de la division de $f$ par $G$. Alors on a
            \begin{align*}
                f = \phi(f) = \sum_{i = 1}^s \phi(q_i)\phi(g_i) + \phi(\overline{f}^G)
            \end{align*}
            et $\phi(g_i) = 0$ car $g_i \in \bra f_i - g_i \mid 1 \leq i \leq s \ket$, et ainsi $f = \phi(\overline{f}^G) \in k[f_1, \cdots, f_r]$.
            \item $\Rightarrow$ : Supposons que $f \in k[f_1, \cdots, f_r]$. Supposons que $g_{u+1}, \cdots, g_s$ sont les éléments de $G$ tels que $LT(g_{u+1}), \cdots, LT(g_s) \in k[y_1, \cdots y_r]$. D'après l'hypothèse sur $<$, $g_{u+1}, \cdots, g_s \in k[y_1, \cdots, y_r]$. Soit $p \in k[y_1, \cdots, y_r]$ tel que $\phi(p) = f$. Montrons que $p = \overline{f}^G$. On dispose aussi d'un morphisme
            \begin{align*}
                \begin{array}{cccc}
                    \Phi : & k[x_1, \cdots, x_n, y_1, \cdots, y_r] & \to & k[x_1, \cdots, x_n] \\
                    & x_i & \mapsto & x_i \\
                    & y_i & \mapsto & f_i \\
                \end{array}
            \end{align*}
            On a que $\ker \Phi = J$. On a aussi que $\Phi(p) = f$. Aussi, on peut écrire
            \begin{align*}
                f = \sum_{i = 1}^s q_ig_i + \overline{f}^G
            \end{align*}
            d'où $f = \Phi(f) = \Phi(\overline{f}^G)$, et ainsi $p - \overline{f}^G \in \ker \Phi = J$. Donc $\overline{p - \overline{f}^G}^G = 0$ et ainsi $\overline{p}^G = \overline{\overline{f}^G}^G = \overline{f}^G$. Comme $p \in k[y_1, \cdots, y_r]$, $LT(p) \in k[y_1, \cdots, y_r] \in k[y1, \cdots, y_r]$ ne peut être divisible par $L(g_1), \cdots, LT(g_u)$ qui font intervenir les $x_i$. Comme les $g_{u + 1}, \cdots, g_s \in k[y_1, \cdots, y_s]$, on a bien que $\overline{p}^G \in k[y_1, \cdots, y_r]$ ce qui conclut la preuve.
        \end{proof}
        \begin{nota}
            \begin{align*}
                \begin{array}{cccc}
                    \phi : & k[y_1, \dots, y_r] & \to & k[x_1, \cdots, x_n] \\
                    & y_i & \mapsto & f_i \\
                \end{array}
            \end{align*}
            $F = (f_1, \cdots, f_r)$. Alors on note $\ker \phi =: I_F$, $\ker \Phi = J_f$.
        \end{nota}
        \begin{prop}
            \begin{align*}
                I_F = J_F \cap k[y_1, \cdots, y_r]
            \end{align*}
        \end{prop}
        \begin{proof}
            Clairement, $\ker \phi = \ker \Phi \cap k[y_1, \cdots y_r]$.
        \end{proof}
        \begin{remq}
            \begin{enumerate}
            \item $k[f_1, \cdots, f_r] \simeq k[y_1, \cdots, y_r]/I_F$. 
            \item D'après le \hyperref[implicitisation]{théorème d'implicitisation}, $V(I_F)$ est la variété paramétrée par les $f_i$.
            \end{enumerate}
        \end{remq}

    \section{Polynômes symétriques}
        L'action du groupe symétrique $\mathfrak{S}_n$ sur $\{1, \cdots, n\}$ induit une action sur $k[x_1, \cdots, x_n]$ donnée par $\sigma \in \mathfrak{S}_n$, $f \in k[x_1, \cdots, x_n]$, alors 
        \begin{align*}
            (f \cdot \sigma)(x_1, \cdots, x_n) = f(x_{\sigma(1)}, \cdots, x_{\sigma(n)})
        \end{align*}
        \begin{defi}
            Un poynôme symétrique est un polynôme $f$ tel que $f \cdot \sigma = f$. L'ensemble des polynômes symétriques est $k[x_1, \cdots, x_n]^{\mathfrak{S}_n}$.
        \end{defi}
        \begin{expl}
            Si $n = 3$, alors $x_1 + x_2 + x_3$, $x_1x_2 + x_1x_3 + x_2x_3$, $x_1x_2x_3$ sont des polynômes symétriques.
        \end{expl}
        \begin{prop}
            $k[x_1, \cdots, x_n]^{\mathfrak{S}_n}$ est un sous-anneau de $k[x_1, \cdots, x_n]$.
        \end{prop}
        \begin{defi}
            Pour $i \in \{1, \cdots, n\}$, le $i$ème polynôme symétrique élémentaire est
            \begin{align*}
                \sigma_i = \sum_{1 \leq j_1 < j_2 < \cdots < j_i \leq n} x_{j_1}x_{j_2} \cdots x_{j_i}
            \end{align*}
        \end{defi}
        \begin{expl}
            Si $n = 3$, $\sigma_1 = x_1 + x_2 + x_3$, $\sigma_2 = x_1x_2 + x_1x_3 + x_2x_3$, $\sigma_3 = x_1x_2x_3$.
        \end{expl}
        \begin{theo} (Théorème de structure des polynômes symétriques)
            L'anneau $k[x_1, \cdots, x_n]^{\mathfrak{S}_n}$ est $k[\sigma_1, \cdots, \sigma_n]$. De plus, le morphisme
            \begin{align*}
                \begin{array}{cccc}
                    \phi : & k[y_1, \cdots, y_n] & \to & k[x_1, \cdots, x_n] \\
                    & y_i & \mapsto & \sigma_i \\
                \end{array}
            \end{align*}
            est injectif d'image $k[x_1, \cdots, x_n]^{\mathfrak{S}_n}$.
        \end{theo}
        \begin{proof}
            L'inclusion $k[\sigma_1, \cdots, \sigma_n] \subseteq k[x_1, \cdots, x_n]^{\mathfrak{S}_n}$ est évidente car les $\sigma_i$ sont symétriques. Montrons $\supseteq$ : Soit $f \in k[x_1, \cdots, x_n]^{\mathfrak{S}_n}$ non nul. Soit $< = <_{lex}$ avec $x_1 > x_1 > \cdots > x_n$. Posons $LT(f) = \lambda x_1^{a_1}x_2^{a_2} \cdots x_n^{a_n}$. Puisque $f$ est symétrique, tout monôme $x_{\sigma(1)}^{a_1} \cdots x_{\sigma(n)}^{a_n}$ ($\sigma \in \mathfrak{S}^n$) apparaît dans $f$. Donc $a_1 \geq a_2 \geq \cdots \geq a_n$. Considérons 
            \begin{align*}
                h = \lambda \sigma_1^{a_1 - a_2} \sigma_2^{a_2 - a_3} \cdots \sigma_{n - 1}^{a_{n - 1} - a_n} \sigma_n^{a_n}
            \end{align*}
            Alors $LT(h) = LT(f)$, donc $f = h + (f - h)$ et $f - h$ est symétrique de terme dominant $< LT(f)$ et $h \in k[\sigma_1, \cdots, \sigma_n]$. Par récurrence, $f \in k[\sigma_1, \cdots, \sigma_n]$. Il reste à montrer que $\phi$ est injectif : soit $g \in \ker \phi$, si $y_1^{b_1} \cdots y_n^{b_n}$ est un monôme de $g$, alors $\phi(y_1^{b_1}, \cdots, y_n^{b_n}) = \sigma_1^{b_1} \cdots \sigma_n^{b_n}$ apparaît dans $\phi(g)$. Son terme dominant est $x_1^{b_1 + \cdots + b_n}x_2^{b_2 + \cdots + b_n} \cdots x_n^{b_n}$. Mais la fonction
            \begin{align*}
                \begin{array}{cccc}
                    & \mathbb{N}^n & \to & \mathbb{N}^n \\
                    & (b_1, \cdots, b_n) & \mapsto & (b_1 + \cdots + b_n, b_2 + \cdots + b_n, \cdots, b_n) \\
                \end{array}
            \end{align*}
            est injective, donc tous les termes de $g$ sont envoyés par $\phi$ sur des polynômes de termes dominants différents, leur somme ne peut donc s'annuler que si elle est vide, i.e. $g = 0$.
        \end{proof}
        \begin{coro}
            L'écriture de $f \in k[x_1, \cdots, x_n]^{\mathfrak{S}_n}$ comme polynôme en les $\sigma_i$ est unique.
        \end{coro}
        \begin{expl}
            Soit $n \geq 2$, et considérons
            \begin{align*}
                f = \prod_{1 \leq i < j \leq n} (x_i - x_j)^2
            \end{align*}
            C'est un polynômes en $n$ variables, qui est symétrique. Il existe donc $\Delta \in \mathbb{Z}[y_1, \cdots, y_n]$ tel que $f = \Delta(\sigma_1, \cdots, \sigma_n)$. On définit $\Delta$ comme le discriminant d'ordre $n$. Par exemple, dans $\mathbb{Z}[x_1, \cdots, x_n][T]$ considérons $P = \prod_{i = 1}^n (T - X_i)$. Alors le dixcriminant de $P$ est $\Delta$.
        \end{expl}
        \begin{defi}
            Pour tout $l \geq 1$, on pose
            \begin{align*}
                p_l = \sum_{i = 1}^l x_i^l
            \end{align*}
        \end{defi}
        \begin{remq}
            $p_l$ est symétrique.
        \end{remq}
        \begin{theo}
            Supposons que $k$ est ce caractéristique $0$. Alors
            \begin{align*}
                k[X_1, \cdots, X_n]^{\mathfrak{S}_n} = k[p_1, \cdots, p_n]
            \end{align*} 
        \end{theo}
        \begin{proof}
            On sait que $k[x_1, \cdots, x_n]^{\mathfrak{S}_n} = k[\sigma_1, \cdots, \sigma_n]$. Il suffit donc de montrer que $\sigma_k \in k[p_1, \cdots, p_n]$, $\forall k \in \lcc 1,n \rcc$. Montrons le par récurrence sur $k$. Si $k = 1$, $\sigma_1 = p_1$ ok. Si $k > 1$, on utilise l'identité de Newton :
            \begin{align}
                \label{*}
                p_k - \sigma_1p_{k - 1} + \sigma_2p_{k - 2} - \cdots + k(-1)^k\sigma_k = 0
            \end{align}
            Si cette identité est vraie, alors par récurrence $\sigma_k \in k[p_1, \cdots, p_n]$. Il faut montrer l'identité :
            \begin{align*}
                P_n(T) &= \prod_{i = 1} ^n (T - X_i) \\
                &= T^n - \sigma_1T^{n-1} + \cdots + (-1)^n\sigma_n
            \end{align*}
            Alors
            \begin{align}
                \label{**}
                0 = \sum_{i = 1}^n P_i(X_i) = p_n - \sigma_1p_{n-1} + \cdots + n(-1)^n\sigma_n
            \end{align}
            donc la formule est vraie pour $k = n$. Montrons finalement la formule pour $k < n$ : pour tout $(n - k)$-uplt de variables parmi $X_1, \cdots, X_n$, envoyer ces variables vers $0$, l'identité \ref{*} tiens toujours. On se retrouve alors dans la situation \ref{**} qui vaut $0$. Donc les coefficients devant les monômes ne faisant pas intervenir les $(n - k)$ variables sont tous nuls. Faisons varier le $(n - k)$-uplet, on obtient le résultat.
        \end{proof}

    \section{Théorie des invariants}
        Supposons que $k$ est un corps algébriquement clos, de caractéristique nulle. On note $GL_n(k)$ le groupe des matrices $n \times n$ inversibles. Soit $L$ le sev de $k[x_1, \cdots, x_n]$ des polynômes homogènes de degré $1$. C'est un $k$ ev de dimension $n$ ayant pour base $(x_1, \cdots, x_n)$. Dans cette base, tout élément de $L$ s'écrit comme $\sum \lambda_ix_i$, et alors $GL_n(k)$ agit donc sur $L$ par multiplication à gauche sur $(\lambda_1, \cdots, \lambda_n)$. Ceci induit une action de $GL_n(k)$ sur tout $k[x_1, \cdots, x_n]$.
        \begin{nota}
            $(f.A)(x_1, \cdots, x_n) = f(A \cdot (x_1, \cdots, x_n))$
        \end{nota}
        Remarquons que si $G$ est un sous-groupe fini de $GL_n(k)$, alors $G$ agit de la même manière sur $k[x_1, \cdots, x_n]$. Ainsi soit $G$ un tel sous-groupe, est-ce-que $k[X_1, \cdots, X_n]^G$ est-il finiment engendré.
        \begin{expl}
            $n = 3$, prenons
            \begin{align*}
                G = \bra
                \begin{bmatrix}
                    0 & 1 & 0 \\
                    0 & 0 & 1 \\
                    1 & 0 & 0 \\
                \end{bmatrix} \ket \simeq \znz{3}
            \end{align*}
            agit par permutation cyclique de $x_1, x_2, x_3$. Par exemple, $f = (x_1 - x_2)(x_2 - x_3)(x_3 - x_1)$ est un polynômes $G$ invariant.
        \end{expl}
        \begin{expl}
            Soit 
            \begin{align*}
                G = \bra
                \begin{bmatrix}
                    1 & 0 & 0 \\
                    0 & \zeta & 0 \\
                    0 & 0 & \zeta^2 \\
                \end{bmatrix} \ket
            \end{align*}
            avec $\zeta$ une racine cubique primitive de l'unité. Par exemple $x_1,x_2^2,x_3^3$ sont $G$ invariants. $x_1x_2x_3$ est $G$ invariant. $(x_1^3 - 1)(x_2^3 - 1)(x_3^3 - 1)$ est $G$-invariant.
        \end{expl}
        \begin{defi}
            L'opérateur de Reynolds de $G$ est 
            \begin{align*}
                \begin{array}{cccc}
                    R_G : & k[x_1, \cdots, x_n] & \to & k[x_1, \cdots, x_n]^G \\
                    & f & \mapsto & \frac 1{|G|} \sum_{A \in G} f \cdot A \\
                \end{array}
            \end{align*}
        \end{defi}
        \begin{prop}
            \ref{prop431}
            \begin{enumerate}
                \item $R_G(f)$ est $G$-invariant, pour tout $f \in k[x_1, \cdots, x_n]$.
                \item $R_G$ est $k$-linéaire (mais n'est pas un morphisme d'anneaux).
                \item Si $f \in k[x_1, \cdots, x_n]^G$, alors $R_G(f) = f$.
            \end{enumerate}
        \end{prop}
        \begin{proof}
            Ok
        \end{proof}
        \begin{expl}
            Prenons $G = \mathfrak{S}_n$ agissant par permutation. Alors
            \begin{align*}
                R_G(x_1) = \frac 1n ((n - 1)!x_1 + (n - 1)!x_2 + \cdots + (n - 1)!x_n) = \frac 1n \sigma_1
            \end{align*}
            De même, $R_G(x_2) = \frac 1n \sigma_1$.
        \end{expl}
        \begin{theo} [Emmy Noether]
            Supposons que $k = \bar k$, $car(k) = 0$, $G$ est un sous-groupe fini de $GL_n(k)$. Alors $k[x_1, \cdots, x_n]^G$ est engendré par les $R_G(x^\beta)$ avec $\deg x^\beta \leq |G|$ est finiment engendré.
        \end{theo}
        \begin{proof}
            \item L'inclusion $k[R_G(x^\beta) \mid |\beta| \leq |G|] \subseteq k[x_1, \cdots, x_n]^G$ est clair d'après le point 1 de la proposition \label{431}.
            \item Soit $f \in k[x_1, \cdots, x_n]^G$, notons le $f = \sum_{\alpha \in \mathbb{N}^n} \lambda_\alpha x^\alpha$. Alors $f = R_G(f) = \sum_{\alpha \in \mathbb{N}^n} \lambda_\alpha R_G(x^\alpha)$. Ceci montre que $k[x_1, \cdots, x_n]^G = k[R_G(x^\alpha) \mid \alpha \in \mathbb{N}^n]$. Il suffit donc de montrer que pour tout $\alpha \in \mathbb{N}^n$, alors $R_G(x^\alpha) \in k[R_G(x^\beta) \mid |\beta| \leq |G|]$.
            \begin{nota}
                Soit $A \in G \subrel{sg fini} GL_n(k)$, notons
                \begin{align*}
                    A =
                    \begin{bmatrix}
                        A_1 \\
                        A_2 \\
                        \vdots \\
                        A_n \\
                    \end{bmatrix}
                \end{align*}
                Alors on définit
                \begin{align*}
                    (A.x)^\alpha = (A_1x)^{\alpha_1} (A_2x)^{\alpha_2} \cdots (A_nx)^{\alpha_n}
                \end{align*}
                Ainsi $x^\alpha = (Ax)^\alpha$.
            \end{nota}
            On a donc
            \begin{align*}
                R_G(x^\alpha) = \frac 1{|G|} \sum_{A \in G} (Ax)^\alpha
            \end{align*}
            Soient $u_1, \cdots, u_n$ est variables. Pour $A \in G$, soit 
            \begin{align*}
                z_A = (u_1A_1x + \cdots + u_nA_nx) \in k[x_1, \cdots, x_n, u_1, \cdots, u_n]
            \end{align*}
            Pour tout $k > 0$, écrivons
            \begin{align*}
                z_1^k &= (u_1A_1x + \cdots + u_nA_nx)^k \\
                &= \sum_{|\alpha| \leq k} a_\alpha (Ax)^\alpha u^\alpha
            \end{align*}
            où $a_\alpha \in k$ ne dépend pas de $A$. Soit
            \begin{align*}
                S_k &= \sum_{A \in G} z_A^k \\
                &= \sum_{A \in G} \sum_{|\alpha| \leq k} a_\alpha (Ax)^\alpha u^\alpha \\
                &= \sum_{|\alpha| \leq k} a_\alpha \left( \sum_{A \in G} (Ax)^\alpha \right) u^\alpha \\
                &= \sum_{|\alpha| \leq k} a_\alpha |G|R_G(x^\alpha) u^\alpha \\
            \end{align*}
            Or $S_k$ est un polynôme symétrique en les $z_A$. Par le théorème précédent, tout polynôme symétrique en les $z_A$ est un polynôme en les $p_1 = S_1, \cdots, p_{|G|} = S_{|G|}$. En particulier, $S_k$ est un polynôme en les $S_1, \cdots, S_{|G|}$, disons $S_k = F(S_1, \cdots, S_{|G|})$. Donc 
            \begin{align*}
                &\sum_{|\alpha| \leq k} a_\alpha |G|R_G(x^\alpha) u^\alpha = S_k = F(S_1, \cdots, S_{|G|}) \\
                &= F\left(\sum_{|\beta| \leq 1} a^1_\beta |G|R_G(X)^\beta u^\beta, \cdots, \sum_{|\beta| \leq |G|} a^{|G|}_\beta |G|R_G(X)^\beta u^\beta \right) \in k[u_1, \cdots, u_n, R_G(x^\beta) \mid |\beta| \leq |G|]
            \end{align*}
            En comparant les coeffs devant $u^\alpha$, on obtient que $R_G(x^\alpha) \in k[R_G(x^\beta) \mid |\beta| \leq |G|]$.
        \end{proof}
        
        \subsection{Interprétation géométrique}
            On suppose toujours que $k = \bar k$, $car k = 0$. $G$ agit sur $k[X_1, \cdots, X_n]$, et $G$ agit sur $\mathbb{A}^n$. Ainsi
            \begin{align*}
                k[y_1, \cdots, y_r]/I_F \simeq k[X_1, \cdots, X_n]^G \injectivearrow k[X_1, \cdots, X_n]
            \end{align*}
            On applique les foncteur variété à ce diagramme, on obtient un morphisme $\pi : \mathbb{A}^n_x \to V(I_F) \subseteq \mathbb{A}^r_y$.
            \begin{theo}
                \begin{enumerate}
                    \item $\pi$ est surjective,
                    \item Si $p,q \in \mathbb{A}^n$, alors $\pi(p) = \pi(q) \iff Gp = Gq$ ($Gp,Gq$ sont les $G$-orbites de $p$ et $q$)
                \end{enumerate}
            \end{theo}
            \begin{remq}
                $V(I_F)$ est en bijection est en bijections avec $\mathbb{A}^n/G$, et $k[V(I_F)] \simeq k[X_1, \cdots, X_n]^G$.
            \end{remq}
            \begin{proof}
                \begin{enumerate}
                    \item On voit le morphisme de $\mathbb{A}^n_x$ dans $\mathbb{A}^r_y$ comme définissant une variété paramétrée. Par le théorème d'implicitisation, $V(I_F) = \overline{\pi(\mathbb{A}^n)}$. On veut montrer que $V(I_F) \subseteq \pi(\mathbb{A}^n)$. Soit $b = (b_1, \cdots, b_r) \in V(I_F)$. On pose 
                    \begin{align*}
                        J = \bra y_1 - f_1, \cdots, y_r - f_r \ket \subrel{id} k[x_1, \cdots, x_n, y_1, \cdots, y_r]
                    \end{align*}
                    \begin{figure}[H]
                        \centering
                        \begin{tikzcd}
                            {k[x_1, \cdots, x_n, y_1, \cdots, y_n]} \arrow[rd, "\Phi"] &                       \\
                            {k[y_1, \cdots, y_r]} \arrow[r]                            & {k[x_1, \cdots, x_n]}
                            \end{tikzcd}
                    \end{figure}
                    $J = \ker \Phi$. On cherche $(a_1, \cdots, a_n) \in \mathbb{A}^n_x$ tel que $\pi(a) = b$. De façon équivalente, $(a_1, \cdots, a_n, b_1, \cdots, b_r) \in V(J)$. On applique le théorème d'extension : rappelons que si $(a_{i+1}, \cdots, a_n, b_1, \cdots, b_r)$ est une solution partielle de $J \cap k[x_{i+1}, \cdots, x_n, y_1, \cdots, y_r]$ et s'il existe $h_i \in J \cap k[x_i, \cdots, x_n, y_1, \cdots, y_r]$ avec 
                    \begin{align*}
                        h_i = g_i(x_{i+1}, \dots, x_n, y)x_i^{N_i} + \text{ termes en }\deg_{X_i} < N_i
                    \end{align*} 
                    et $g_i(a_{i + 1}, \cdots, a_n, b) \neq 0$ alors la solution s'étend en une solution $(a_i, a_{i+1}, \cdots, a_n, b)$. Trouvons un tel $h_i$ pour chaque $i$ : pour chaque $h \in k[x_1, \cdots, x_n]$, considérons
                    \begin{align*}
                        \prod_{A \in G} (u - hA) = \sum_{j = 0}^{|G|} g_j(x)u^j \in k[u,x]
                    \end{align*}
                    Comme ce polynôme est $G$ invariant, on a que $g_j(x) \in k[x]^G$, $\forall j$. On sait que l'on peut écrire $k[x]^G = k[f_1, \cdots, f_r]$. Ainsi $\exists p_0, \cdots, p_{|G| - 1} \in k[y_1, \cdots, y_r]$ tels que $g_j(x) = p_j(f_1, \cdots, f_r)$ (remarquons que l'on traite pas le cas $|G|$ puisque $g_{|G|} = 1$). En évaluant $u$ en $h$, on trouve $0$ :
                    \begin{align*}
                        0 &= \sum_{j = 0}^{|G|} p_j(f_1, \cdots, f_r) h^j \\
                        &= \sum_{j = 0}^{|G| - 1} p_j(f_1, \cdots, f_r) h^j + h^{|G|} \\
                    \end{align*}
                    On applique à $h = x_i$ :
                    \begin{align*}
                        h_i := \sum_{j = 0}^{|G| - 1} p_j(y_1, \cdots, y_r) x_i^j + x_i^{|G|} \in J \\
                    \end{align*}
                    puisque $h_i(f_1, \cdots, f_r) = 0$ d'après l'équation précédente. Avec ce choix de $h_i$ pour tout $i$, on peut appliquer le théorème d'extension récursivement. On a ainsi montré que $\pi$ est surjective.
                    \item $\pi : \mathbb{A}^n \to V(I_F)$, puisque les $f_i$ sont $G$ invariants,
                    \begin{align*}
                        \pi(Ax) = (f_1(Ax), \cdots, f_r(Ax)) = (f_1(x), \cdots, f_r(x)) = \pi(x)
                    \end{align*}
                    Il reste à montrer que si $Ga \neq Gb$, alors $\pi(a) \neq \pi(b)$. Il suffit de trouver $g \in k[x_1, \cdots, x_n]^G$ tel que $g(a) \neq g(b)$. Posons $E = (Gb \cup Ga) \bs \{a\}$. C'est un ensemble fini, donc c'est une variété algébrique (au sens d'ensemble algébrique dans l'autre cours). Soit $L \subrel{id} k[X]$ tel que $E = V(L)$, comme $a \notin E$, il existe $f \in L$ tel que $f(a) \neq 0$. Prenons maintenant $g = R_G(f) \in k[x_1, \cdots, x_n]^G$. Alors $g(b) = 0$, et $g(a) = \neq 0$. 
                \end{enumerate}
            \end{proof}
            \begin{remq}
                $\pi : \mathbb{A}^n \to V(I_f)$ n'est plus forcément surjective si $k \neq \bar k$. Par exemple prenons $\mathbb{R}[x_1, x_2]^{\mathfrak{S}_2} = \mathbb{R}[\sigma_1, \sigma_2]$. Alors $I_F = \ker(y_i \in k[y_1, y_2] \mapsto \sigma_k k[x_1, x_2]) = 0$. Donc $V(I_F) = \mathbb{A}^2_y$ mais
                \begin{align*}
                    \begin{array}{cccc}
                        \pi : & \mathbb{A}^2 & \to & \mathbb{A}^2_y \\
                        & (x_1, x_2) & \mapsto & (x_1 + x_2, x_1x_2) \\
                    \end{array}
                \end{align*}
                n'est pas surjective sur $\mathbb{R}$ : en effet, si $(y_1, y_2) = \pi(x_1, x_2) = (x_1 + x_2, x_1x_2)$ et ainsi $(X - x_1)(X - x_2) = X^2 - \sigma_1X + \sigma_2 = X^2 - y_1X + y_2$ donc $X^2 - y_1X + y2$ a des racines réelles, i.e. $y_1^2 - 4y_2 \geq 0$.
            \end{remq}
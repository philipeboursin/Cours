\chapter*{Aide mémoire programmation}
    \section*{Compilation}
        \mintinline{c}{>gcc [options] fichier.c [arguments] [librairies]}. Les options sont :
        \begin{itemize}
            \item \mintinline{c}{-E} : préprocesseur uniquement
            \item \mintinline{c}{-o nom} : nomme \mintinline{c}{nom.out} le binaire produit
            \item \mintinline{c}{-O, -O1, -O2, -O3} : choix d'efficacité de la compilation, de la pire a la meilleure.
            \item \mintinline{c}{-S} : donne le fichier assembleur
            \item \mintinline{c}{-W} : donne plus de messages d'avertissement
            \item \mintinline{c}{-Wall} : donns tous les messages d'avertissement
        \end{itemize}
        \mintinline{c}{>time ./a.out} pour mesurer le temps d'exécution du programme

    \section*{Programme et fonctions}
        structure d'un programme :
        \begin{minted}[]{c}
            [directives préprocesseur]
            [déclarations variables externes]
            [prototype des fonctions externes]
            [déclaration des fonctions secondaires]

            int main()
            {
                instructions
            }
        \end{minted}
        \mintinline{c}{#include<lib.h>} (ex \mintinline{c}{stdio.h}) pour inclure une librarie avec le préprocesseur \\
        template d'une fonction \mintinline{c}{type fonction(arguments);} \\
        structure d'une fonction
        \begin{minted}{c}
            type fonction(arguments)
            {
                instructions;
                return var;
            }
        \end{minted}
        On l'appelle en écrivant \mintinline{c}{fonction(arguments)}

    \section*{Types de variables}
        \begin{itemize}
            \item Globales : sont stockées sous le tas et est déclarée en dehors du main (donc utilisable par tout le programme)
            \item Temporaires : variable déclarée à l'intérieur d'une fonction, est détruite en sortie de fonction
            \item Statiques : \mintinline{c}{static type var} variable associée à une fonction qui change lorsque l'on appelle cette fonction mais n'est pas détruite en sortie de fonction et conserve sa valeur.
        \end{itemize}
        \begin{minted}{c}
#include<stdio.h>
#include<string.h>


//------
typedef struct Polynome {
unsigned int k1;
unsigned int k2;
unsigned int k3;
} polynome;


//------
// Affiche un ui sous forme binaire
void print_bits(unsigned int a);
// Renvoie un polynome primitif de F2^n : F2 sous forme polynome
polynome auxi(unsigned int n);
// Tests de la question 1
void q1();
// Affiche un polynome de F2[T]
void poly_print(unsigned int P);
// Tests de la question 2
void q2();


//------ Fonctions auxiliaires
void print_bits(unsigned int a)
{
    for (int i = 0; i < 32; i++)
    {
        printf("%d", (a >> i) & 0x1);
    }
}


//------ Q1
polynome auxi(unsigned int n)
{
    switch (n)
    {
        case 5:
        case 11:
        case 21:
        case 29:
            return (polynome){2, 0, 0};
        case 10:
        case 17:
        case 20:
        case 25:
        case 28:
        case 31:
            return (polynome){3, 0, 0};
        case 9:
            return (polynome){4, 0, 0};
        case 23:
            return (polynome){5, 0, 0};
        case 18:
            return (polynome){7, 0, 0};
        case 8:
        case 19:
            return (polynome){6, 5, 1};
        case 12:
            return (polynome){7, 4, 3};
        case 13:
        case 24:
            return (polynome){4, 3, 1};
        case 14:
            return (polynome){12, 11, 1};
        case 16:
            return (polynome){5, 3, 2};
        case 26:
        case 27:
            return (polynome){8, 7, 1};
        case 30:
            return (polynome){16, 15, 1};
        default:
            return (polynome){1, 0, 0};
    }
}

unsigned int poly_prim(unsigned int n)
{
    polynome P = auxi(n);
    return 1 ^ (1 << P.k1) ^ (1 << P.k2) ^ (1 << P.k3) ^ (1 << n);
}

void q1()
{
    for (int n = 2; n < 32; n++)
    {
        print_bits(poly_prim(n));
        printf("\n");
    }
}


//------ Q2
void poly_print(unsigned int P)
{
    if (P == 0) printf("0");
    else
    {
        int b = 0;
        int i = 1;
        if (P & 1 == 1)
        {
            printf("%d", 1);
            b = 1;
        }
        P = P >> 1;
        while (P != 0)
        {
            if (P & 1 == 1)
            {
                if (b == 0)
                {
                    printf("X^%d", i);
                    b = 1;
                }
                else printf(" + X^%d", i);
            }
            i++;
            P = P >> 1;
        }
    }
}

void q2()
{
    for (int n = 2; n < 32; n++)
    {
        poly_print(poly_prim(n));
        printf("\n");
    }
}


//------ Q3



//------
int main()
{
    // q1();
    q2();
}
        \end{minted}





% //// Boucles :
% while (condition)
% {
% 	code
% }

% for (variable, condition de sortie, opération à faire à chaque tour de boucle sur la variable)
% {
% 	code
% }
% expl :
% for (int i=0; i<10, i++)
% {
% 	code
% }

% //// Formats :
% \%d -> int
% \%c -> char

% Tableaux :
% type nom_tab[N];
% type nom_tab[N] = {a_1, ..., a_N};
% int i = valeur;
% nom_tab[i] // Renvoie la valeur du tableau en i
% // Les cases d'un tableau sont indicée en démarrant à 0
% char tab[7] = "bonjour" // Pour les string
% // Dans le cas où le tableau est spécifié lors de sa définition, pas besoin de donner sa taille

% //// Structures
% struct framework
% {
% 	type_1 variable_1;
% 	...
% 	type_n variable_n;
% }
% struct framework objet; // pour définir un objet du struct
% // peut écrire objet juste derriere l'acollade du struct pour déclarer un object.
% enum booleen {vrai, faux};
% // pour les types énumérés
% typedef enum booleen booleen // pour créer un type booléen
% booleen a;

% //// Pointeurs
% int *p; // pour définir un pointeur d'entier
% int a; // &a renvoie l'adresse de a.
% p = &a; // Donne a p la valeur &a
% *p += 1; // *p représente le contenu de la case d'adresse p. Cette opération rajoute 1 au contenu de la case adréssée par p, donc rajoute 1 a 'a'

% //// Allocation de mémoire
% malloc(nombre d'octets) // Alloue de la mémoire. On peut définir un pointeur et lui donner cette valeur pour spécifier l'adresse du début de cette zone mémoire allouée
% free(pt) // Libère la mémoire associée à un pointeur qui pointe sur une zone de mémoire. On libère toujours le début de la zone (free(pt + 1) n'est pas valide par exemple)

\chapter*{Aide mémoire programmation}
    \section*{Compilation}
        \mintinline{c}{>gcc [options] fichier.c [arguments] [librairies]}. Les options sont :
        \begin{itemize}
            \item \mintinline{c}{-E} : préprocesseur uniquement
            \item \mintinline{c}{-o nom} : nomme \mintinline{c}{nom.out} le binaire produit
            \item \mintinline{c}{-O, -O1, -O2, -O3} : choix d'efficacité de la compilation, de la pire a la meilleure.
            \item \mintinline{c}{-S} : donne le fichier assembleur
            \item \mintinline{c}{-W} : donne plus de messages d'avertissement
            \item \mintinline{c}{-Wall} : donns tous les messages d'avertissement
        \end{itemize}
        \mintinline{c}{>time ./a.out} pour mesurer le temps d'exécution du programme

    \section*{Programme et fonctions}
        structure d'un programme :
        \begin{minted}[]{c}
            [directives préprocesseur]
            [déclarations variables externes]
            [prototype des fonctions externes]
            [déclaration des fonctions secondaires]

            int main()
            {
                instructions
            }
        \end{minted}
        \mintinline{c}{#include<lib.h>} (ex \mintinline{c}{stdio.h}) pour inclure une librarie avec le préprocesseur \\
        template d'une fonction \mintinline{c}{type fonction(arguments);} \\
        structure d'une fonction
        \begin{minted}{c}
            type fonction(arguments)
            {
                instructions;
                return var;
            }
        \end{minted}
        On l'appelle en écrivant \mintinline{c}{fonction(arguments)}

    \section*{Types de variables}
        \begin{itemize}
            \item Globales : sont stockées sous le tas et est déclarée en dehors du main (donc utilisable par tout le programme)
            \item Temporaires : variable déclarée à l'intérieur d'une fonction, est détruite en sortie de fonction
            \item Statiques : \mintinline{c}{static type var} variable associée à une fonction qui change lorsque l'on appelle cette fonction mais n'est pas détruite en sortie de fonction et conserve sa valeur.
        \end{itemize}
% //// Boucles :
% while (condition)
% {
% 	code
% }

% for (variable, condition de sortie, opération à faire à chaque tour de boucle sur la variable)
% {
% 	code
% }
% expl :
% for (int i=0; i<10, i++)
% {
% 	code
% }

% //// Formats :
% \%d -> int
% \%c -> char

% Tableaux :
% type nom_tab[N];
% type nom_tab[N] = {a_1, ..., a_N};
% int i = valeur;
% nom_tab[i] // Renvoie la valeur du tableau en i
% // Les cases d'un tableau sont indicée en démarrant à 0
% char tab[7] = "bonjour" // Pour les string
% // Dans le cas où le tableau est spécifié lors de sa définition, pas besoin de donner sa taille

% //// Structures
% struct framework
% {
% 	type_1 variable_1;
% 	...
% 	type_n variable_n;
% }
% struct framework objet; // pour définir un objet du struct
% // peut écrire objet juste derriere l'acollade du struct pour déclarer un object.
% enum booleen {vrai, faux};
% // pour les types énumérés
% typedef enum booleen booleen // pour créer un type booléen
% booleen a;

% //// Pointeurs
% int *p; // pour définir un pointeur d'entier
% int a; // &a renvoie l'adresse de a.
% p = &a; // Donne a p la valeur &a
% *p += 1; // *p représente le contenu de la case d'adresse p. Cette opération rajoute 1 au contenu de la case adréssée par p, donc rajoute 1 a 'a'

% //// Allocation de mémoire
% malloc(nombre d'octets) // Alloue de la mémoire. On peut définir un pointeur et lui donner cette valeur pour spécifier l'adresse du début de cette zone mémoire allouée
% free(pt) // Libère la mémoire associée à un pointeur qui pointe sur une zone de mémoire. On libère toujours le début de la zone (free(pt + 1) n'est pas valide par exemple)

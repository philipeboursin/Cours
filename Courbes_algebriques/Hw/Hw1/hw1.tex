\documentclass[11pt]{article}

\usepackage{import}
\import{D:/Bureau/Documents/Maths/Latex/Packages/}{article.tex}

% Couleur de correction
% \newcommand{\cor}[1]{{#1}}
\newcommand{\cor}[1]{{\color{red} #1}}

\begin{document}

%Page de garde
\title{Exercices}
\date{\today}
\author{Alexandre Guillemot}
\maketitle

\section*{Exercice 10, feuille 1, question 3}
    \begin{question}{3)}
        Tout d'abord, si le corps est fini, alors $V(Y^2 - X^3)$ contiens $(0,0)$ et $(1,1)$, donc n'est pas irréductible. Supposons maintenant que $|k| = \infty$, montrons que
        \begin{align*}
            V(Y^2 - X^3) = \{(t^2, t^3) \in k^2 \mid t \in k\} =: V
        \end{align*}
        Si $(x,y) \in V$, alors $\exists t \in k \mid (x,y) = (t^2, t^3)$. Et alors $y^2 - x^3 = t^6 - t^6 = 0$, donc $(x,y) \in V(Y^2 - X^3)$. Réciproquement, si $(x,y) \in V(Y^2 - X^3)$, alors $y^2 = x^3$ dans $k$. Et alors si $x = 0$, alors $y = 0$ et $(0,0) \in V$. Sinon,
        \begin{align*}
            &\left(\frac{y}{x}\right)^2 = x \\
            &\left(\frac{y}{x}\right)^3 = y \\
        \end{align*}
        donc en posant $t = y/x$, $(x,y) = (t^2, t^3) \in V$. \\
        Ensuite, montrons que $I(V(Y^2 - X^3)) = (Y^2 - X^3)$ : remarquons dans un premier temps que pour tout $P \in k[T]$, $P = 0 \iff P(t) = 0$, $\forall t \in k$ du fait que $|k| = \infty$. Ainsi prouver $(Y^2 - X^3) = I(V(Y^2 - X^3))$ reviens à prouver que le noyau de
        \begin{align*}
            \begin{array}{cccc}
                \varphi : & k[X,Y] & \to & k[T] \\
                & P & \mapsto & P(T^2,T^3) \\
            \end{array}
        \end{align*}
        vaut $(Y^2 - X^3)$. En effet, $P \in I(V(Y^2 - X^3)) \iff P(t^3, t^3) = 0$, $\forall t \in k$ $\iff P(T^2, T^3) = 0$ au vu de la remarque faite précédemment, donc $\ker \varphi = I(V(Y^2 - X^3))$. Il est clair que $(Y^2 - X^3) \subseteq \ker \varphi$. Réciproquement, soit $P \in \ker \varphi$, réalisons la division euclidienne de $P$ par $Y^2 - X^3$ dans $k[X][Y]$ :
        \begin{align*}
            P(X,Y) = Q(X,Y)(Y^2 - X^3) + R(X,Y)
        \end{align*}
        où $\deg_Y R \leq 1$. Ecrivons alors $R(X,Y) = a(X)Y + b(X)$, montrons que $a$ et $b$ sont nuls. Développons alors $a$ et $b$ : si on écrit 
        \begin{align*}
            &a(X) = \sum_{i \geq 0} a_i X^i \\
            &b(X) = \sum_{i \geq 0} b_i X^i \\
        \end{align*}
        on a 
        \begin{align*}
            R(T^2, T^3) &= a(T^2)T^3 + b(T^2) \\
            &= \sum_{i \geq 0} a_i T^{2i + 3} + b_i T^{2i} \\
            &= \sum_{i \geq 0} a_i T^{2i + 3} + b_i T^{2i} \\
            &= \sum_{j \geq 0} c_j T^j \\
        \end{align*}
        où
        \begin{align*}
            c_j =
            \begin{cases}
                a_i &\text{ si } j = 2i + 3 \text{ pour un certain } i \in \mathbb{N} \\
                b_i &\text{ si } j = 2i \text{ pour un certain } i \in \mathbb{N} \\
                0 & \text{ sinon}
            \end{cases}
        \end{align*}
        (Les coefficients de $a(T^2)T^3$ n'intéragissent pas avec ceux de $b(T^2)$, car devant des monômes de degré impair alors que ceux de $b(T^2)$ n'aparaissent que devant des monômes de degré pair). Ainsi comme $P \in \ker \varphi$, $0 = \varphi(R) = R(T^2, T^3)$ et donc $a_i, b_i = 0$ pour tout $i \geq 0$. Finalement, $a,b = 0$ et donc $R = 0$, d'où $P \in (Y^2 - X^3)$. Ainsi on a bien égalité $I(V(Y^2 - X^3)) = (Y^2 - X^3)$, et $V(Y^2 - X^3)$ est irréductible puisque $K[X,Y]/(Y^2 , X^3)$ s'injecte dans $k[T]$ qui est lui-même intègre.
    \end{question}

\section*{Exercice 15, feuille 1}
    Soient $V_1 \subseteq \mathbb{A}_k^n$, $V_2 \subseteq \mathbb{A}_k^m$ des ensembles algébriques affines. On note
    \begin{align*}
        &k[x_1, \cdots, x_n] =: A \\
        &k[y_1, \cdots, y_m] =: B \\
        &k[x_1, \cdots, x_n, y_1, \cdots, y_m] =: C \\
    \end{align*}
    Alors il existe $I \subrel{id} A$ et $J \subrel{id} B$ tels que $V_1 = V(I)$ et $V_2 = V(J)$. Considérons le morphisme
    \begin{align}
        \varphi := p_I \otimes p_J : A \otimes_k B \to A/I \otimes_k B/J
    \end{align}
    Où $p_I : A \to A/I$, $p_J : B \to B/J$ sont les projections canoniques des quotients respectifs. On sait que le morphisme $A \otimes_k B \to C$ induit par les morphismes canoniques $i_1 : A \to C$, $i_B : B \to C$ (issus de la propriété universelle des anneaux de polynômes) est un isomorphisme ($\sum_{finie} P_i \otimes Q_i$ est envoyé sur $\sum_{finie} i_A(P_j)i_B(Q_j)$). Une dernière remarque est qu'au vu de la naturalité de $\Hom_{\mathbf{Sets}}(S,k) \simeq \Hom_{k-\mathbf{CAlg}}(k[S],k)$, nous avons la commutativité du diagramme
    \begin{figure}[H]
        \centering
        \begin{tikzcd}
            A \arrow[r, "i_1"] \arrow[rd, "\mathrm{ev}_a"'] & C \arrow[d, "{\mathrm{ev}_{(a,b)}}"] & B \arrow[l, "i_2"'] \arrow[ld, "\mathrm{ev}_b"] \\
                                                            & k                                    &                                                
            \end{tikzcd}
    \end{figure} \noindent
    Pour terminer l'exercice, montrons que $V(\ker \varphi) = V_1 \times V_2$ (où $\ker \varphi$ est vu comme un idéal de $C$ par l'isomorphisme naturel donné précédemment). Prenons $(a,b) \in V(\ker \varphi)$, puis soient $P \in I$, $Q \in J$. Alors $P \otimes 1, 1 \otimes Q \in \ker \varphi$ et donc
    \begin{align*}
        0 = \mathrm{ev}_{(a,b)}(i_1(P)i_2(1)) =  P(a)
    \end{align*}
    et de même, $Q(b) = 0$, et ainsi $(a,b) \in V_1 \times V_2$. Réciproquement, soit $(a,b) \in V_1 \times V_2$. Alors tout élément de $\ker \varphi$ s'écrit comme une somme finie $\sum_{\mathrm{\mathrm{finie}}} P_j \otimes Q_j$. Mais
    \begin{align*}
        \mathrm{ev}_{(a,b)}\left(\sum_{\mathrm{finie}} i_A(P_j)i_B(Q_j)\right) = \sum_{\mathrm{finie}} P_j(a)Q_j(b) = 0
    \end{align*}
    et ainsi $(a,b) \in V(\ker \varphi)$.

\section*{Exercice 1, feuille 2}
    \begin{question}{1)}
        Montrons que $D(f) \cap D(g) = D(fg)$ : en passant au complémentaire, il faut montrer que $V(fg) = V(f) \cup V(g)$, ce que l'on sait vrai d'après le cours.
    \end{question}
    \begin{question}{2)}
        Soit $U = \mathbb{A}^n \bs V(I)$ un ouvert de $\mathbb{A}^n$, avec $I \subrel{id} k[x_1, \cdots, x_n]$. Alors
        \begin{align*}
            V(I) &= V\left( \bigcup_{f \in I} (f) \right) \\
            &= \bigcap_{f \in I} V((f))
        \end{align*}
        donc finalement
        \begin{align*}
            U = \bigcup_{f \in I} D(f)
        \end{align*}
        en passant au complémentaire.
    \end{question}
    \begin{question}{3)}
        $D(f) = \emptyset \iff V((f)) = \mathbb{A}^n_k \iff \forall x \in k^n,\, f(x) = 0 \iff f = 0$, la dernière équivalence provenant du fait que $|k| = \infty$ (résultat que l'on a prouvé par récurrence en td).
    \end{question}
    \begin{question}{4)}
        On utilise les questions précédentes : comme les ensembles $D(f)$ forment une base pour la topologie de $\mathbb{A}^n$ (question 2), et que $U,V \neq \emptyset$, pour tout $x \in U$, $y \in V$, il existe $f,g \in k[x_1, \cdots, x_n]$ tels que $x \in D(f) \subseteq U$ et $y \in D(g) \subseteq V$. Maintenant $D(f) \cap D(g) = D(fg)$ (question 1) mais alors si $D(f) \cap D(g) = \emptyset$, alors $fg = 0$ (question 3) donc $f = 0$ ou $g = 0$ et donc $D(f) = \emptyset$ ou $D(g) = \emptyset$, absurde. Ainsi, $D(f) \cap D(g)$ est non vide, et donc $U \cap V \neq \emptyset$.
    \end{question}

\end{document}

\documentclass[11pt]{article}

\usepackage{import}
\import{D:/Bureau/Documents/Maths/Latex/Packages/}{article.tex}

% Couleur de correction
% \newcommand{\cor}[1]{{#1}}
\newcommand{\cor}[1]{{\color{red} #1}}

\begin{document}

%Page de garde
\title{Exercices}
\date{\today}
\author{Alexandre Guillemot}
\maketitle

\section*{Exercice 10, feuille 2}
Avant de résoudre les questions, il faut prouver que $C$ est bien une variété algébrique, i.e. $Y^2 - X(X-1)(X-a)$ est irréductible (car dans ce cas $I(C) = \sqrt{(Y^2 - X(X-1)(X-a))}$ sera un idéal premier). Pour cela, utilisons le résultat suivant :
\begin{lemm}
    Soit $A$ un anneau commutatif, et $a \in A$. Alors le polynôme $X^2 - a \in A[X]$ est irréductible si et seulement si $a$ est un carré dans $A$.
\end{lemm}
\begin{proof}
    \item Si $a$ est un carré, disons $a = b^2$ avec $b \in A$, alors $X^2 - a = (X - b)(X + b)$ et donc $X^2 - a$ est réductible.
    \item Supposons maintenant que $X^2 - a$ est réductible, écrivons $X^2 - a = PQ$ avec $P,Q \in A[X]$ qui ne sont pas des unités dans $A$. Alors 3 cas se présentent :
    \begin{enumerate}
        \item $\deg P = 0$, $\deg Q = 2$, écrivons $P = \lambda_1X^2 + \lambda_2X + \lambda_3$, et $Q = \lambda$. Alors $\lambda \lambda_1 = 1$ donc $\lambda$ est inversible dans $A$, absurde.
        \item $\deg P = 2$, $\deg Q = 0$, c'est le même cas que dans le point précédent.
        \item $\deg P = \deg Q = 1$ : écrivons $P = \lambda_1X + \lambda_2$, $Q = \lambda_3X + \lambda_4$. Comme $\lambda_1\lambda_2 = 1$, on peut se ramener au cas où $\lambda_1 = \lambda_2 = 1$. Mais alors on dispose des équations $\lambda_2 + \lambda_4 = 0$ et $\lambda_2 \lambda_4 = -a$, d'où $a = \lambda_2^2$ est bien un carré dans $A$.
    \end{enumerate}
\end{proof}
Maintenant considérons $f$ comme un élément de $k[X][Y]$, alors par analyse de degré en $X$ on remarque que $X(X-1)(X-a)$ n'est pas un carré dans $k[X]$, et ainsi par le lemme précédent $f$ est irréductible dans $k[X][Y] = k[X,Y]$.
\begin{question}{1.}
    Comme $k$ est algébriquement clos et $f$ est irréductible,
    \begin{align*}
        K[V] = k[X,Y]/I(C) = k[X,Y]/I(V(f)) = k[X,Y]/\sqrt{(f)} = k[X,Y]/(f) =: k[x,y]
    \end{align*}
    où $x = [X], y = [Y] \in k[X,Y]/(f)$ (et donc $y^2 = x(x-1)(x-a)$).
\end{question}

\section*{Exercice 6, feuille 3}
    Si deux variétés sont isomorphes, alors leurs algèbres de fonctions régulières sont isomorphes. La dimension d'une variété étant égale à la dimension de Krull de leurs algèbres de fonctions régulières, deux variétés isomorphes sont de même dimension. Pour terminer l'exercice, considérons les isomorphismes
    \begin{align*}
        \begin{array}{cccc}
            & \mathbb{A}^1 & \to & V(X - Y) \subseteq \mathbb{A}^2 \\
            & x & \mapsto & (x, x) \\
        \end{array}
    \end{align*}
    \begin{align*}
        \begin{array}{cccc}
            & \mathbb{A}^2 & \to & V(X - Y) \subseteq \mathbb{A}^3 \\
            & (x,y) & \mapsto & (x, x, y) \\
        \end{array}
    \end{align*}
    Ainsi $V(X - Y) \subseteq \mathbb{A}^2$ est de dimension $1$, alors que $V(X - Y) \subseteq \mathbb{A}^3$ est de dimension $2$.

\section*{Exercice 7, feuille 3}

    \begin{enumerate}
        \item Première méthode : si le corps $k$ est infini, on a déja bu
        
        
        on a déja vu dans un exercice précédent que $V(X^2 - Y, Y^2 - Z) = \{(t,t^2,t^4) \in \mathbb{A}^3 \mid t \in k\}$ et $I(V(X^2 - Y, Y^2 - Z)) = (X^2 - Y, Y^2 - Z)$. Ainsi considérons l'application
        \begin{align*}
            \begin{array}{cccc}
                \varphi & \mathbb{A}^3 & \to & \mathbb{A}^2 \\
                & (x,y,z) & \mapsto & (x^2 - y, y^2 - z) \\
            \end{array}
        \end{align*}
        sa jacobienne en $(x,y,z) \in \mathbb{A}^3$ est donné par
        \begin{align*}
            \begin{bmatrix}
                2x & -1 & 0 \\
                0 & 2y & -1 \\
            \end{bmatrix}
        \end{align*}
        Maintenant soit $(x,y,z) \in V(X^2 - Y, Y^2 - Z)$, alors $\exists t \in k$ tq $(x,y,z) = (t, t^2, t^4)$. Mais alors
        
    \end{enumerate}

\end{document}

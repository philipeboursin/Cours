\chapter{TD1}
    \section*{Exercice 1}
        Soit $V \subset \mathbb{A}^1$ un souos ensemble algébrique, alors il existe $M \subseteq k[x]$ tq $V = V(M)$. Maintenant $V(M) = V((M))$ et comme $k[x]$ est principal, il existe $P \in k[x]$ tq $V = V(P)$. Remarquons alors que $P \neq 0$ car sinon $V(P) = V(0) = \mathbb{A}^1$. Mais alors $V(P) = \{a \in \mathbb{A}^1 \mid P(a) = 0\}$ donc c'est l'ensemble des racines, qui est un ensemble fini (de cardinal inférieur à $\deg P$).

    \section*{Exercice 2}
        Vérifions la double inclusion : L'inclusion $\mathfrak{m}_a \subseteq \ker ev_a$ est triviale. Réciproquement, prenons $P \in k[x_1, \cdots, x_n]$ tq $P(a) = 0$. Alors par divisions euclidiennes successives, on peut écrire
        \begin{align*}
            P(x_1, \cdots, x_n) = Q_1(x_1, \cdots, x_n)(x_1 - a_1) + \cdots + Q_n(x_1, \cdots, x_n)(x_n - a_n) + r
        \end{align*}
        avec $r$ un polynôme constant. Alors $r = 0$ puisque $P(a) = 0$ et ainsi $P \in \mathfrak{m}_a$.

    \section*{Exercice 3}
        Par récurrence (flemme)
    
    \section*{Exercice 4}
        

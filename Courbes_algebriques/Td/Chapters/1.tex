\chapter{TD1}
    \section{Exercice 1}
        Soit $V \subset \mathbb{A}^1$ un souos ensemble algébrique, alors il existe $M \subseteq k[x]$ tq $V = V(M)$. Maintenant $V(M) = V((M))$ et comme $k[x]$ est principal, il existe $P \in k[x]$ tq $V = V(P)$. Remarquons alors que $P \neq 0$ car sinon $V(P) = V(0) = \mathbb{A}^1$. Mais alors $V(P) = \{a \in \mathbb{A}^1 \mid P(a) = 0\}$ donc c'est l'ensemble des racines, qui est un ensemble fini (de cardinal inférieur à $\deg P$).

    \section{Exercice 2}
        Vérifions la double inclusion : L'inclusion $\mathfrak{m}_a \subseteq \ker ev_a$ est triviale. Réciproquement, prenons $P \in k[x_1, \cdots, x_n]$ tq $P(a) = 0$. Alors par divisions euclidiennes successives, on peut écrire
        \begin{align*}
            P(x_1, \cdots, x_n) = Q_1(x_1, \cdots, x_n)(x_1 - a_1) + \cdots + Q_n(x_1, \cdots, x_n)(x_n - a_n) + r
        \end{align*}
        avec $r$ un polynôme constant. Alors $r = 0$ puisque $P(a) = 0$ et ainsi $P \in \mathfrak{m}_a$.

    \section{Exercice 3}
        \label{exo3}
        Soit $k$ un corps infini. On montre par récurrence sur $n$ que $I(\mathbb{A}_k)^n = 0$ :
        \begin{enumerate}
            \item Si $n = 1$, alors $I(\mathbb{A}_k^n) = \{f \in k[x] \mid \forall a \in k,\, f(a) = 0\}$. Mais alors soit $f \in I(\mathbb{A}_k^n)$, $f$ a une infinité de racines, donc $f$ est forcément nul (tout polynôme $g$ non nul ayant au maximum $\deg g$ racines).
            \item Soit $f \in I(\mathbb{A}_k^n)$. Alors regardons $f$ comme un élément de $k[x_1, \cdots, x_{n-1}][x_n]$ : 
            \begin{align*}
                f = \sum Q_i x_n^i
            \end{align*}
            avec $Q_i \in k[x_1, \cdots, x_{n-1}]$. Maintenant fixons $(a_1, \cdots, a_{n-1}) \in k^{n-1}$, alors pour tout $t \in k$
            \begin{align*}
                f(a_1, \cdots, a_{n-1}, t) = 0
            \end{align*}
            donc le polynome $\sum Q_i(a_1, \cdots, a_n)x_n^i \in k[x_n]$ est nul (on utilise l'initialisation). Ainsi chaque $Q_i(a_1, \cdots, a_n)$ est nul, et ceci pour tout $(a_1, \cdots, a_{n-1}) \in k^{n-1}$. Ainsi par hypothèse de récurrence les $Q_i \in k[x_1, \cdots, x_{n-1}]$ sont nuls et alors $f$ est nul, donc $I(\mathbb{A}_k^n) = 0$.
        \end{enumerate}
    
    \section{Exercice 4}
        $\supseteq$ est trivial. Réciproquement, soit $f \in \mathbb{F}_q[x]$ tel que $f(a) = 0$, pour tout $a \in \mathbb{F}_q$. Remarquons alors que $x^q - x$ s'annule sur tout $\mathbb{F}_q$ et a au maximum $q$ racines, donc doit forcément s'écrire $x^q - x = \prod_{a \in \mathbb{F}_q} (x - a)$. Maintenant, on peut factoriser $f$ en
        \begin{align*}
            f = g \prod_{a \in \mathbb{F}_q} (x - a) = g (x^q - x) \in (x^q - x)
        \end{align*}
        et donc l'inclusion réciproque est prouvée.

    \section{Exercice 5}
        \begin{question}{1)}
            Montrons que $V = V(x^2 + y2 - 1)$ : il est clair que $V \subseteq V(x^2 + y^2 - 1)$. Réciproquement, soit $(a,b) \in \mathbb{R}^2$ tels que $a^2 + b^2 - 1 = 0$. Alors $a \in [-1,1]$ donc il existe $t \in \mathbb{R} \mid x = \cos t$. Et alors $b^2 = 1 - (\cos t)^2 = (\sin t)^2$ donc $b = \pm \sin t$. Si $b = \sin t$, alors on a terminé, sinon posons $t' = -t$, alors $a = cos t'$ et $b = \sin t'$ et donc $(a,b) \in V$.
        \end{question}
        \begin{question}{2)}
            Supposons que $V_2$ est algébrique, disons $V_2 = V(I)$ pour $I \subrel{id} k[x,y]$. Alors prenons $P \in I$, on a $P(t, \sin t) = 0$ pour tout $t \in \mathbb{R}$. Mais alors regardons $P$ comme un polynôme de $k[x][y]$
            \begin{align*}
                P = \sum Q_i y^i
            \end{align*}
            avec $Q_i \in k[x]$. Alors fixons $t \in \mathbb{R}$, alors $\sum Q_i(t) y^i \in k[y]$ admet une infinité de racines, puisque $\sin (t + 2k\pi)$ sont des racines, pour $k \in \mathbb{Z}$ : en effet, $P(t,\sin (t + 2k\pi)) = P(t, \sin t) = 0$. Ainsi $\sum Q_i(t) y^i =0 \in k[y]$. Donc pour tout $t \in \mathbb{R}$, $Q_i(t) = 0$ et donc $Q_i = 0 \in k[x]$, et ainsi $P = 0$. Mais alors $I = 0$, donc $V_2 = \mathbb{A}_\mathbb{R}^2$ absurde.
        \end{question}
        \begin{question}{3)}
            Supposons que $V_3 = V(I)$. Alors soit $P \in I$, alors $P(t, \e^t) = 0$ pour tout $t \in \mathbb{R}$. Supposons que $P$ est non nul, alors regardons $P$ comme un élément de $k[x][y]$
            \begin{align*}
                P = \sum_{n = 1}^k Q_n y^n
            \end{align*}
            où $Q_k \neq 0$. Alors
            \begin{align*}
                0 = \sum_{n=1}^k Q_n(t) e^{nt} \iff 0 = \sum_{n=1}^k Q_n(t) e^{(n-k)t}
            \end{align*}
            et alors en passant à la limite, par croissances comparées on obtiens que $Q_n(t) \xrightarrow[]{t \to \infty} 0$ et donc $Q_n = 0 \in k[x]$ absurde. Ainsi $P = 0$, donc $V_3 = \mathbb{A}_\mathbb{R}^2$, absurde.
        \end{question}

    \section{Exercice 6}
        \begin{question}{1)}
            Il est clair que $V_1 = V(y - x^2, z - x^3)$.
        \end{question}
        \begin{question}{2)}
            Montrons que $V_2 = V(xy - 1)$ : $\subseteq$ est claire. Réciproquement, soit $(a,b) \in V(xy - 1)$, alors $ab = 1$. Maintenant $a$ et $b$ sont non nuls, et alors $b = 1/a$, donc $(a,b) = (a, 1/a) \in V_2$.
        \end{question}
        \begin{question}{3)}
            Remarquons dans un premier temps que
            \begin{align*}
                V_3 = \{(t, (t+1)^2 - 1) \in \mathbb{A}^2 \mid t \in k\} = \{(t, t^2 + 2t)\in \mathbb{A}^2 \mid t \in k\}
            \end{align*}
            Ainsi il est clair que $V_3 = V(x^2 + 2x - y)$.
        \end{question}

    \section{Exercice 7}
        \begin{question}{1)}
            Soit $(x,y) \in V(I)$. Alors $xy^3 = 0$ et $x^2 + y^2 = 0$. Alors
            \begin{enumerate}
                \item Soit $x = 0$ et alors $y^2 = 0$ donc $y = 0$
                \item Soit $y^3 = 0 \Rightarrow y = 0 \Rightarrow x^2 = 0 \Rightarrow x = 0$
            \end{enumerate}
            et ainsi $V(I) = \{0\}$. Soit $(x,y) \in V(J)$, alors $x^2 = 0$ et $y^3 = 0$, donc $x = 0$ et $y = 0$. AInsi $V(J) = \{0\}$.
        \end{question}
        \begin{question}{2)}
            $I(V(I)) = I(V(J)) = (x,y)$.
        \end{question}

    \section{Exercice 8}
        \begin{question}{1)}
            Comme $k$ est un corps infini, $I(V) = 0$ (cf \ref{exo3}). On a donc $V(I(V)) = \mathbb{A}^2$.
        \end{question}
        \begin{question}{2)}
            Comme $V \neq V(I(V))$, $V$ n'est pas un ensemble algébrique affine.
        \end{question}

    \section{Exercice 9}
        \begin{question}{1)}
            Oui, vu qu'un singleton n'a aucun sous ensemble propre.
        \end{question}
        \begin{question}{2)}
            Non. Une paire de points et l'union de deux points qui sont des sous-ensembles algébriques propres de cette paire de points.
        \end{question}
        \begin{question}{3)}
            Non : d'après le cours, $V(xy) = V(x) \cup V(y)$.
        \end{question}
        \begin{question}{4)}
            Si le corps n'est pas infini, alors $V(X-Y) = V((X-Y)^2)$ est un union fini disjoint de points, donc n'est pas irréductible. Si le corps est infini, montrons que $I(V(x-y)) = I(V((x-y)^2)) = (x - y)$ : $\supseteq$ est donné directement par le cours. Réciproquement, soit $P \in I(V((x-y)^2))$, alors $V((x-y)^2) = \{(t,t) \in \mathbb{A}^2 \mid t \in k\}$ et donc $P(t,t) = 0$ pour tout $t \in k$. Ainsi si on considère $P$ en tant qu'élément de $k[x][y]$ puis qu'on réalise la division euclidienne de celui-ci par $x-y$, alors on obtiens
            \begin{align*}
                P = Q_1(x,y)(x-y) + R(x,y)
            \end{align*}
            et $R$ s'identifie à un polynôme de $k[x]$ vu que $\deg_Y R < 1$. Mais alors $|k| = \infty$ et $R(t) = 0$ pour tout $t \in k$, donc finalement $R = 0$ et $P \in (x,y)$. Pour conclure, remarquons (au vu de ce que l'on vient de faire) que $(x-y)$ est le noyau de
            \begin{align*}
                \begin{array}{cccc}
                    & k[x,y] & \to & k[t] \\
                    & P & \mapsto & P(t,t) \\
                \end{array}
            \end{align*}
            donc finalement $k[x,y]/(x-y) = k[t]$ qui est intègre donc $(x-y)$ est premier, prouvant l'irréductibilité de $V(x-y) = V((x - y)^2)$.
        \end{question}
        \begin{question}{5)}
            $V(y - x^2) = \{(t,t^2) \mid t \in k\}$. Montrons alors que $I(V(y -x^2)) = (y - x^2)$ (si $|k| = \infty$). Si $k$ est fini, alors $V(y - x^2)$ contiens au moins deux points ($(0,0)$ et $(1,1)$ par exemple) et n'est donc pas irréductible. Sinon, prouver l'égalité souhaitée revient à prouver que le noyau de
            \begin{align*}
                \begin{array}{cccc}
                    \varphi : & k[x,y] & \to & k[t] \\
                    & P & \mapsto & P(t,t^2) \\
                \end{array}
            \end{align*}
            vaut $(y - x^2)$ (du fait que dans un corps infini un polynome est nul si et seulement si sa fonction polynomiale associée est nulle). Mais alors soit $P \in \ker \varphi$, on réalise la division euclidienne de $P$ par $y - x^2$ dans $k[x][y]$ :
            \begin{align*}
                P = Q(y - x^2) + R(x,y)
            \end{align*}
            mais $R$ s'identifie à un polynôme de $k[x]$ puisque $\deg_y R < 1$. Mais alors $R(a) = 0$ pour tout $a \in k$ et comme $|k| = \infty$, $R = 0$ et donc $P \in (y - x^2)$. L'inclusion réciproque est triviale. Finalement, on a bien $\ker \varphi = (y - x^2)$ et donc $I(V(y - x^2)) = (y - x^2)$ est un idéal premier, du fait que $k[x,y]/(y - x^2) \simeq k[t]$ qui est un anneau intègre.
        \end{question}
        \begin{question}{6)}
            \begin{enumerate}
                \item $V(x^2 - y^2) = V((x - y)(x + y)) = V(x - y) \cup V(x + y)$, donc $V(x^2 - y^2)$ n'est pas irréductible en caractéristique différente de $2$. En caractéristique $2$,
                \begin{align*}
                    V(x^2 - y^2) = V(x^2 + y^2) = V((x + y)^2) = V(x + y)
                \end{align*}
                est irréductible si et seulement si $|k| = \infty$.
                \item On sépare en deux cas
                \begin{enumerate}
                    \item S'il existe $i \in k$ tel que $i^2 = -1$, alors $V(x^2 + y^2) = V(x - iy) \cup V(x + iy)$ et ces sous-ensembles sont popres si $\mathrm{char} \, k \neq 2$. En caractéristique $2$, $V(x^2 + y^2) = V((x + y)^2) = V(x + y)$ qui est irréductible si $|k| = \infty$, et réductible sinon.
                    \item Si $-1$ n'est pas un carré dans $k$, alors $V(x^2 + y^2) = \{0\}$ : soit $(a,b) \in V(x^2 + y^2)$, alors $a^2 + b^2 = 0$. Alors si $a$ est non nul,
                    \begin{align*}
                        b^2 = -a^2 \iff \left( \frac{b}{a} \right)^2 = -1
                    \end{align*}
                    absurde. Ainsi $a = 0$ et donc $b = 0$. $V(x^2 + y^2)$ est donc irréductible dans ce cas. 
                \end{enumerate}
            \end{enumerate}
        \end{question}
        \begin{question}{7)}
            Montrons que $V(y^4 - x^2, y - x) = \{\pm (1,1)\}$ : si $(a,b) \in V(y^4 - x^2, y - x)$ alors $a = b$ et $a^2 = b^4$. Ainsi $a^2 = a^4$ et donc $a^2 = 1$, donc soit $a = 1$ et donc $b = 1$, soit $a = -1$ et donc $b = -1$. Ainsi si la caractéristique est différente de $2$, c'est un ensemble réductible, sinon il est irréductible car composé d'un seul point.
        \end{question}

    \section{Exercice 10}
        \begin{question}{1)}
            Montrons que $I(V(x^3)) = (x)$ : clairement, $V(x^3) = \{(0,b,c) \in \mathbb{A}^3 \mid \}$. Maintenant soit $P \in I(V(x^3))$, alors $P(0,b,c) = 0$ pour tous $b,c \in k$. Mais alors en réalisant la division euclidiennez de $P$ par $x$ dans $k[y,z][x]$, on voit facilement que $P \in (x)$ (dans le cas où $|k| = \infty$). Finalement, $k[x,y,z]/(x) \simeq k[y,z]$ qui est un anneau intègre, donc $V(x^3)$ est irréductible.
        \end{question}
        \begin{question}{2)}
            \cor{aled}
        \end{question}
        \begin{question}{3)}
            Tout d'abord, si le corps est fini, alors $V(Y^2 - X^3)$ contiens $(0,0)$ et $(1,1)$, donc n'est pas irréductible. Supposons maintenant que $|k| = \infty$, montrons que
            \begin{align*}
                V(Y^2 - X^3) = \{(t^2, t^3) \in k^2 \mid t \in k\} =: V
            \end{align*}
            Si $(x,y) \in V$, alors $\exists t \in k \mid (x,y) = (t^2, t^3)$. Et alors $y^2 - x^3 = t^6 - t^6 = 0$, donc $(x,y) \in V(Y^2 - X^3)$. Réciproquement, si $(x,y) \in V(Y^2 - X^3)$, alors $y^2 = x^3$ dans $k$. Et alors si $x = 0$, alors $y = 0$ et $(0,0) \in V$. Sinon,
            \begin{align*}
                &\left(\frac{y}{x}\right)^2 = x \\
                &\left(\frac{y}{x}\right)^3 = y \\
            \end{align*}
            donc en posant $t = y/x$, $(x,y) = (t^2, t^3) \in V$. \\
            Ensuite, montrons que $I(V(Y^2 - X^3)) = (Y^2 - X^3)$ : remarquons dans un premier temps que pour tout $P \in k[T]$, $P = 0 \iff P(t) = 0$, $\forall t \in k$ du fait que $|k| = \infty$. Ainsi prouver $(Y^2 - X^3) = I(V(Y^2 - X^3))$ reviens à prouver que le noyau de
            \begin{align*}
                \begin{array}{cccc}
                    \varphi : & k[X,Y] & \to & k[T] \\
                    & P & \mapsto & P(T^2,T^3) \\
                \end{array}
            \end{align*}
            vaut $(Y^2 - X^3)$. En effet, $P \in I(V(Y^2 - X^3)) \iff P(t^3, t^3) = 0$, $\forall t \in k$ $\iff P(T^2, T^3) = 0$ au vu de la remarque faite précédemment, donc $\ker \varphi = I(V(Y^2 - X^3))$. Il est clair que $(Y^2 - X^3) \subseteq \ker \varphi$. Réciproquement, soit $P \in \ker \varphi$, réalisons la division euclidienne de $P$ par $Y^2 - X^3$ dans $k[X][Y]$ :
            \begin{align*}
                P(X,Y) = Q(X,Y)(Y^2 - X^3) + R(X,Y)
            \end{align*}
            où $\deg_Y R \leq 1$. Ecrivons alors $R(X,Y) = a(X)Y + b(X)$, montrons que $a$ et $b$ sont nuls. Développons alors $a$ et $b$ : si on écrit 
            \begin{align*}
                &a(X) = \sum_{i \geq 0} a_i X^i \\
                &b(X) = \sum_{i \geq 0} b_i X^i \\
            \end{align*}
            on a 
            \begin{align*}
                R(T^2, T^3) &= a(T^2)T^3 + b(T^2) \\
                &= \sum_{i \geq 0} a_i T^{2i + 3} + b_i T^{2i} \\
                &= \sum_{i \geq 0} a_i T^{2i + 3} + b_i T^{2i} \\
                &= \sum_{j \geq 0} c_j T^j \\
            \end{align*}
            où
            \begin{align*}
                c_j =
                \begin{cases}
                    a_i &\text{ si } j = 2i + 3 \text{ pour un certain } i \in \mathbb{N} \\
                    b_i &\text{ si } j = 2i \text{ pour un certain } i \in \mathbb{N} \\
                    0 & \text{ sinon}
                \end{cases}
            \end{align*}
            (Les coefficients de $a(T^2)T^3$ n'intéragissent pas avec ceux de $b(T^2)$, car devant des monômes de degré impair alors que ceux de $b(T^2)$ n'aparaissent que devant des monômes de degré pair). Ainsi comme $P \in \ker \varphi$, $0 = \varphi(R) = R(T^2, T^3)$ et donc $a_i, b_i = 0$ pour tout $i \geq 0$. Finalement, $a,b = 0$ et donc $R = 0$, d'où $P \in (Y^2 - X^3)$. Ainsi on a bien égalité $I(V(Y^2 - X^3)) = (Y^2 - X^3)$, et $V(Y^2 - X^3)$ est irréductible puisque $K[X,Y]/(Y^2 , X^3)$ s'injecte dans $k[T]$ qui est lui-même intègre.
        \end{question}

    \section{Exercice 14}
        \begin{question}{1)}
            Il est clair que $V = V(X_2 - X_1^2, \cdots, X_n - X_1^n)$.
        \end{question}
        \begin{question}{2)}
            Montrons que $I(V) = (X_2 - X_1^2, \cdots, X_n - X_1^n)$. $\supseteq$ est claire, montrons l'inclusion réciproque : soit $P \in I(V)$, alors on peut écrire $P = \sum_{i = 2}^n Q_i (X_i - X_1^i) + R$
            où $R \in k[X_1]$. Maintenant pour tout $t \in k$, $P(t, t^2, \cdots, t^n) = R(t) = 0$ et donc comme $k$ est de caractéristique nulle, il est infini et $R = 0$. Finalement $P \in (X_2 - X_1^2, \cdots, X_n - X_1^n)$ et on a égalité. Finalement le noyau du morphisme
            \begin{align*}
                \begin{array}{cccc}
                    & k[X_1, \cdots, X_n] & \to & K[T] \\
                    & X_i & \mapsto & T^i\\
                \end{array}
            \end{align*}
            est de noyau $(X_2 - X_1^2, \cdots, X_n - X_1^n) = I(V)$, et est surjectif, donc $k[V] \simeq k[T]$
        \end{question}
        \begin{question}{3)}
            $k[V]$ est intègre, donc $V$ est irréductible.
        \end{question}

    \section{Exercice 15}
        Soient $V_1 \subseteq \mathbb{A}_k^n$, $V_2 \subseteq \mathbb{A}_k^m$ des ensembles algébriques affines. On note
        \begin{align*}
            &k[x_1, \cdots, x_n] =: A \\
            &k[y_1, \cdots, y_m] =: B \\
            &k[x_1, \cdots, x_n, y_1, \cdots, y_m] =: C \\
        \end{align*}
        Alors il existe $I \subrel{id} A$ et $J \subrel{id} B$ tels que $V_1 = V(I)$ et $V_2 = V(J)$. Considérons le morphisme
        \begin{align}
            \varphi := p_I \otimes p_J : A \otimes_k B \to A/I \otimes_k B/J
        \end{align}
        Où $p_I : A \to A/I$, $p_J : B \to B/J$ sont les projections canoniques des quotients respectifs. On sait que le morphisme $A \otimes_k B \to C$ induit par les morphismes canoniques $i_1 : A \to C$, $i_B : B \to C$ (issus de la propriété universelle des anneaux de polynômes) est un isomorphisme ($\sum_{finie} P_i \otimes Q_i$ est envoyé sur $\sum_{finie} i_A(P_j)i_B(Q_j)$). Une dernière remarque est qu'au vu de la naturalité de $\Hom_{\mathbf{Sets}}(S,k) \simeq \Hom_{k-\mathbf{CAlg}}(k[S],k)$, nous avons la commutativité du diagramme
        \begin{figure}[H]
            \centering
            \begin{tikzcd}
                A \arrow[r, "i_1"] \arrow[rd, "\mathrm{ev}_a"'] & C \arrow[d, "{\mathrm{ev}_{(a,b)}}"] & B \arrow[l, "i_2"'] \arrow[ld, "\mathrm{ev}_b"] \\
                                                                & k                                    &                                                
                \end{tikzcd}
        \end{figure} \noindent
        Pour terminer l'exercice, montrons que $V(\ker \varphi) = V_1 \times V_2$ (où $\ker \varphi$ est vu comme un idéal de $C$ par l'isomorphisme naturel donné précédemment). Prenons $(a,b) \in V(\ker \varphi)$, puis soient $P \in I$, $Q \in J$. Alors $P \otimes 1, 1 \otimes Q \in \ker \varphi$ et donc
        \begin{align*}
            0 = \mathrm{ev}_{(a,b)}(i_1(P)i_2(1)) =  P(a)
        \end{align*}
        et de même, $Q(b) = 0$, et ainsi $(a,b) \in V_1 \times V_2$. Réciproquement, soit $(a,b) \in V_1 \times V_2$. Alors tout élément de $\ker \varphi$ s'écrit comme une somme finie $\sum_{\mathrm{\mathrm{finie}}} P_j \otimes Q_j$. Mais
        \begin{align*}
            \mathrm{ev}_{(a,b)}\left(\sum_{\mathrm{finie}} i_A(P_j)i_B(Q_j)\right) = \sum_{\mathrm{finie}} P_j(a)Q_j(b) = 0
        \end{align*}
        et ainsi $(a,b) \in V(\ker \varphi)$.

    \section{Exercice 16}
        
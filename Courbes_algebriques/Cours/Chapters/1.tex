\chapter{Ensembles algébriques affines}
    \section{Définition}
        $k$ un corps, $n \in \mathbb{Z}$.
        \begin{defi} (Espace affine)
            $\mathbb{A}^n_k := k^n$ est l'espace affine sur le corps $k$ de dimension $n$.
        \end{defi}
        \begin{remq}
            Ce n'est pas vraiment la définition de l'espace affine, c'est la définition de l'ensemble sous-jacent à l'espace affine, sachant que les espaces affines sont des variétés algébriques.
        \end{remq}
        \cor{
        \begin{expl}
            Si $n = 1$, c'est une "droite". Si $n = 2$, c'est un "plan".
        \end{expl}
        }
        \begin{defi}
            Soit $S \subseteq k[x_1, \cdots, x_n]$, on définit
            \begin{align*}
                V(S) := \{a \in \mathbb{A}_k^n \mid \forall P \in S ,\, P(a) = 0 \}
            \end{align*}
            On appelle de tels ensembles des ensembles algébriques affines.
        \end{defi}
        \begin{remq}
            Si $S = \{P_1, \cdots, P_r\}$, on écrit $V(P_1, \cdots, P_r) := V(S)$.
        \end{remq}
        \begin{expl}
            \begin{enumerate}
                \item $V(\emptyset) = \mathbb{A}_k^n$
                \item $V(1) = \emptyset$
                \item $P = X^4 - 1 \in k[X]$, si $k = \mathbb{R}$, $V(P) = \{1, -1\}$. Si $k = \mathbb{C}$, $V(P) = \{1,-1,i,-i\}$. Si $k = \mathbb{F}_2$, $V(P) = \{1\}$.
                \item $P = X^2 + Y^2 + 1 \in k[X,Y]$, si $k = \mathbb{R}$, $V(P) = \emptyset$. Si $k = \mathbb{C}$, $V(P)$ est isomorphe (en tant que variété algébrique, même si cela n'a pour le moment aucun sens) au cercle complexe (en considérant le changement de variables $a_j = ib_j$).
                \item $P_i = \sum a_{ij} x_j - b_i \in k[x_1, \cdots, x_n]$, $i \in \lcc 1,r \rcc$.
                \begin{align*}
                    V(P_i) = \{x \in k^n \mid (a_{ij})x = b\} \simeq \mathbb{A}_k^n \text{ ou } \emptyset
                \end{align*}
            \end{enumerate}
        \end{expl}
        \begin{exo}
            Les ensembles algébriques de $\mathbb{A}^1_k$ sont : $\emptyset$, $\mathbb{A}_k^1$, tous les sous-ensembles finis.
        \end{exo}
        \begin{expl}
            Les sous-ensembles algébriques de $\mathbb{A}^2_k$ sont $\emptyset$, tout le plan, les sous-ensembles finis et des réunions finies des sous-ensembles finis avec des courbes planes, i.e. $V(P) \neq \emptyset$ les zéros d'un seul polynôme non constant. Donnons des exemples de courbes planes :
            \begin{enumerate}
                \item Les droites $V(ax + by + c) \in \mathbb{A}_k^2$, avec $a \neq 0$ ou $b \neq 0$.
                \item Les coniques $V(ax^2 + by^2 + cxy + dx + ey + f) \subseteq \mathbb{A}_k^2$ ($a \neq 0$ ou $b \neq 0$ ou $c \neq 0$). Dans $\mathbb{P}_\mathbb{C}^2$, toutes les coniques sont de type cercle, droite ou droites qui se croisent.
                \item $y^2 = x^3 + ax + b$, $a,b \in k$ définissent ce qu'on appelle des courbes elliptiques.
            \end{enumerate}
        \end{expl}
        \begin{remq}
            $V(S) = V(T)$ n'implique pas que $S = T$. Par exemple $V(x^2 + y^2 + 1) = V(x^4 + 1) \subseteq \mathbb{A}_\mathbb{R}^2$. Plus généralement, sur n'importe quel corps, $V(P^2) = V(P)$ avec $P = k[x_1, \cdots, x_n]$.
        \end{remq}
        \begin{prop}
            \begin{enumerate}
                \item Si $S \subseteq T \subseteq k[x_1, \cdots, x_n]$, alors $V(T) \subseteq V(S) \subseteq \mathbb{A}_k^n$.
                \item $S \subseteq k[x_1, \cdots, x_n]$, $I = (S)$ idéal engendré par $S$, alors $V(S) = V(I)$
                \item $S \subseteq k[x_1, \cdots, x_n]$, alors
                \begin{align*}
                    V(S) = \bigcap_{p \in S} V(P)
                \end{align*}
                \item
                \begin{align*}
                    \bigcap_{j \in J} V(S_j) = V\left(\bigcup_{j \in J} S_j\right),\, S_j \subseteq k[x_1, \cdots, x_n]
                \end{align*}
                \item $V(PQ) = V(P) \cup V(Q)$ pour $P,Q \in k[x_1, \cdots, x_n]$
                \item Plus généralement, $V(IJ) = V(I) \cup V(J) = V(I \cap J)$ avec $I,J \subrel{id} k[x_1, \cdots, x_n]$
            \end{enumerate}
        \end{prop}
        \begin{proof}
            Prouvons 6 : $IJ \subseteq I \cap J \subseteq I$ donc $V(I) \subseteq V(I \cap J) \subseteq V(IJ)$ et donc par symétrie $V(I) \cup V(J) \subseteq V(I \cap J) \subseteq V(IJ)$. Supposons qu'il existe $x \in V(IJ)$ tq $x \notin V(I) \cup V(J)$. Alors $\exists P \in I$, $Q \in J$ tq $P(x) \neq 0$ et $Q(x) \neq 0$. Mais $PQ \in IJ$ donc $PQ(x) = 0$, contradiction. Les autres points sont en exercice.
        \end{proof}
        \begin{coro}
            Les ensembles algébriques de $\mathbb{A}_k^n$ forment les fermés d'une topologie. On appelera cette topologie la topologie de Zariski.
        \end{coro}
        \begin{defi}
            Soit $E \subseteq \mathbb{A}_k^n$. On définit
            \begin{align*}
                I(E) = \{P \in k[x_1, \cdots, x_n] \mid P(a) = 0 ,\, \forall a \in E\}
            \end{align*}
        \end{defi}
        \begin{expl}
            \begin{enumerate}
                \item $I(\emptyset) = k[x_1, \cdots, x_n]$
                \item $I(a) = (x_1 - a_1, \cdots, x_n - a_n) =: \mathfrak{m}_a$. Remarquons que cet idéal est un idéal maximal.
                \item $I(\mathbb{A}_k^n) = \{0\}$ si le corps est infini.
            \end{enumerate}
        \end{expl}
        \begin{defi}
            $I \subrel{id} A$, alors 
            \begin{align*}
                \sqrt{I} = \{f \in A \mid \exists n > 0,\, f^n \in I\}
            \end{align*}
            est le radical de $I$. $I$ est un idéal radical si $I = \sqrt{I}$
        \end{defi}
        \begin{prop}
            \begin{enumerate}
                \item $E \subseteq E' \subseteq \mathbb{A}_k^n$, alors $I(E') \subseteq I(E)$
                \item $I(E \cup E') = I(E) \cap I(E')$
                \item $J \subseteq I(V(J))$ pour tout $J \subrel{id} k[x_1, \cdots, x_n]$.
                \item $E \subseteq V(I(E))$ pour tout $E \subseteq \mathbb{A}_k^n$.
                \item $V(I) = V(\sqrt{I}) \subseteq \mathbb{A}_k^n$, pour tout $I \subrel{id} k[x_1, \cdots, x_n]$
            \end{enumerate}
        \end{prop}
        \begin{proof}
            Exercice
        \end{proof}
        \begin{lemm}
            $E = V(I(E)) \iff E$ est un ensemble algébrique. 
        \end{lemm}
        \begin{proof}
            Montrons $V(I(E)) \subseteq E$ : Supposons que $E = V(J)$, $J \subrel{id} k[x_1, \cdots, x_n]$. ALors $J \subseteq I(V(J))$ et ainsi $V(I(E)) \subseteq E$.
        \end{proof}
        \begin{expl}
            Le segment ouvert $(0,1) \subseteq \mathbb{A}_\mathbb{R}^1$ n'est pas un ensemble algébrique.
        \end{expl}
        \begin{theo} (Nullstellensatz, 1)
            \label{Null_1}
            Si $k = \bar k$, alors on a $I(V(J)) = \sqrt{J}$ pour tout $J \subrel{id} k[x_1, \cdots, x_n]$
        \end{theo}
        \begin{expl}
            Si $k = \mathbb{R}$, $P = x^2 + y^2 + 1 \in \mathbb{R}[x,y]$ irréductible. $I = (P)$ est un idéal premier, donc radical, mais $I(V(P)) = I(\emptyset) = \mathbb{R}[x,y] \neq (P)$.
        \end{expl}
        \begin{theo}
            Pour tout $n \geq 1$, $k[x_1, \cdots, x_n]$ est un anneau noéthérien.
        \end{theo}
        \begin{coro}
            Chaque ensemble algébrique $V \subseteq \mathbb{A}_k^n$ est de la forme $V = V(P_1, \cdots, P_r)$ avec $P_i \in k[x_1, \cdots, x_n]$
        \end{coro}
        Ainsi $V$ et $I$ nous donnent des applications entre les idéaux radicaux de $k[x_1, \cdots, x_n]$ et les sous espaces algébriques de $\mathbb{A}_k^n$. Vérifier que $I(E)$ est un idéal radical. De plus, si $k$ est algébriquement clos, d'après le nullstellensatz $I$ et $V$ sont inverses l'une de l'autre. Par cette bijection, les idéaux premiers vont correspondre aux ensembles irréductibles. Les idéaux maximaux vont correspondre à des points.
        \begin{defi}
            $V \subseteq \mathbb{A}_k^n$ ensemble algébrique. $V$ est irréductible si pour toute décomposition $V = V_1 \cup V_2$ avec $V_1,V_2$ ensembles algébriques, on a $V = V_1$ ou $V = V_2$. On dit sinon que $V$ est réductible.
        \end{defi}
        \begin{prop}
            $V \subseteq \mathbb{A}_k^n$ ensemble algébrique. Alors tfae
            \begin{enumerate}
                \item $V$ est irréductible
                \item $I(V)$ est un idéal premier
                \item $k[x_1, \cdots, x_n]/I(V)$ est un anneau intègre
            \end{enumerate}
        \end{prop}
        \begin{proof}
            1 $\Rightarrow$ 2 : Soient $f,g \in k[x_1, \cdots, x_n]$ tq $fg \in I(V)$. Mais $V(fg) = V(f) \cup V(g)$, puis soit $V_1  V \cap V(f)$, $V_2 = V \cap V(g)$, alors $V_1 \cup V_2 = V \cap V(fg) = V$. Ainsi $V_1 = V$ ou $V_2 = V$, donc $f \in V$ ou $g \in V$. \\
            2 $\Rightarrow$ 1 : Soit $V \subseteq \mathbb{A}_k^n$ ensemble algébrique tq $I(V)$ est un idéal premier. Supposons que $V$ est réductible, alors $V = V_1 \cup V_2$ avec $V \neq V_1, V \neq V_2$. Comme $V_1,V_2$ sont algébriques, alors $V(I(V)) = V$, $V(I(V_i)) = V_i$, et ainsi $V(I(V)) \neq V(I(V_1))$ et $I(V) \subseteq I(V_1)$. Donc il existe $f_1 \in I(V_1)$ tq $f \notin I(V)$. De même, il existe $f_2 \in I(V_2)$ tq $f_2 \notin I(V)$. Mais $f_1f_2 \in I(V_1) \cap I(V_2) = I(V)$ et ainsi $I(V)$ n'est pas premier.
        \end{proof}
        \begin{theo}
            Soit $V \subseteq \mathbb{A}_k^n$ un ensemble algébrique. Alors $\exists V_1, \cdots, V_m \subseteq \mathbb{A}_k^n$ irréductibles tels que
            \begin{enumerate}
                \item $V = V_1 \cup V_2 \cup \cdots \cup V_m$
                \item $\forall i \neq j$, $V_i \nsubseteq V_j$
            \end{enumerate}
            Les $\{V_i\}_{i \in \lcc 1,m \rcc}$ avec ces propriétés sont uniques à ordre près, on les appelle les composantes irréductibles de $V$.
        \end{theo}
        \begin{expl}
            Soit $V := V(xy, (x-1)z) \subseteq \mathbb{A}_k^n$, $k$ de caractéristique $0$. Sur $V$, on a 
            \begin{align*}
                &(x = 0 \lor z = 0) \land (x = 1 \lor y = 0) \\
                \iff &(x = 0 \land y = 0) \lor (z = 0 \land x = 1) \lor (z = 0 \land y = 0)
            \end{align*}
            Ainsi $V = V_1 \cup V_2 \cup V_3$ avec $V_1 = V(x,y)$, $V_2 = V(x-1, z)$ et $V_3 = V(y,z)$. On peut alors prouver que ce sont les composantes irréductibles de $V$.
        \end{expl}
        \begin{proof}
            Soit $V \subseteq \mathbb{A}_k^n$ un ensemble algébrique. Si $V$ est irréductible, on a terminé. Sinon il existe des sous-ensembles algébriques propres de $V_1, V_2 \nsubseteq V$ tels que $V = V_1 \cup V_2$. Si $V_1,V_2$ sont irréductibles, alors on a finit. Sinon on itère le procédé sur $V_1$ et $V_2$. Alors supposons que le procédé ne termine pas, il va exister une suite strictement décroissante $\cdots \nsubseteq W_2 \nsubseteq W_1 \nsubseteq V$ d'ensembles algébriques. Ainsi on obtiens une suite croissante
            \begin{align*}
                I(W) \subseteq I(W_1) \subseteq I(W_2) \subseteq \cdots
            \end{align*}
            Remarquons alors qu'elle es strictement croissante puisque $V(I(W_i)) = W_i$ et la suite des $W_i$ est strictement décroissante. Ainsi on obtiens une contradiction avec le fait que $k[x_1, \cdots, x_n]$ est noéthérien. \\
            Occupons nous maintenant de l'unicité : Supposons que 
            \begin{align*}
                V = \bigcup_{i = 1}^s V_i = \bigcup_{i = 1}^t W_i
            \end{align*}
            On veut montrer que l'ensemble $\{V_i\}_{i \in \lcc 1,s \rcc}$ est égal à l'ensemble $\{W_i\}_{i \in \lcc 1,t \rcc}$. On va montrer une inclusion : montrons qu'il existe $j \in \lcc 1,t \rcc$ tel que $V_i = W_j$, avec $i \in \lcc 1,s \rcc$. Comme $V_i \subseteq \bigcup_{j \in \lcc 1,t \rcc} W_j$, on a
            \begin{align*}
                V_i \subseteq \bigcup_{j \in \lcc 1,t \rcc} W_j \cap V_i
            \end{align*}
            Mais $V_i$ est irréductible, donc $\exists j \in \lcc 1,t \rcc$ tel que $V_i = W_j \cap V_j$, et en particulier $V_i \subseteq W_j$. Maintenant de la même manière on peut prouver qu'il existe $i' \in \lcc 1,s \rcc$ tel que $W_i \subseteq V_{i'}$. Mais alors $V_i \subseteq W_j \subseteq V_{i'}$ et donc $i = i'$, d'où $V_i = W_j$.
        \end{proof}
        Donnons $2$ reformulations du Nullstellensatz
        \begin{prop} (Nullstellensatz 2,3)
            Considérons l'anneau $k[x_1, \cdots, x_n]$. Tfae :
            \begin{enumerate}
                \item Pour tout $J \subrel{id} k[x_1, \cdots, x_n]$, $I(V(J)) = \sqrt{J}$
                \item Pour tout $J \subrel{id} k[x_1, \cdots, x_n]$, $J$ propre implique que $V(J) \neq \emptyset$
                \item Les idéaux maximaux de $k[x_1, \cdots, x_n]$ sont exactement les idéaux
                \begin{align*}
                    \mathfrak{m}_a = (x_1 - a_1, \cdots, x_n - a_n)
                \end{align*}
            \end{enumerate}
        \end{prop}
        % \begin{prop} (Nullstellensatz, 2)
        %     \label{Null_2}
        %     Soit $k$ un corps algébriquement clos. Si $J \subrel{id} k[x_1, \cdots, x_n]$ est un idéal propre, alors $V(J) \neq \emptyset$.
        % \end{prop}
        % \begin{prop} (Nullstellensatz, 3)
        %     \label{Null_3}
        %     Soit $k$ un corps algébriquement clos. Les idéaux maximaux de $k[x_1, \cdots, x_n]$ sont exactement les $\mathfrak{m}_a = (x_1 - a_1, \cdots, x_n - a_n)$.
        % \end{prop}
        \begin{proof}
            2 $\Rightarrow$ 3 : Soit $\mathfrak{m} \subrel{max} k[x_1, \cdots, x_n]$. C'est un idéal propre, donc $V(\mathfrak{m}) \neq \emptyset$. Alors soit $a \in V(\mathfrak{m})$, remarquons que pour tout $f \in \mathfrak{m}$, $f(a) = 0$ donc $f \in \mathfrak{m}_a$ (vu que l'on peut écrire $f = Q_1(x_1 - a_1) + \cdots + Q_i(x_i - a_i) + c$). Ainsi $\mathfrak{m} \subseteq \mathfrak{m}_a$ mais $\mathfrak{m}$ est maximal donc $\mathfrak{m} = \mathfrak{m}_a$ ce qui prouve simultanément que $(x_1 - a_1, \cdots, x_n - a_n)$ est un idéal maximal et que $\mathfrak{m}$ est cet idéal. \\
            1 $\Rightarrow$ 2 : Soit $J \subrel{id} k[x_1, \cdots, x_n]$ idéal propre. On a $\sqrt{J} = I(V(J))$. Supposons que $V(J) = \emptyset$, alors $\sqrt{J} = I(V(J)) = k[x_1, \cdots, x_n]$ et donc $J = k[x_1, \cdots, x_n]$, contradiction.\\
            3 $\Rightarrow$ 1 : Soit $I \subrel{id} k[x_1, \cdots, x_n]$, on veut mq $\sqrt{I} = I(V(I))$. Comme $I \subseteq I(V(I))$, on a directement le première inclusion du fait que $\sqrt{I(V(I))} = I(V(I))$. Dans l'autre sens, si $I = k[x_1, \cdots, x_n]$, l'égalité est claire. Sinon soit $f \in I(V(I))$, écrivons $I = (P_1, \cdots, P_r)$. Maintenant considérons l'anneau $k[x_1, \cdots, x_n, x_{n+1}]$, puis l'idéal
            \begin{align*}
                (P_1, \cdots, P_r, 1 - x_{n+1}f) =: J \subrel{id} k[x_1, \cdots, x_{n+1}]
            \end{align*}
            Si $J$ est un idéal propre, alors d'après le théorème de Krull il existe $\mathfrak{m} \subrel{max} k[x_1, \cdots, x_{n+1}]$ tel que $J \subseteq \mathfrak{m}$. Maintenant par hypothèse il existe $(a_1, \cdots, a_n, b) \in \mathbb{A}_k^{n+1}$ tel que
            \begin{align*}
                \mathfrak{m} = (x_1 - a_1, \cdots, x_n - a_n, x_{n+1} - b)
            \end{align*}
            Mais alors pour tout $i \in \lcc 1,r \rcc$, $P_i(a) = 0$ et $1 - bf(a) = 0$. Mais alors la première série d'égalités nous indique que $a \in V(I)$, et comme $f \in I(V(I))$, $f(a) = 0$, ce qui est absurde. Ainsi $J$ est $k[x_1, \cdots, x_{n+1}]$ tout entier, donc en particulier il existe $Q_1, \cdots, Q_n, Q \in k[x_1, \cdots, x_{n+1}]$ tels que
            \label{1.1}
            \begin{align}
                1 = P_1Q_1 + \cdots + P_rQ_r + Q(1 + x_{n+1}f)
             \end{align}
            Maintenant le morphisme de localisation $k[x_1, \cdots, x_n] \to k[x_1, \cdots, x_n, 1/f]$ et le choix de l'élément $1/f$ induit un morphisme d'évaluation
            \begin{figure}[H]
                \centering
                \begin{tikzcd}
                    {k[x_1, \cdots, x_n]} \arrow[r] \arrow[d]                      & {k[x_1, \cdots, x_n, 1/f]} \arrow[r, hook] & {k(x_1, \cdots, x_n)} \\
                    {k[x_1, \cdots, x_n][x_{n+1}]} \arrow[ru, "!\exists"', dashed] &                                            &                      
                    \end{tikzcd}
            \end{figure}
            Ainsi au travers de ce morphisme l'égalité \ref{1.1} deviens
            \begin{align*}
                1 = P_1(x_1, \cdots, x_n)Q_1(x_1, \cdots, x_n, 1/f) + \cdots + P_r(x_1, \cdots, x_n)Q_r(x_1, \cdots, x_n, 1/f)
            \end{align*}
            Alors écrivons les $Q_i$ comme des éléments de $k[x_1, \cdots, x_n][x_{n+1}]$,
            \begin{align*}
                Q_i = \sum_{l = 0}^{d_i} R_{i,l}(x_1, \cdots, x_n)x^l_{n+1}
            \end{align*}
            En les passant au travers du morphisme d'évaluation précédent on peut les réécrire
            \begin{align*}
                Q_i = \frac{R_i(x_1, \cdots, x_n)}{f^{d_i}}
            \end{align*}
            et alors \ref{1.1} deviens
            \begin{align*}
                1 = \sum_{i = 1}^r \frac{P_iR_i}{f^{d_i}}
            \end{align*}
            et ainsi en notant $d = \max \{d_i\}$
            \begin{align*}
                f^d = \sum_{i = 1}^r P_iR_if^{d - d_i}
            \end{align*} 
            dans $k(x_1, \cdots, x_n)$ donc dans $k[x_1, \cdots, x_n]$. Finalement si $d = 0$, alors $1 \in I$ absurde puisque l'on avait supposé $I$ propre. Sinon, $f^d \in I$ et donc $f \in \sqrt{I}$.
        \end{proof}
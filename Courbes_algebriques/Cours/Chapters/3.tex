\chapter{Dimension, espace tangent}
    \section{Topologie induite sur les ensembles algébriques}
        $\mathbb{A}_k^n$ est muni d'une topologie, dont les fermés sont les $V(I)$ pour $I$ un idéal de $k[X_1, \cdots, X_n]$. Ainsi on définit la topologie de Zariski sur $V \subseteq \mathbb{A}_k^n$ un esnemble algébrique comme la topologie induite sur $V$. Plus concrètement, les fermés de $V$ sont les $V(I) \cap V$, pour $I$ un idéal de $k[X_1, \cdots, X_n]$ (i.e. les ensembles algébriques $W \subseteq V$).
        \begin{exo}
            Les ouverts distingués $D(f)$ forment une base pour la topologie de zariski de $\mathbb{A}^n$.
        \end{exo}
        Ainsi $\{D(f) \cap V\}_f$ est une base des ouverts pour la topologie de Zariski sur $V$ un ensemble algébrique fixé.
        \begin{prop}
            Soient $V,W \subseteq \mathbb{A}^n, \mathbb{A}^l$. Tout morphisme $\varphi : V \to W$ est continu pour la topologie de zariski induite sur $V$ et $W$.
        \end{prop}
        \begin{proof}
            \ajout{
                Dans un premier temps, soit $f_i \in k[X_1, \cdots, X_n]$ tels que $\varphi(a) = (f_1(a), \cdots, f_i(a))$. Alors montrer que $\varphi$ est continue revient à montrer que $\tilde \varphi : \mathbb{A}^n \to \mathbb{A}^l$ définie par $\tilde \varphi(a) = (f_1(a), \cdots, f_i(a))$ pour tout $a \in \mathbb{A}^n$ est continue. En effet, soit $Z$ un fermé de $W$, alors $Z = Z' \cap W$ pour $Z$ un fermé de $\mathbb{A}^l$. Maintenant $\varphi^{-1}(Z) = \tilde \varphi^{-1}(Z') \cap V$ et est donc un fermé si et seulement si $\tilde \varphi^{-1}(Z')$ est un fermé. On peut donc se ramener au cas $\varphi : \mathbb{A}^n \to \mathbb{A}^l$. Ainsi considérons un fermé $V(J) \subrel{fermé} \mathbb{A}^l$, posons $I := k[\varphi](J)$, et montrons que $\varphi^{-1}(V(J)) = V(I)$.
                \item $\subseteq$ : soit $x \in \varphi^{-1}(V(J))$, alors pour tout $P \in I$, il existe $Q \in J$ tel que $P = k[\varphi](Q)$. Maintenant
                \begin{align*}
                    P(x) &= k[\varphi](Q)(x) = Q(\varphi(x)) = 0
                \end{align*}
                puisque $Q \in J$ et $\varphi(x) \in V(J)$.
                \item $\supseteq$ : Soit $x \in V(I)$, alors pour tout $Q \in J$, $k[\varphi](Q) \in I$ et donc
                \begin{align*}
                    Q(\varphi(x)) = k[\varphi](Q)(x) = 0 \\
                \end{align*}
                et donc $\varphi(x) \in V(J)$.
            }
        \end{proof}
        \begin{exo}
            Soit $f : X \to Y$ un morphisme dans $\mathbf{Top}$, si $X$ est irréductible, alors $\overline{f(X)}$ irréductible.
        \end{exo}
        \begin{expl}
            ($k = \bar k$) $f : \mathbb{A}^1 \to V := \{(x,y) \mid y^2 = x^3\}$ est surjectif, donc $V$ est irréductible.
        \end{expl}
        \begin{expl}
            $V = \{(x,y) \mid xy = 1\} \subseteq \mathbb{A}^2$. Notons $f : V \to \mathbb{A}^1$ la projection sur la première coordonnée, alors $f(V) = \mathbb{A}^1 \bs \{0\}$ n'est pas fermé (si $|k| = \infty$) et donc ne peut pas être un ensemble algébrique.
        \end{expl}
        \begin{exo}
            $E \subseteq \mathbb{A}^n_k$ ensemble quelconque, alors $\bar E = V(I(E))$. \ajout{Soit $E \subseteq V(J)$ un fermé Alors $J \subseteq I(V(J)) \subseteq I(E)$ et donc $V(I(E)) \subseteq V(J)$, ce qui prouve que $V(I(E)) = \bar E$.}
        \end{exo}

    \section{Variétés affines, dimension}
        \begin{defi} (Variété affine)
            Une variété affine est un ensemble algébrique affine irréductible.
        \end{defi}
        Ainsi si $V$ est une variété affine, alors $k[V] = k[X_1, \cdots, X_n]/I(V)$ est intègre (vu que $I(V)$ est un idéal premier).
        \subsection{Dimension d'une variété affine}
            \begin{defi}
                $k(V) := \mathrm{Frac}\, k[V]$ est le corps de fonctions rationnelles sur $V$.
                \begin{align*}
                    k(V) = \left\{ \frac{P}{Q} \mid P,Q \in k[V],\, Q \neq 0 \right\}
                \end{align*}
            \end{defi}
            \begin{defi} ($k = \bar k$)
                $V$ variété affine. On définit la dimension de $V$ par
                \begin{align*}
                    \mathrm{dim} V := \mathrm{trdeg}_k k(V)
                \end{align*}
                où $\mathrm{trdeg}_k k(V)$ est le degré de transcendance de $k(V)$ sur $k$.
            \end{defi}
            \begin{defi}
                $k \subseteq K$ extension de corps.
                \begin{enumerate}
                    \item Une partie $S \subseteq K$ est algébriquement indépendante si pour tout $m \geq 1$, tout $s_1, \cdots, s_m \in S$, si $P \in k[X_1, \cdots, X_m]$ est tel que $P(s_1, \cdots, s_m) = 0$, alors $P = 0$.
                    \item $S \subseteq K$ est une base de transcendance de $K$ (sur $k$) si $S$ est algébriquement indépendante et $k(S) \subseteq K$ est algébrique.
                    \item On dit que $k \subseteq K$ est purement transcendante si $\exists S$ base de transcendance $k \subseteq k(S) = K$.
                \end{enumerate}
            \end{defi}
            \begin{remq}
                Si $|S| = n$, alors $k(S) \simeq k(X_1, \cdots, X_n)$. Si $S_1,S_2$ sont deux bases de transcendance de $K/k$, alors $|S_1| = |S_2|$. 
            \end{remq}
            \begin{remq}
                $\dim V \leq n$ pour toute variété algébrique dans $\mathbb{A}^n$.
            \end{remq}
            \begin{defi}
                $\mathrm{trdeg}_k(K) = |S|$, $S$ base de transcendance de $K/k$.
            \end{defi}
            \begin{expl}
                \begin{enumerate}
                    \item $\dim \mathbb{A}_k^n = n$ : $V = \mathbb{A}_k^n$, $k[V] = k[X_1, \cdots, X_n]$. $I(V) = \{0\}$. Ainsi $k(V) = k(X_1, \cdots, X_n)$. Et $\{X_1, \cdots, X_n\}$ est une base de transcendance de $k(V)$, donc $\dim V = n$.
                    \item $V = \{(x,y) \mid xy = 1\} \subseteq \mathbb{A}^1$. Alors $V = V(XY - 1)$ est irréductible; Ainsi $V$ est une variété affine. $k[V] = k[X,Y]/(XY - 1) = k[x,y]$ où $x = [X]$, $y = [Y]$ (et $xy = 1$). $k(V) = \mathrm{Frac} (k[x,y]) =: k(x,y)$. Maintenant $k(x,y) = k(x)$ vu que $y = 1/x$. Maintenant $\{x\}$ est une base de trascendance de $k(x)$ : sinon il existe $P \in k[X]$ non nul tel que $P(x) = 0 \in k(x)$, et en particulier dans $k[x] \subseteq k[V]$. Ainsi $P \in I(V)$ donc $P(X) = (XY - 1)Q(X,Y)$ dans $k[X,Y]$ avec $Q \in k[X,Y]$, ce qui est absude puisque $\deg_YP = 0$. Ainsi $\dim V = 1$
                \end{enumerate} 
            \end{expl}
            \begin{lemm}
                ($k = \bar k$) Soit $f \in k[X_1, \cdots, X_n]$ irréductible. Alors $V := V(f) \subseteq \mathbb{A}_k^n$ est une variété affine de dimension $n-1$.
            \end{lemm}
            \begin{proof}
                $f$ non constant. On peut supposer que $\deg_{X_n}(f) > 0$. Notons $k[V] = k[x_1, \cdots, x_n]$. Ainsi $f(x_1, \cdots, x_n) = 0$ vu que $I(V) = (f)$. . Maoitenant $k \subseteq k(x_1, \cdots, x_{n-1}) \subseteq k(x_1, \cdots, x_{n-1})(x_n)$ est algébrique car $f(x_1, \cdots, x_n) = 0$. Montrons que $\{x_1, \cdots, x_{n-1}\}$ sont algébriquement indépendants sur $k$ : si $g \in k[X_1, \cdots, X_{n-1}]$ tel que $g(x_1, \cdots, x_{n-1}) = 0$ dans $k(V)$ (donc dans $k[V]$). Alors $g(x_1, \cdots, x_{n-1}) \in I(V) = (f)$ mais $\deg_{X_n} g = 0$, donc $g$ est nul.
            \end{proof}
            \begin{remq}
                \label{rq113}
                Soient $f_1, \cdots, f_r \in k[X_1, \cdots, X_n]$, $V := V(f_1, \cdots, f_r) \subseteq \mathbb{A}^n_k$. Supposons que $V$ est irréductible, alors $\dim V \geq n-r$. \cor{Preuve en exercice}
            \end{remq}
            \begin{expl}
                $V := V(Y - X^2, Z - X^3, XZ - Y^2) \subseteq \mathbb{A}^3_k$. Alors $V = \{(t,t^2, t^3) \mid t \in k\}$ est une variété affine de dimension $1$ (on parle de courbe affine). En effet, $V$ est irréductible, puis $k[V] \simeq k[T]$ donc $\frac k[V] \simeq k(T)$ est de degré de transcendance $1$ sur $k$. Comme $V$ est définie par $3$ équations, $XZ - Y^2$ peut s'exprimer en fonction des deux autres polynômes et est donc superflue.
            \end{expl}
            \begin{remq}
                Si $V,W$ sont des variétés affines isomorphes, alors $k[V] \simeq k[W]$ et ainsi $k(V) \simeq k(W)$ donc $\dim V = \dim W$. Dans l'exemple précédente, on peut conclure que $V$ est de dimension $1$ avec cette remarque.
            \end{remq}
            \begin{expl}
                $V = V(XZ - Y^2 ,  XW - YZ, YW - Z^2) = V(f_1, f_2, f_3) \subseteq \mathbb{A}^4$. On sait que $\dim V \geq 1$ d'après la remarque \ref{rq113}. En fait, $\dim V = 2$, et $f_i \notin (f_j, f_l)$ pour tout $i,j,k$ différents deux à deux. Montrons par exemple que $f_3 \notin (f_1, f_2)$ : Soit $W = V(f_1, f_2)$, si $f_3 \in (f_1, f_2)$, alors $V = W$. Mais $V \cap \{x = 0\} \neq W \cap \{x = 0\}$ :
                \begin{align*}
                    &V \cap \{x = 0\} = \{(0,0,0,w) \mid w \in k\} \\
                    &W \cap \{x = 0\} = \{(0,0,z,w) \mid z,w \in k\} \\
                \end{align*}
                On peut montrer les autres de manière similaire, ainsi on ne peut pas éliminer d'équation, mais pourtant $\dim V = 2$ : Soit $V' = V(f_1)$, $V'' = V(f_1, f_2)$ ($V \subseteq V'' \subseteq V'$). Calculons la dimension de ces différentes variétés :
                \item 
                \begin{align*}
                    k[V'] = k[X,Y,Z,W]/(XZ - Y^2) = k[x,y,z,w]
                \end{align*}
                où $xz = y^2$. Alors $k(V') = k(x,y,z,w) = k(x,y,w)$ puisque $z = y^2/x$ dans $k(V')$. Enfin on peut prouver que $x,y,z$ sont algébriquement indépendants, et donc $\dim V' = 3$. 
                \item $k[V''] = k[x,y,z,w]$ avec $xz = y^2$, $sw = yz$. Ainsi $k(V'') = k(x,y)$ puisque $z = y^2/x$, $w = yz/x$, et on peut prouver que $x,y$ sont algébriquement indépendants, i.e. $\dim V'' = 2$. 
                \item $k(V) = k(x,y,z,w)$, mais $z = y^2/x$, $w = yz/x$, mais la $3$eme équation ne nous donne pas d'autre relation (c'est la même que celle donnée par la première équation). Il est donc possible de prouver que $x,y$ sont algébriquements indépendnts et alors $\dim V = 2$.
            \end{expl}

        \subsection{Dimension de krull}
            Soit $A \in \mathbf{CRings}$.
            \begin{defi} (Dimension de Krull)
                \begin{align*}
                    \dim A := \sup \{l \geq 0 \mid \mathfrak{p}_0 \nsubseteq \mathfrak{p}_1 \nsubseteq \cdots \nsubseteq \mathfrak{p}_l \subseteq A\}
                \end{align*}
            \end{defi}
            \begin{expl}
                \begin{enumerate}
                    \item $A = k[X_1, \cdots, X_n]$. Alors considérons $\mathfrak{p}_i = (X_1, \cdots, X_i)$ pour $0 \leq i \leq n$, on a donc $\dim A \geq n$. On peut en fait montrer que $\dim A = n$. 
                    \item La dimension d'un corps vaut $0$,
                    \item $\dim \mathbb{Z} = 1$.
                \end{enumerate}
            \end{expl}
            \begin{remq}
                $\mathfrak{p} \subseteq A$ idéal premier, alors
                \begin{align*}
                    \dim (A/\mathfrak{p}) = \sup \{l \geq 0 \mid \mathfrak{p} = \mathfrak{p}_0 \nsubseteq \cdots \nsubseteq \mathfrak{p}_l \}
                \end{align*}
            \end{remq}
            \begin{defi} (Hauteur)
                Soit $\mathfrak{p} \subrel{pr} A$, alors
                \begin{align*}
                    ht(\mathfrak{p}) = \sup \{s \geq 0 \mid \mathfrak{p}_0 \nsubseteq \cdots \nsubseteq \mathfrak{p}_s = \mathfrak{p} \}
                \end{align*}
            \end{defi}
            \begin{expl}
                \begin{enumerate}
                    \item $A = \mathbb{Z}$, alors $ht(p \mathbb{Z}) = 1$. 
                    \item $A = k[T]$, alors $ht((f)) = 1$.
                    \item $A = k[X_1, \cdots, X_n]$, $\mathfrak{p} = (X_1, \cdots, X_s)$. Alors $ht \mathfrak{p} = s$.
                \end{enumerate}
            \end{expl}
            \begin{theo}
                $k$ corps, $A$ $k$-algèbre de type fini. Soit $\mathfrak{p}$ idéal premier, alors 
                \begin{align*}
                    ht \mathfrak{p} + \dim A/\mathfrak{p} = \dim A
                \end{align*}
            \end{theo}
            \begin{expl}
                \begin{enumerate}
                    \item $A = \mathbb{Z}$, $\mathfrak{p} = p \mathbb{Z}$ avec $p$ premier; Alors $ht \mathfrak{p} = 1$, $A/ \mathfrak{p} = \mathbb{F}_p$ est de dimension $0$.
                    \item $A = k[X_1, \cdots, X_n]$, $\mathfrak{p} = (X_1, \cdots, X_s)$ avec $s \in \lcc 1,n \rcc$. $ht \mathfrak{p} = s$, $A/\mathfrak{p} \simeq k[X_{s+1}, \cdots, X_n]$, $\dim A/\mathfrak{p} = n-s$.
                    \item La dimension peut être infinie : par exemple $A = k[\mathbb{N}]$.
                \end{enumerate}
            \end{expl}
            \begin{theo}
                Soit $V$ une variété affine. Alors
                \begin{align*}
                    \dim k[V] = \dim V
                \end{align*}
            \end{theo}

    \section{Singularités}
        Soit $V \subseteq \mathbb{A}^n_k$ une variété affine. Notons $d = \dim V$, et soit $I(V) = (P_1, \cdots, P_r) \subrel{id} k[X_1, \cdots, X_n]$. On a $d \geq n - r$. Considérons le morphisme
        \begin{align*}
            \begin{array}{cccc}
                \varphi : & \mathbb{A}^n & \to & \mathbb{A}^r \\
                & x & \mapsto & (P_1(x), \cdots, P_r(x)) \\
            \end{array}
        \end{align*}
        En particulier, $\varphi_{|V} = 0$.
        \begin{nota}
            On note
            \begin{align*}
                d \varphi (x) = 
                \begin{bmatrix}
                    \frac{\partial P_1}{\partial X_1}(x) & \cdots & \frac{\partial P_1}{\partial X_n}(x) \\
                    \vdots & & \vdots \\
                    \frac{\partial P_r}{\partial X_1}(x) & \cdots & \frac{\partial P_r}{\partial X_n}(x) \\
                \end{bmatrix}
                \in M_{r, n}(k)
            \end{align*}
            la matrice jacobienne de $\varphi$ en $x$.
        \end{nota}
        \begin{defi}
            $a \in V$ est un point régulier (ou encore non singulier) de $V$ si 
            \begin{align*}
                rk (d \varphi(a)) = n - d
            \end{align*}
            Si $rk (d \varphi(a)) < n - d$, on dit que $a$ est un point singulier de $V$.
        \end{defi}
        \begin{remq}
            \begin{enumerate}
                \item $\forall a \in V$, $rk (d \varphi(a)) \leq n - d$
                \item La définition précédente ne dépend pas du choix de $P_1, \cdots, P_r$
            \end{enumerate}
        \end{remq}
        \begin{defi}
            $V$ est lisse si $\forall a \in V$, $a$ est un point régulier.
        \end{defi}
        \begin{expl}
            \begin{enumerate}
                \item $\mathbb{A}^n$ est lisse.
                \item $V = V(f) \subseteq \mathbb{A}^n$, pour $f \in k[X_1, \cdots, X_n]$ irréductible est une variété affine de dimension $n-1$?. Alors
                \begin{align*}
                    d \varphi =
                    \begin{bmatrix}
                        \frac{\partial f}{X_1}, \cdots, \frac{\partial f}{X_n}
                    \end{bmatrix}
                \end{align*}
                Ainsi $a \in V(f)$ est singuler ssi $rk (d \varphi(a)) = 0$ ssi $\frac{\partial f}{X_1}(a) = \cdots = \frac{\partial f}{X_n}(a) = 0$.
                \item Si $f = XY - 1$, $V = V(f) \subseteq \mathbb{A}^2$. Alors $\frac{\partial f}{X} = Y$ et $\frac{\partial f}{Y} = X$, alors $a \in V$ singulier si $a_2 = a_1 = 0$, mais $a \in V \iff a_1a_2 = 1$, donc tout point de $V$ est régulier et $V$ est donc lisse.
                \item $f = Y^2 - X^3$, $V := V(f) \subseteq \mathbb{A}^2$. Si $char \, k \neq 2,3$, alors $a \in V$ singulier ssi $a = (0,0)$. 
                \item $f = Y^2 - X(X-1)(X - \lambda)$, $\lambda \in k$ ("courbe ellipitique" si $\lambda \neq 0,1$). $f = Y^2 - X^3 + (\lambda + 1)X^2 - \lambda X$. Alors
                \begin{align*}
                    & \frac{\partial f}{\partial X} = -3X^2 + 2(\lambda + 1)X - \lambda \\
                    & \frac{\partial f}{\partial Y} = 2Y \\
                \end{align*}
                et donc $(a,b) \in V(f)$ singulier ssi
                \begin{align*}
                    &\begin{cases}
                    -3x^2 + 2(\lambda + 1)x - \lambda = 0 \\
                    2y = 0 \\
                    y^2 = x(x-1)(x - \lambda) \\
                    \end{cases}
                    &\iff 
                    \begin{cases}
                        y = 0 \\
                        x = 0,1,\lambda \\
                        -3x^2 + 2(\lambda + 1)x - \lambda = 0 \\
                    \end{cases}
                \end{align*}
                Alors
                \begin{enumerate}
                    \item Si $x = 0$, $\lambda = 0$ et $(0,0)$ est le seul point singulier.
                    \item Si $x = 1$, alors $\lambda = 1$ et $(1,0)$ est le seul point singulier.
                    \item Si $x = \lambda$, alors $\lambda = 0,1$ et donc c'est les cas précédents.
                \end{enumerate}
                Ainsi si $\lambda \neq 0,1$, alors $V(f)$ est lisse.
            \end{enumerate}
        \end{expl}

    \section{Espace tangent}
        \subsection{Anneau des fonctions régulières en un point}
            Soit $V \subseteq \mathbb{A}^n_k$ une variété affine, $a \in V$. Par définition,
            \begin{align*}
                k[V] = \frac{k[X_1, \cdots, X_n]}{I(V)}
            \end{align*}
            puis $k(V) = \Frac k[V]$.
            \begin{defi}
                $\alpha \in k(V)$ est bien définie au point $a \in V$ si $\exists f,g \in k[V]$ telles que $\alpha = f/g \in k(V)$ et $g(a) \neq 0$. Dans ce cas, la valeur de $\alpha$ en $a$ est définie comme $\alpha(a) := f(a)/g(a) \in k$.
            \end{defi}
            \begin{remq}
                En général, pour $\alpha \in k(V)$, on peut toujours écrire $\alpha = f/g$ mais $f$ et $g$ ne sont pas uniques.
            \end{remq}
            \begin{expl}
                $V = V(Y^2 - X^3) \subseteq \mathbb{A}^2$; Alors $k[V] = k[x,y]$ avec $y^2 = x^3$. Alors 
                \begin{align*}
                    \frac yx = \frac{x^2}y \in k(V)
                \end{align*}
            \end{expl}
            \begin{prop}
                Soit $\alpha \in k(V)$ bien définie en $a \in V$. Alors $\alpha(a)$ est bien définie.
            \end{prop}
            \begin{proof}
                Si $\alpha = f/g = f'/g'$ avec $f,g,f',g' \in k[V]$ et $g(a), g'(a) \neq 0$. Alors $fg' - f'g = 0 \in k[V]$ et donc $f(a)g'(a) - f'(a)g(a) = 0$, ce qui implique que
                \begin{align*}
                    \frac{f(a)}{g(a)} = \frac{f'(a)}{g'(a)} \in k
                \end{align*}
            \end{proof}
            \begin{nota}
                \begin{align*}
                    k[V]_a := \{\alpha \in k(V) \mid \alpha \text{ est définie en } a \}
                \end{align*}
                C'est un sous anneau de $k(V)$, dit anneau local de fonctions régulières autour de $a$.
            \end{nota}
            \begin{remq}
                \begin{enumerate}
                \item $k \subseteq k[V] \subseteq k[V]_a \subseteq k(V)$
                \item $a \in V \iff I(V) \subseteq \mathfrak{m}_a$. Si $g \in k[V]$, $g(a) = 0 \iff g \in \mathfrak{m}_a/I(V) \subrel{max} k[V]$.
                \item $k[V]_a$ est la localisation de $k[V]$ en $S = A \bs \mathfrak{p}$, où $\mathfrak{p} = \mathfrak{m}_a/I(V)$. 
                \end{enumerate}
            \end{remq}

        \subsection{Anneaux locaux}
            Rappelons que si $A$ est un anneau, $S = A \bs \mathfrak{p}$, l'image $S^{-1} \mathfrak{p}$ de $\mathfrak{p}$ par le morphisme canonique $A \to S^{-1}A$ est l'unique idéal maximal de $S^{-1}A$. En effet, $S^{-1} A \bs S^{-1}\mathfrak{p}$ sont des inversibles de $S^{-1}A$. Ainsi $S^{-1}A$ est un anneau local. 
            \begin{nota}
                On notera $(A, \mathfrak{m})$ les anneaux locaux, avec $\mathfrak{m}$ leur unique idéal maximal. $A/\mathfrak{m}$ est le corps résiduel de $A$.
            \end{nota}
            \begin{prop}
                Soit $A$ un anneau noethérien intègre, $S \subseteq A$ un ensemble multiplicatif. Alors $S^{-1}A$ est noethérien.
            \end{prop}
            \begin{proof}
                Soit $I \subrel{id} S^{-1}A$. Ainsi le morphisme $\varphi : A \to S^{-1}A$ est injectif. Alors $J := \varphi^{-1}(I)$ est un idéal de $A$, qui est de type fini. Ainsi il existe $P_1, \cdots, P_r \in J$ tel que $J = (P_1, \cdots, P_r)$. Montrons que $I = (\varphi(P_1), \cdots, \varphi(P_r)) = (P_1/1, \cdots, P_r/1)$. Soit $a/s \in I$, alors 
                \begin{align*}
                    \varphi(a) = \frac a1 = \frac as \frac S1
                \end{align*}
                donc il existe $f_1, \cdots, f_r \in A$ tq $a = \sum f_i P_i \in A$. Et donc
                \begin{align*}
                    \frac as = \sum \frac{f_i}s \frac{P_i}1
                \end{align*}
                et donc $a/s \in (P_1/1, \cdots, P_r/1)$.
            \end{proof}
            \begin{coro}
                Soit $V$ une variété algébrique. Alors $k[V]_a$ est un anneau noethérien.
            \end{coro}
            \begin{proof}
                $k[V]$ est noethérien et intègre car $V$ est une variété algébrique. Ainsi $k[V]_a$ est noéthérien comme localisation de $k[V]$.
            \end{proof}
            \begin{lemm} (Nakayama)
                $(A, \mathfrak{m})$ anneau local noethérien, et $M$ un $A$-module de type fini. Si $\mathfrak{m} M = M$, alors $M = 0$.
            \end{lemm}
            \begin{coro}
                $\mathfrak{m}^i/ \mathfrak{m}^{i+1} = 0 \iff \mathfrak{m}^i = 0$. En particulier, $\mathfrak{m}/\mathfrak{m}^2 = 0 \iff \mathfrak{m} = 0 \iff A$ est un corps.
            \end{coro}
            \begin{remq}
                Soit $(A, \mathfrak{m})$ un anneau local noethérien. $\mathfrak{m}/\mathfrak{m}^2$ est un $A$-module, mais aussi un $k$-ev où $k = A/\mathfrak{m}$ de type fini (en tant que $A$-module et en tant que $k$-ev)
            \end{remq}
            \begin{theo}
                $(A, \mathfrak{m})$ anneau local noethérien, $k = A/\mathfrak{m}$. Alors
                \begin{align*}
                    \dim_k \mathfrak{m}/\mathfrak{m}^2 \geq \dim A
                \end{align*}
                où $\dim A$ est la dimension e krull de $A$. De plus, si on a égalité (on note $d = \dim A$), alors $\mathfrak{m} = (x_1, \cdots ,x_d)$ avec $x_i \in A$, et on dit que $(A, \mathfrak{m})$ est un anneau régulier.
            \end{theo}
        
        \subsection{Espace tangent de Zariski}
            Soit $V$ une variété affine, $a \in V$. Comme $k[V]_a$ est la localisation de $k[V]$ en l'idéal maximal $\mathfrak{m}_a/I(V)$. C'est donc un anneau local d'idéal maximal noté 
            $\mathfrak{m}$.
            \begin{exo}
                Considérons le morphisme
                \begin{align*}
                    \begin{array}{cccc}
                        ev_a : & k[V]_a & \to & k \\
                        & \alpha & \mapsto & ev_a(\alpha) = \alpha(a) \\
                    \end{array}
                \end{align*}
                Alors $\ker(ev) = \mathfrak{m}$ idéal maximal de $k[V]_a$, et $k[V]_a/\mathfrak{m} \simeq k$
            \end{exo}
            \begin{defi}
                $V \subseteq \mathbb{A}^n$ variété affine, $a \in V$. Soit $A = k[V]_a$ l'anneau local associé à $a$. L'espace tangent de $V$ en $a$ est le $k$-ev
                \begin{align*}
                    T_aV := (\mathfrak{m}/ \mathfrak{m}^2)^{\vee} := \Hom_k(\mathfrak{m}/ \mathfrak{m}^2, k)
                \end{align*}
                l'espace tangent à $V$ en $a$.
            \end{defi}
            \begin{remq}
                $\mathfrak{m}_a/\mathfrak{m}_a^2 \simeq M_a/M_a^2$ où $M_a = \ker ev_a : k[V] \to k \subrel{id} k[V]$.
            \end{remq}
            \begin{theo}
                \cor{
                    $\dim k[V] = \dim k[V]_a$
                }
            \end{theo}
            Donc $(A, \mathfrak{m})$ est régulier ssi $\dim V =\dim A = \dim_k \mathfrak{m}/ \mathfrak{m}^2 = \dim_k T_aV$. On va montrer que $(A, \mathfrak{m})$ est régulier si et seulement si $a \in V$ est régulier.
            \begin{theo}
                $\dim_k T_aV \geq \dim V$ avec égalité si et seulement si $a \in V$ est un point régulier.
            \end{theo}
            Pour montrer ce théorème, nous devons parler d'espace tangent géométrique.
        
        \subsection{Espace tangent géométrique}
            Soit $V = V(P_1, \cdots, P_r) \subseteq \mathbb{A}^n$, avec $P_1, \cdots, P_r \in k[X_1, \cdots, X_n]$, et $a \in V$.
            \begin{defi} (Espace tangent géométrique)
                L'espace tangent géométrique de $V$ en $a$ est définit comme
                \begin{align*}
                    T_a^\mathrm{geom} V = V(P^1_1, \cdots, P_r^1) \subseteq \mathbb{A}^n
                \end{align*}
                où 
                \begin{align*}
                    P_i^1 = \sumç_{j = 1}^n \frac{\partial P_i}{\partial X_j}(a) (X_j - a_j)
                \end{align*}
            \end{defi}
            \begin{remq}
                Soit $P \in k[X_1, \cdots, X_n]$, on peut décomposer $P$ en $P = \sum_{i = 0}^d P^i$ où les $P^i$ sont des polynômes homogènes de degré $i$ (tous les monômes sont de la forme $X^\alpha$ avec $|\alpha| = i$) en réalisant l'expansion de taylor de $P$ en $0$. Ainsi, 
                \begin{align*}
                    P^1 = \sum_{j = 1}^n \frac{\partial P}{\partial X_j}(0) X_j
                \end{align*}
                Maintenant si $a \in \mathbb{A}^n$, alors on peut réaliser l'expansion de taylor de $P$ en $a$, $P = \sum P^i$, et alors
                \begin{align*}
                    P^1 = \sum_{j = 1}^n \frac{\partial P}{\partial X_j}(a) (X_j - a_j)
                \end{align*}
            \end{remq}
            \begin{expl}
                \begin{enumerate}
                    \item Soit $V = \{(x,y) \mid y^2 = x\} \subseteq \mathbb{A}^2$. Alors $V = V(P)$ avec $P = Y^2 - X$. Alors soit $a = (a_1, \cdots, a_2) \in V$ (donc $a_2^2 = a_2$), on a 
                    \begin{align*}
                        &\frac{\partial P}{\partial X} = -1 \\
                        &\frac{\partial P}{\partial Y} = 2Y \\
                    \end{align*}
                    On a donc 
                    \begin{align*}
                        T_a^\mathrm{geom} = V(-(X - a_1) + 2a_2(Y - a_2))
                    \end{align*}
                    est la droite tangente au point $(a_1, a_2)$, qui est isomorphe à $k$. En effet, $\dim T_a^\mathrm{geom} = 1$ (c'est le translaté d'un sous-espace de $k^2$ de dimension $1$). De même, $\dim V = 1$ vu que $k[V] = k[Y]$, puis $V$ est lisse donc tout point est régulier.
                    \item Cas particulier : $a = (0,0)$, alors $T_a^\mathrm{geom} V = V(x)$. Maintenant $k[V] \simeq k[X]$, et $k[V]_a = k[x,y]_{(x,y)} \simeq k[x]/(x)$. Finalement, $\mathfrak{m} = (X/1)$ et donc $\dim_k \mathfrak{m}/\mathfrak{m}^2 = 1$ ($X/1$ est une base de ce $k$-ev). Ainsi $\dim T_aV = 1$ en $0$.
                \end{enumerate}
            \end{expl}
            Soit $a \in V$, considérons l'application
            \begin{align*}
                \begin{array}{cccc}
                    \tau_a : & \mathbb{A}^n & \to & \mathbb{A}^n \\
                    & x & \mapsto & x - a \\
                \end{array}
            \end{align*}
            C'est un isomorphisme, et alors on peut considérer l'image par $\tau_a$ de $T_a^\mathrm{geom}V$. Elle est alors munie d'une structure canonique d'espace vectoriel. Alors on a les deux résultats suivants :
            \begin{theo}
                \label{theo353}
                Pour tout $a \in V$,
                \begin{align*}
                    \tau_a(T_a^\mathrm{geom}V) \simeq T_aV
                \end{align*}
            \end{theo}
            \begin{theo}
                \label{theo354}
                Pour tout $a \in V$,
                \begin{align*}
                    \dim T_a^\mathrm{geom}V = n - rk d\varphi(a)
                \end{align*}
            \end{theo}
            Ces deux théorèmes permettent de faire le lien entre les deux notions d'espace tangent, ainsi qu'entre leur dimension et la notion de point régulier. Nous résumons cela dans le corollaire suivant :
            \begin{coro}
                \label{coro351}
                Soit $a \in V$. Alors $\dim T_aV = \dim T_a^\mathrm{geom}V \geq d$, avec égalité si et seulement si $a$ est un point régulier.
            \end{coro}
            \begin{proof} (\ref{coro351})
                Ce résultat découle directement du fait que $rk d\varphi(a) \leq n-d$, avec égalité si et seulement si $a$ est régulier, par définition.
            \end{proof}
            Avant de prouver les deux théorèmes précédents, vérifions les sur un exemple :
            \begin{expl}
                $k = \bar k$, $char k = 0$. Soit $V = V(Y^2 - X^3) \subseteq \mathbb{A}^2$, on a $\dim V = 1$. Ainsi
                \begin{align*}
                    &\frac{\partial P}{\partial X} = -3X^2 \\
                    &\frac{\partial P}{\partial Y} = 2Y \\
                \end{align*}
                Alors $(a,b) \in V$ est singulier si et seulement si $(a,b) = 0$. Maintenant soit $(a,b) \in V$, alors
                \begin{align*}
                    T_a^\mathrm{geom}V = \{(x,y) \in \mathbb{A}^2 \mid -3a^2(x-a) + 2b(Y - b) = 0\}
                \end{align*}
                droite dans $\mathbb{A}^2$ si $(a,b) \neq 0$. Si $(a,b) = 0$, alors $T_0^\mathrm{geom}V = \mathbb{A}^2$ est de dimension $2$. On peut aussi calculer l'espace tangent de Zariski :
                \begin{align*}
                    K[V]_{(a,b)} &= \left( k[X,Y]/(Y^2 - X^3) \right)_{(X-a, Y-b)} \\
                    &= k[x,y]_{(x-a, y-b)} \\ 
                    &= \left\{ \frac PQ \mid P,Q \in k[x,y],\, Q(a,b) \neq 0 \right\}
                \end{align*}
                Alors on peut voir que $\mathfrak{m}_{(a,b)} = (x-a,y-b) \subseteq K[V]_{(a,b)}$. Maintenant $(x-a, y-b)$ engendre $\mathfrak{m}/\mathfrak{m}^2$ en tant que $k$-ev, donc $\dim_k \mathfrak{m}/\mathfrak{m}^2 \leq 2$. Maintenant si $(a,b) \neq 0$, alors $\{x-a, y-b\}$ n'est pas libre modulo $\mathfrak{m}^2$ : en effet, dans $k[x,y]$,
                \begin{align*}
                    3a^2(x-a) - 2b(y-b) = (y - b)^2 - 3a(x - a)^2 - (x - a) ^3 \in \mathfrak{m}^2
                \end{align*}
                et $3a, 2b \neq 0,0$. Maintenant, si $(a,b) = 0$, alors $\mathfrak{m} := \mathfrak{m}_0 = (x,y)$, montrons que $\{x,y\}$ est libre sur $k$ modulo $\mathfrak{m}^2$. Soient $\alpha, \beta \in k$ tels que $\alpha x + \beta y = 0$ mod $\mathfrak{m}^2$. Ainsi $\alpha x + \beta y \in \mathfrak{m}^2 \subseteq k[V]_0$, et donc il existe $P_i/Q_i \in k[V]_0$ avec 
                \begin{align*}
                    \alpha x + \beta y = \frac{P_1}{Q_1}x^2 + \frac{P_2}{Q_2}xy + \frac{P_3}{Q_3}y^2
                \end{align*}
                dans $k[V]_0$. Maintenant $Q_i(0) \ne 0$, donc $\exists Q \in k[x,y]$ tel que $Q(0) \neq 0$ et $Q(\alpha x + \beta y) \in (x^2, xy, y^2)$ dans $k[x,y]$ cette fois. Ainsi $\exists R_i \in k[x,y]$ tels que 
                \begin{align*}
                    Q(X,Y)(\alpha X + \beta Y) - R_1X^2 - R_2XY - R_3Y^2 \in I(V) = (Y^2 - X^3)
                \end{align*}
                dans $k[X,Y]$. Ainsi $Q(X,Y)(\alpha X + \beta Y) \in (X^2, XY, Y^2)$ car $Y^2 - X^3 \in (X^2, XY, Y^2)$. Remarquons qu'en général, si $P \in k[X_1, \cdots, X_n]$, alors $P^0 = 0 \iff P \in (X_1, \cdots, X_n)$, $P^0 = P^1 = 0 \iff P \in (X_1, \cdots, X_n)^2$. Mais alors $Q$ a un terme constant car $Q(0) = 0$, et donc le terme linéaire de $Q(\alpha X + \beta Y)$ doit etre $Q^0(\alpha X + \beta Y)$ et doit être $0$ car $Q(\alpha X + \beta Y) \in (X,Y)^2$. Ainsi $Q^0\alpha, Q^0 \beta = 0 \in k$ et donc $\alpha, \beta = 0$ ce qui prouve la liberté de $\{x,y\}$.
            \end{expl}

            Soit $a \in V$, notons
            \begin{align*}
                H_\alpha := \{x \in \mathbb{A}^n \mid \sum \alpha^j (x_j - a_j) = 0\} \subseteq \mathbb{A}^2
            \end{align*}
            Remarquons alors que $a \in H_\alpha$. Alors on a
            \begin{align*}
                \tau_a(H_\alpha) = \{z \in \mathbb{A}^n \mid \sum \alpha^jz_j = 0\} \subseteq \mathbb{A}^n
            \end{align*}
            et c'est un sev de $k^n$. Ainsi 
            \begin{align*}
                \tau_a(T_a^\mathrm{geom}V) = \{z \in k^n \mid \forall i \in \lcc 1,r \rcc,\, \sum_j \frac{\partial P_i}{\partial X_j}(a) z_x = 0\}
            \end{align*}
            Si $V = V(P_1, \cdots, P_r)$. C'est le noyau de l'application $k^n \to k^r$ correspondant à la matrice jacobienne $d \varphi(a)$, et ainsi $\dim T_a^\mathrm{geom}V = n - rk d \varphi(a)$. \\
            Montrons maintenant que $\tau_a(T_a^\mathrm{geom}V) \simeq T_aV$ en tant que $k$-ev. On peut supposer que $a = 0 \in \mathbb{A}^n_k$, montrons que 
            \begin{align*}
                \Hom_k(T_a^\mathrm{geom}V, k) \simeq \mathfrak{m}/\mathfrak{m}^2
            \end{align*}
            On note $x_i = [X_i] \in k[V]$, $M = (x_1, \cdots, x_n)$.
            \begin{exo}
                $\mathfrak{m}/\mathfrak{m}^2 \simeq M/M^2$ en tant que $k$-ev ($\mathfrak{m}$ est l'image de $M$ par $k[V] \to k[V]_a$ où $M = (x_i - a_i)$) 
            \end{exo}
            Aisni prouver le point précédent reviens a prouver qu'il existe un isom
            \begin{align*}
                M/M^2 \simeq \Hom_k(T_a^\mathrm{geom}V, k) 
            \end{align*}
            Notons ainsi $V = V(P_1, \cdots, P_r)$, $W := T_a^\mathrm{geom}V$. Comme $a = 0 \in V$, alors $P_i(0) = 0$ pour tout $i$. Alors soit
            \begin{align*}
                \begin{array}{cccc}
                    \varphi : & M & \to & \Hom_k(W,k) \\
                    & [P] & \mapsto & P^1_{|W} : x \in W \mapsto P^1(x) \in k \\
                \end{array}
            \end{align*}
            \begin{enumerate}
                \item Montrons que $\varphi$ est bien définie : soient $P,Q \in k[X_1, \cdots, X_n]$ tels que $[P] = [Q]$, alors $P - Q \in I(V)$. Mais écrivons $P,Q = P^1,Q^1 + \cdots + P^d,Q^e$ avec $P^i,Q^j$ homogènes. Alors $P-Q = (P^1 - Q^1) + \cdots$ et $(P - Q)^1 = P^1 - Q^1$. Maintenant $P - Q \in I(V) \Rightarrow P^1 - Q^1 \in I(W)$ et donc $P^1 - Q^1$ est nul sur $W$. Précisons l'implication précédente : si $R \in I(V)$, alors $R^1 \in I(W)$. EN effet, écrivons $R = \sum P_iQ_i$, puis écrivons $P_i = P_i^1 + P_i^2 + \cdots$, $Q_i = Q_i^0 + Q_i^1 + \cdots$. Mais alors $(P_iQ_i)^1 = P_i^1Q_i^0$ donc $R^1 = \sum P_i^1Q_i^0 \in I(W)$.
                \item Montrons que $\varphi$ est surjective. Soit $f : W \to k$ $k$-linéaire. Alors $\exists \tilde f : k^n \to k$ $k$-linéaire telle que $\tilde f_{|W} = f$, puis 
                \begin{align*}
                    \tilde f(x) = \sum \alpha_i x_i
                \end{align*}
                Ainsi $\varphi([\sum \alpha_i X_i]) = \tilde f_{|W} = f$.
                \item Montrons que $\ker \varphi = M^2$ : \cor{flemme}   
            \end{enumerate}
\chapter{Courbes algébriques affines}
    \section{Rappels d'algèbre commutative}
        $k$ corps algébriquement clos. Une courbe est une variété affine de dimension $1$. $a \in C$ dans une courbe est régulier ssi $\dim \mathfrak{m}_a/\mathfrak{m}_a^2 = 1$.
        \begin{defi}
            Un anneau de valuation discrète (DVR) est un anneau local $(A, \mathfrak{m})$ tel que $A$ intègre et principal, mais n'est pas un corps.
        \end{defi}
        En particulier, un tel anneau est
        \begin{enumerate}
            \item Noethérien, car principal,
            \item $\mathfrak{m} = (t)$ (on appelle $t \in A$ une uniformisante). Un tel $t$ est irréductible (car $\mathfrak{m}$ est un idéal premier et $A$ un anneau principal).
            \item Factoriel, car principal,
            \item $\dim A = 1$, car principal.\\
        \end{enumerate}
        \begin{prop}
            Soit $(A, \mathfrak{m})$ un DVR, notons $K = \Frac A$ et $\mathfrak{m} = (t)$, $t \in A$.
            \begin{enumerate}
                \item Les idéaux non nuls de $A$ sont les $\mathfrak{m}^i = (t^i)$, $i \in \mathbb{N}$.
                \item $\bigcap_{i \geq 0} \mathfrak{m}^i = \{0\}$
                \item $\forall x \in K \bs \{0\}$, $\exists ! n \in \mathbb{Z}$ tel que $x = t^nu$, $u \in A^\times$ et l'application
                \begin{align*}
                    \begin{array}{cccc}
                        v : & K \bs \{0\} & \to & \mathbb{Z} \\
                        & x & \mapsto & v(x) = n \\
                    \end{array}
                \end{align*}
                satisfait 
                \begin{align*}
                    \begin{cases}
                        v(xy) = v(x) + v(y) \\
                        v(x + y) \geq \min \{v(x), v(y)\} \\
                    \end{cases}
                \end{align*}
                ($v$ est une valuation sur $K$)
            \end{enumerate}
            De plus,
            \begin{align*}
                &A = \{x \in K \bs \{0\} \mid v(x) \geq 0\} \cup \{0\} \\
                &\mathfrak{m} = \{x \in K \bs \{0\} \mid v(x) > 0\} \cup \{0\} \\
            \end{align*}
        \end{prop}
        \begin{exo}
            Si $v : K \bs 0 \to \mathbb{Z}$ est une valuation d'un corps $K$, alors $A = \{x \in K \bs 0 \mid v(x) \geq 0\} \cup \{0\}$ est un DVR d'idéal maximal $\mathfrak{m} = \{x \in K \bs 0 \mid v(x) > 0\} \cup \{0\}$. \ajout{En effet, c'est un anneau pour les opérations induites par $K$ au vu des équations sur la valuation requises pour que $v$ soit une valuation. Maintenant $A$ est intègre car $K$ l'est, puis considérons $I \subrel{id} A$, } \cor{A finir}
        \end{exo}
        \begin{proof}
            \begin{enumerate}
                \item $A$ factoriel, $t$ irréductible. Pour tout $x \in A \bs 0$, $\exists N \geq 0$ maximal avec la propriété que $t^N \mid x$ dans $A$. Alors $x = t^Ny$, $y \in A^\times$ (vu que $A$ est local donc $A^\times = A \bs \mathfrak{m}$, et $N$ est maximal). Maintenant soit $I \subrel{id} A$ idéal, alors $\exists x \in A$ non nul tq $I = (x)$; Alors on a que $x = t^Ny$ avec $y \in \mathbb{A}^x$, et donc $I = (t^N)$.
                \item On a $0 \subseteq \mathfrak{m}^i \nsubseteq \mathfrak{m}^{i-1} \subseteq A$. Alors soit $x \in \bigcap_{\mathbb{N}} \mathfrak{m}^i$ non nul, on a $x = t^Ny$ pour un $N$ maximal, $y \in A^\times$. Mais $x \in \mathfrak{m}^{N+1}$ donc $t^{N+1} \mid t^Ny$ absurde.
                \item Soit $x \in K \bs 0$. Alors $x = y/z$, $y,z \in A$ et $z \neq 0$. Alors $\exists N,M \in \mathbb{N}$ t.q. $y = t^Nu$, $z = t^Mv$ avec $u,v \in A^\times$. Ainsi $x = t^{N-M}(uv^{-1})$. Maintenant vérifions que la valuation est bien définie : si $x = t^iu = t^jv$, alors ops $i > j$ et alors $t^{i-j} \in \mathbb{A}^\times$, absurde. Anis $i = j$ et $u = v$. Il est finalement facile de voir que $v$ est une valuation.
            \end{enumerate}
        \end{proof}

    \section{Régularité des courbes algébriques}
        Dans l'objectif de caractériser les points réguliers d'une courbe, ainsi que la propriété d'être lisse, faisons quelques rappels d'algèbre commutative :
        \begin{defi} (Entiers sur un anneau, anneau intégralement clos dans son corps des fractions)
            $A$ anneau intègre, $K = \mathrm{Frac} A$. Alors
            \begin{enumerate}
                \item $x \in K$ est entier sur $A$ s'il existe $P \in A[X]$ unitaire tel que $P(x) = 0$ (dans $K$).
                \item $A$ intégralement clos dans $K$ si $x \in K$ entier sur $A$ implique que $x \in A$.
            \end{enumerate}
        \end{defi}
        \begin{expl}
            \begin{enumerate}
                \item $\mathbb{Z}$ est intégralement clos dans $\mathbb{Q}$.
                \item $\mathbb{Z}[i]$ est intégralement clos dans $\mathbb{Q}$.
                \item $A = k[X,Y]/(Y^2 - X^3) = k[x,y]$ n'est pas intégralement clos. Par exemple, $y/x$ est entier sur $A$ car $(y/x)^2 - x = 0$ mais pas dans $A$.
            \end{enumerate}
        \end{expl}
        \begin{theo}
            Soit $A$ un anneau intègre. Alors $A$ est intégralement clos si et seulement si $A_\mathfrak{p}$ est intégralement clos, pour tout $\mathfrak{p} \subrel{prm} A$.
        \end{theo}
        \begin{proof}
            \cor{Reference}
        \end{proof}
        \begin{theo}
            Soit $A$ un anneau local intègre, tel que $\dim A = 1$. Alors $A$ est intégralement clos si et seulement si $A$ est un DVR.
        \end{theo}
        \begin{proof}
            \cor{Référence}
        \end{proof}
        \old{
        \begin{defi}
            \old{$A$ anneau de dedekind si intègre, noeth, intégralement clos, de dimension $1$.}
        \end{defi}
        \begin{expl}
            $\mathbb{Z}$, $\mathbb{Z}[i]$, $k[C]$ pour $C$ une courbe lisse.
        \end{expl}
        }
        Maintenant remarquons que
        \begin{theo}
            $C$ courbe, $a \in C$. Alors $a \in C$ est un point régulier ssi $k[C]_a$ est un DVR.
        \end{theo}
        \begin{proof}
            \cor{A voir}
        \end{proof}
        \begin{theo}
            $C$ courbe affine. $C$ est lisse si et seulement si $k[C]$ est intégralement clos dans $k(C)$.
        \end{theo}
        \begin{proof}
            \ajout{$C$ lisse $\iff \forall a \in C$, $k[C]_a$ est un DVR $\iff \forall a \in C$, $k[C]_a$ est intégralement clos. Maintenant, si $k[C]$ est intégralement clos, alors toutes les localisations de $k[C]$ en un idéal premier sont des anneaux intégralement clos, et en particulier les localisations $k[C]_a$, pour tout $a \in C$. Réciproquement, \cor{Je sais pas haha}}.
        \end{proof}
        \begin{theo}
            $C$ courbe affine.
            \begin{align*}
                Sing(C) = \{a \in C \mid a \text{ singulier} \}
            \end{align*}
            est un ensemble propre de $C$
        \end{theo}
        \begin{proof}
            \cor{A voir}
        \end{proof}
        \begin{exo}
            $Sing(C)$ est un sous-ensemble algébrique.
        \end{exo}
        \begin{prop}
            \label{prop412}
            Soit $C$ une courbe affine. Les sous-ensemblers propres de $C$ sont finis.
        \end{prop}
        \begin{coro}
            Soit $C$ une courbe algébrique, alors$|Sing(C)| < \infty$
        \end{coro}
        \begin{proof} (\ref{prop412})
            Soit $V \subseteq C \subseteq \mathbb{A}^n$ ensemble algébrique propre. OPS $V$ irréductible(sinon on considère une composante de $V$). Alors $V \nsubseteq C$ implique que $I(C) \nsubseteq I(V) \subseteq k[X_1, \cdots, X_n]$. Montrons que $I(V) = \mathfrak{m}_a$, $a \in \mathbb{A}^n$. Maintenant
            \begin{align*}
                0 \subseteq I(V)/I(C) \subseteq k[C]
            \end{align*}
            est un idéal premier. Mais $\dim k[C] = 1$, donc $I(V)/I(C)$ est maximal, donc $I(V)$ maximal. Ainsi $I(V) = \mathfrak{m}_a$ car $k = \bar k$, donc $V$ doit être un point.
        \end{proof}